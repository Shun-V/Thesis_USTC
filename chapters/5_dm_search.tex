% !TeX root = ../main.tex


\chapter{\texorpdfstring{Search for Dark Matter with $ll + E^{miss}_{T}$ Final State}{Search for Dark Matter with ll + MET Final State}}
\label{chap:monoZ}

\section{Introduction}
\label{sec:monoZ_intro}

The existence of Dark Matter (DM) is one of the most compelling evidences for physics Beyond the Standard Model (BSM). While astrophysical observations, such as galaxy rotation curves and the Cosmic Microwave Background anisotropy, provide strong gravitational evidence for DM, its particle nature remains unknown. The Large Hadron Collider (LHC) offers a unique opportunity to produce DM particles in high-energy proton-proton collisions. Since DM particles are electrically neutral and stable, they do not interact with the ATLAS detector and escape undetected, manifesting only as an imbalance in the transverse momentum, denoted as missing transverse momentum ($E_{\mathrm{T}}^{\text{miss}}$).

To identify such events, the DM particles must be produced in association with a visible Standard Model particle ("X"), resulting in a "Mono-X" signature. This chapter presents a search for Dark Matter produced in association with a $Z$ boson, where the $Z$ boson decays into a pair of charged leptons ($\ell^+\ell^-$, with $\ell = e, \mu$). This is referred to as the \textit{Mono-Z} channel.

This analysis builds directly upon the measurement of the Standard Model $ZZ \to \ell\ell\nu\nu$ process presented in Chapter~\ref{chap:zz_analysis}. Both analyses share the same final state topology: two charged leptons and significant missing transverse momentum. However, they probe different kinematic regimes and interpretations:
\begin{itemize}
    \item In the previous chapter, the $ZZ \to \ell\ell\nu\nu$ process was the \textbf{signal}, and the focus was on measuring the inclusive production cross-section in the precision regime.
    \item In this chapter, the $ZZ \to \ell\ell\nu\nu$ process represents the dominant, irreducible \textbf{background}. The search focuses on the high-$E_{\mathrm{T}}^{\text{miss}}$ tails where BSM signals are expected to appear over the Standard Model expectation.
\end{itemize}

The search is interpreted within the context of Simplified Model scenarios involving vector or axial-vector mediators, as well as more complex models such as Two-Higgs-Doublet Models with an additional pseudoscalar mediator (2HDM+$a$). These models predict an excess of events with large missing transverse momentum, distinct from the spectra of the Standard Model backgrounds.

The structure of this chapter is as follows. Section~\ref{sec:monoZ_samples} details the simulated signal samples used to model DM production. Section~\ref{sec:monoZ_selection} describes the event selection criteria, optimized to suppress the $Z+\text{jets}$ and diboson backgrounds while retaining sensitivity to DM signals. The estimation of background contributions, particularly the data-driven techniques used to constrain the dominant $ZZ$ and $WZ$ backgrounds, is discussed in Section~\ref{sec:monoZ_bkg}. Section~\ref{sec:monoZ_syst} outlines the systematic uncertainties. Finally, the statistical interpretation and the resulting exclusion limits on DM mediator masses are presented in Section~\ref{sec:monoZ_results}.

\clearpage
%\section{Data and Monte Carlo Samples}
\section{Data and MC Samples}
\label{sec:monoZ_samples}

This search utilizes the full ATLAS Run 2 dataset ($\sqrt{s}=13$ TeV), corresponding to an integrated luminosity of 139 fb$^{-1}$. Monte Carlo (MC) simulations are used to model the detector response and signal acceptance for both the Dark Matter (DM) signals and the Standard Model (SM) backgrounds. All simulated events were processed through the full ATLAS detector simulation based on \textsc{Geant4} or a fast simulation where indicated, and reconstructed using the same software as the data. Pile-up effects (multiple $pp$ interactions in the same or neighbouring bunch crossings) were modelled by overlaying simulated minimum-bias events generated with \textsc{Pythia} 8 \cite{Sjostrand:2007gs}.

\subsection{Signal Samples}
\label{subsec:signal_samples}

Three distinct classes of signal models are simulated to interpret the results of this search: the invisible decay of the Higgs boson, simplified DM models with spin-1 mediators, and extended Two-Higgs-Doublet Models (2HDM+$a$).

\paragraph{Higgs Boson to Invisible:}
The Standard Model $ZH$ production is simulated using the \textsc{Powheg Box} v2 generator \cite{Nason:2009xu} at Next-to-Leading Order (NLO) accuracy in QCD. Both quark-induced ($q\bar{q} \to ZH$) and gluon-induced ($gg \to ZH$) processes are generated. The events are interfaced with \textsc{Pythia} 8 \cite{Sjostrand:2007gs} for the simulation of the parton shower, hadronisation, and underlying event, using the AZNLO tune and the NNPDF3.0 PDF set. The samples are normalised to cross-sections calculated at Next-to-Next-to-Leading Order (NNLO) in QCD with NLO Electroweak (EW) corrections \cite{deFlorian:2016spz}. The Higgs boson mass is set to $m_H = 125$ GeV, and it is required to decay to four neutrinos to mimic the invisible signature.

\paragraph{Simplified Dark Matter Models:}
Signals for vector and axial-vector mediators decaying into Dirac Dark Matter particles ($\chi$) are generated using \textsc{MadGraph5\_aMC@NLO} v2.6.2 \cite{Alwall:2014hca} at NLO accuracy. The generation uses the \texttt{DMsimp\_s\_spin1} model implementation. Events are interfaced with \textsc{Pythia} 8 using the A14 set of tuned parameters (tune) and the NNPDF3.0 PDF set. The coupling of the mediator to the DM particles ($g_{\chi}$) is set to 1.0, and the universal coupling to quarks ($g_{q}$) is set to 0.25. A grid of samples is produced spanning the mediator mass ($m_{med}$) and DM mass ($m_{\chi}$) plane.

\paragraph{2HDM+$a$ Models:}
The 2HDM+$a$ signal samples are generated using \textsc{MadGraph5\_aMC@NLO} \cite{Alwall:2014hca} at Leading Order (LO) accuracy. Both gluon-gluon fusion and $b$-quark associated production mechanisms are simulated. The showering and hadronisation are performed by \textsc{Pythia} 8 with the A14 tune. The kinematics of the signal are scanned across four parameters: the pseudoscalar masses $m_a$ and $m_A$, the ratio of the vacuum expectation values of the two Higgs doublets $\tan\beta$, and the mixing angle $\sin\theta$.

\subsection{Background Samples}
\label{subsec:bkg_samples}

The dominant background contributions arise from diboson ($ZZ, WZ$) production and $Z+\text{jets}$ processes. Other contributions include top-quark pair production ($t\bar{t}$), single top, and triboson processes.

\paragraph{Diboson ($ZZ, WZ, WW$):}
The $q\bar{q}$-initiated diboson processes ($ZZ \to \ell\ell\nu\nu$, $ZZ \to 4\ell$, $WZ \to \ell\nu\ell\ell$) are simulated using the \textsc{Sherpa} 2.2.2 generator \cite{Bothmann:2019yzt}. Matrix elements are calculated at NLO accuracy for up to one additional parton and at LO accuracy for up to three additional partons. The loop-induced process $gg \to ZZ$ is simulated at LO accuracy using \textsc{Sherpa} 2.2.2 with up to one additional parton. All diboson samples use the NNPDF3.0NNLO PDF set.

\paragraph{$Z+\text{jets}$:}
The production of a $Z$ boson in association with jets is modeled using \textsc{Sherpa} 2.2.1 \cite{Bothmann:2019yzt}. Matrix elements are calculated at NLO for up to two partons and at LO for up to four partons, using the Comix and OpenLoops libraries \cite{Cascioli:2011va}. The samples are normalised to a Next-to-Next-to-Leading Order (NNLO) prediction. This background is particularly relevant for events with fake $E_{\mathrm{T}}^{\text{miss}}$ arising from jet mis-measurement.

\paragraph{Top Quark Processes:}
Top-quark pair ($t\bar{t}$) and single-top production (Wt, s-channel, and t-channel) are simulated using \textsc{Powheg Box} v2 \cite{Nason:2009xu, Hamilton:2012rf} interfaced with \textsc{Pythia} 8 \cite{Sjostrand:2007gs}. The A14 tune and NNPDF3.0 PDF set are used. The $t\bar{t}$ samples are normalised to the NNLO cross-section including soft-gluon resummation to next-to-next-to-leading-log (NNLL) accuracy.

\paragraph{Other Backgrounds:}
Rare backgrounds such as triboson ($VVV$) and top-pair production in association with a vector boson ($t\bar{t}V$) are simulated using \textsc{Sherpa} 2.2.2 \cite{Bothmann:2019yzt} and \textsc{MadGraph5\_aMC@NLO} \cite{Alwall:2014hca}, respectively. While their contribution to the signal region is minor ($<1\%$), they are included for completeness in the background estimation.


\clearpage
%\section{Event Selection}
\section{Event Selection}
\label{sec:monoZ_selection}

The analysis targets events with a leptonic $Z$ boson candidate and large missing transverse momentum. The selection is optimized to maximize the significance of DM signals against the Standard Model background, primarily $Z$+jets and diboson processes.

\subsection{Object Selection}
\label{subsec:object_sel}

Candidate events are required to have a primary vertex with at least two associated tracks with $p_{\mathrm{T}} > 1$ GeV. The physics objects used in this analysis—electrons, muons, jets, and missing transverse momentum—are defined as follows.

\paragraph{Electrons:}
Electrons are reconstructed from energy clusters in the electromagnetic calorimeter matched to tracks in the inner detector~\cite{PERF-2017-01}. They are required to be within the acceptance region $|\eta| < 2.47$ and satisfy $p_{\mathrm{T}} > 7$ GeV. For the final signal selection, electrons must satisfy "Medium" likelihood-based identification criteria and "FixedCutPflowLoose" isolation requirements.

\paragraph{Muons:}
Muons are reconstructed by combining tracks in the inner detector with tracks in the muon spectrometer~\cite{PERF-2015-10}. They are required to satisfy $p_{\mathrm{T}} > 7$ GeV and $|\eta| < 2.5$. Signal muons must satisfy "Medium" identification criteria and "FixedCutPflowLoose" isolation requirements.

\paragraph{Jets:}
Jets are reconstructed using the anti-$k_t$ algorithm with a radius parameter $R=0.4$~\cite{PERF-2016-04}. They are required to have $p_{\mathrm{T}} > 30$ GeV and $|\eta| < 4.5$. To suppress jets originating from pile-up interactions, jets with $p_{\mathrm{T}} < 60$ GeV and $|\eta| < 2.4$ must satisfy the Jet Vertex Tagger (JVT) requirement (JVT $> 0.5$).

\paragraph{Missing Transverse Momentum:}
The missing transverse momentum, $\vec{E}_{\mathrm{T}}^{\text{miss}}$, is calculated as the negative vector sum of the transverse momenta of all calibrated physics objects (electrons, muons, jets) and a "soft term" calculated from tracks associated with the primary vertex but not matched to any reconstructed object~\cite{ATL-PHYS-PUB-2015-027}. The magnitude is denoted as $E_{\mathrm{T}}^{\text{miss}}$. This analysis uses the "Tight" $E_{\mathrm{T}}^{\text{miss}}$ working point to improve the rejection of fake $E_{\mathrm{T}}^{\text{miss}}$ from pile-up jets, as recommended by the Jet/EtMiss group.

\subsection{Event Pre-selection}
\label{subsec:pre_selection}

Events are selected using single-electron or single-muon triggers. The baseline pre-selection criteria are designed to identify a $Z$ boson candidate while ensuring high efficiency:

\begin{itemize}
    \item \textbf{Leptons:} Exactly two same-flavor, opposite-sign (SFOS) leptons ($\ell = e, \mu$) are required. The leading lepton must have $p_{\mathrm{T}} > 30$ GeV, and the sub-leading lepton must have $p_{\mathrm{T}} > 20$ GeV.
    \item \textbf{Z-mass Window:} The invariant mass of the lepton pair must fall within $76 < m_{\ell\ell} < 106$ GeV, consistent with the $Z$ boson mass to suppress non-resonant backgrounds.
    \item \textbf{Third Lepton Veto:} Events containing a third lepton with $p_{\mathrm{T}} > 7$ GeV (satisfying "Loose" identification) are vetoed to suppress the $WZ$ background.
    \item \textbf{$b$-jet Veto:} Events containing any $b$-tagged jets ($p_{\mathrm{T}} > 20$ GeV, $|\eta| < 2.5$) are vetoed to suppress top-quark backgrounds ($t\bar{t}$, $Wt$). The MV2c10 tagger is used at the 85\% efficiency working point.
    \item \textbf{$E_{\mathrm{T}}^{\text{miss}}$ Threshold:} A loose requirement of $E_{\mathrm{T}}^{\text{miss}} > 70$ GeV is applied at the pre-selection stage.
\end{itemize}

\subsection{Signal Region Definition}
\label{subsec:sr_def}

To further enhance the sensitivity to Dark Matter signals, specific kinematic requirements are applied to define the Signal Region (SR). The optimization was performed using an iterative BDT approach and significance calculations:

\begin{itemize}
    \item \textbf{$E_{\mathrm{T}}^{\text{miss}} > 90$ GeV:} A tighter $E_{\mathrm{T}}^{\text{miss}}$ cut is applied to reduce the $Z+\text{jets}$ background, where $E_{\mathrm{T}}^{\text{miss}}$ often arises from instrumental mis-measurement.
    \item \textbf{$E_{\mathrm{T}}^{\text{miss}}$-significance $> 9$:} The object-based $E_{\mathrm{T}}^{\text{miss}}$-significance ($S$) is a powerful discriminant against fake missing energy. It assesses whether the observed $E_{\mathrm{T}}^{\text{miss}}$ is consistent with the resolution of the reconstructed objects. A cut of $S > 9$ significantly suppresses the $Z+\text{jets}$ contribution. The significance is defined as~\cite{ATL-PHYS-PUB-2020-025}:
    \begin{equation}
        Z = \sqrt{2\left((s+b)\ln\left[\frac{(s+b)(b+\sigma_{b}^{2})}{b^{2}+(s+b)\sigma_{b}^{2}}\right] - \frac{b^{2}}{\sigma_{b}^{2}}\ln\left[1+\frac{s\sigma_{b}^{2}}{b(b+\sigma_{b}^{2})}\right]\right)}
    \end{equation}
    \item \textbf{$\Delta R_{\ell\ell} < 1.8$:} The angular separation between the two leptons is required to be small. This selects events where the $Z$ boson is boosted, recoiling against the invisible DM system.
\end{itemize}

For the final statistical interpretation of the Mono-$Z$ channel, the transverse mass of the $ZZ$ system, $m_T^{ZZ}$, is used as the discriminant variable. It is defined as:
\begin{equation}
    m_{T}^{ZZ} = \sqrt{\left(\sqrt{m_{Z}^{2} + |p_{T}^{\ell\ell}|^{2}} + \sqrt{m_{Z}^{2} + |E_{T}^{\text{miss}}|^{2}}\right)^{2} - \left|\vec{p}_{T}^{\,\ell\ell} + \vec{E}_{T}^{\text{miss}}\right|^{2}}
\end{equation}
This variable effectively separates the signal (which peaks at higher values correlated with the mediator mass) from the SM backgrounds. Only events with $m_T^{ZZ} > 200$ GeV are included in the fit.


\clearpage
%\section{Background Estimation}
\section{Backgrounds and }
\label{sec:monoZ_bkg}


\clearpage
%\section{Systematic Uncertainty}
\input{chapters/5_4_syst.tex}


\clearpage
%\section{Unfolding}
\section{Statistical Interpretation}
\label{sec:monoZ_stats}


\clearpage
%\section{Results}
\section{Results}
\label{sec:monoZ_results}