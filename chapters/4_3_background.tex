\section{Background Estimation}
\label{sec:background_estimate}

In the measurement of the \( ZZ \to \ell^+\ell^-\nu\bar{\nu} \) cross-section, several background processes contribute. These can be categorized as follows:

\begin{itemize}

    % --- FIRST TOP-LEVEL ITEM ---
    \item \textbf{Irreducible Backgrounds}
    \begin{itemize}
        \item \textbf{Diboson processes:}
        \begin{itemize}
            \item \( WW \to \ell\nu\ell\nu \): The two neutrinos contribute to the MET signature, faking the \( ZZ \to \ell\ell\nu\nu \) process.
            \item \( WZ \to \ell\nu\ell\ell \): If one lepton from the \( Z \) is missing, this process could fake the signal.
            \item \( ZZ \to \ell\ell\ell\ell \): If two leptons escape detection, this process contributes as a background.
        \end{itemize}
    \end{itemize}

    % --- SECOND TOP-LEVEL ITEM ---
    \item \textbf{Reducible Backgrounds}
    \begin{itemize}
        \item \textbf{Top-quark backgrounds:}
        \begin{itemize}
            % THIS IS THE LINE THAT HAD THE ERROR
            \item \( t\bar{t} \): If b-jets in the \( t\bar{t} \) final states are not tagged properly, it may fake the signal.
            \item Single top \( (tW) \): Can contribute similarly.
        \end{itemize}
        \item \textbf{Drell-Yan + jets} (\( Z \to \ell\ell \) + jets): Jet energy mismeasurement can lead to artificial MET.
        \item \textbf{W + jets}: A single \( W \) boson decaying leptonically with jets can mimic the signal region due to fake MET.
    \end{itemize}

\end{itemize}

To achieve precise cross-section measurements, these backgrounds must be carefully estimated and subtracted using control regions. Considering the final states of the backgrounds, they can be categorized as following Control Regions(CR).

\begin{itemize}
    \item \textbf{Non-Resonant Backgrounds ($e\mu$ flavor symmetry)}
    These processes decay via two $W$ bosons (or $\tau$ leptons) rather than a $Z$ boson, resulting in uncorrelated lepton flavors. They are further categorized by $b$-jet multiplicity to separate Top processes from Diboson processes:
    \begin{itemize}
        \item \textbf{0 $b$-jet Category (Diboson Enriched):}
        \begin{itemize}
            \item $WW \to \ell\nu\ell\nu$: The dominant contribution in the 0 $b$-tag phase space. The neutrinos create genuine MET.
            \item $t\bar{t}$ / Single Top ($Wt$): Top events contribute to this category only if the $b$-jets fail to be identified (tagging inefficiency) or fall outside detector acceptance.
        \end{itemize}
        \item \textbf{$\geq 1$ $b$-jet Category (Top Enriched):}
        \begin{itemize}
             \item $t\bar{t} \to WbWb$ and $Wt$: The dominant contribution when at least one $b$-jet is identified. These events contain real MET from the $W$ decays.
        \end{itemize}
    \end{itemize}

    \item \textbf{3-Lepton Backgrounds ($WZ$)}
    \begin{itemize}
        \item $WZ \to \ell\nu\ell\ell$: This is an irreducible background if one of the three leptons is lost (fails reconstruction) or falls outside the detector acceptance, mimicking the $2\ell + E_{miss}^{T}$ signature.
    \end{itemize}
    
    \item \textbf{Z+Jets Backgrounds (Instrumental MET)}
    \begin{itemize}
        \item \textbf{Drell-Yan + jets} ($Z/\gamma^* \to \ell\ell$ + jets): The dominant reducible background. It contains real leptons from the $Z$ boson, but the MET is artificial ("fake"), caused by jet energy mismeasurement or pile-up interactions.
    \end{itemize}
   
    \item \textbf{Fake Lepton / Instrumental Backgrounds}
    \begin{itemize}
        \item \textbf{W + jets}: A process with one real lepton from a $W$ boson and a jet misidentified as a second lepton (fake lepton), creating a fake $2\ell$ signature.
        \item \textbf{Fake MET}: Detector anomalies, cosmic-ray muons, or beam halo effects that produce large artificial missing energy.
    \end{itemize}
\end{itemize}

To accurately estimate these background contributions, control regions (CRs) are defined, each enriched with a dominant background category. A simultaneous fit is performed across all CRs and the signal region to constrain both the yields and shapes of these backgrounds. This fit takes event distributions as input and derives scale factors to adjust Monte Carlo (MC) predictions to match data in the CRs. The scaled MC predictions then provide the background estimates in the signal region, ensuring a reliable determination of the differential cross-section.

To further refine background estimation, each control region is designed to enhance a specific type of background, allowing us to isolate and constrain individual contributions effectively. The simultaneous fit across all CRs and the signal region ensures a consistent estimation of background yields and shapes. By utilizing event distributions, this fit derives scale factors that adjust MC predictions to align with data in all CRs. These scaled MC predictions then serve as the background estimates in the signal region, providing a foundation for an accurate differential cross-section measurement.

The following sections describe the estimation strategy for each control region in detail.



\subsection{Non-Resonant with/without 1 b-jet)}
\label{sec:bkg_emuCR}

Considering the high percentage of the non-resonant backgrounds, it would be better if Control Regions are well defined to control the backgrounds from MC estimations. Therefore, 2 Control Regions are defined to further estimate the yields from those processes. 

One is the non-resonant background, whcih requires Opposite Flavor Opposite Sign(OFOS) leptons in the final state. For example, in the \( WW \to \ell\nu\ell\nu \) process, the observable final state could be either Same Flavor Opposite Sign(SFOS) leptons, which fakes the true signals, or Opposite Flavor Opposite Sign(OFOS) leptons. Considering that there is an equal chance for W boson decaying to electron or muon or tau, the yield ratio between SFOS and OFOS final states would be 1:2. Since the OFOS events are well seperated from the SFOS, this would be a good approach to estimate the background yields in Signal Region(SR) using the CR events. 

However, since the final state from top quark pair or single top decay contains 1 or 2 b jets, the observed final state could contain 1 or more b jets. Therefore define a CR with 1 b jet to further estimate the yield from the top backgrounds. This non-resonant with a b-jet CR defination is the same as the common non-resonant CR, except the b-jet number requirement. 

Since this is aimming to estimate the yields in the SR, the non-resonant background definition is the same as the SR, except requiring the OFOS events. the detailed definations of non-resonant with/without 1 b-jet are shown in the following table. 



This is a table TBA
including the yields from different backgrounds. 














\subsection{3-Lepton Control Region (3l CR)}
\label{sec:bkg_3lCR}

The $WZ \to \ell\nu\ell\ell$ process constitutes a major irreducible background for the $ZZ \to \ell\ell\nu\nu$ analysis. This process mimics the signal topology if one of the three leptons fails to be reconstructed or falls outside the detector acceptance, resulting in apparent missing transverse energy. To constrain the normalization and modeling of the $WZ$ background, an inclusive 3-lepton Control Region (3l CR) is defined.

\subsubsection{Definition and Event Selection}

The selection criteria for the 3l CR are summarized in Table~\ref{table:3lCR_definition_sim}. The primary requirement distinguishing this region from the Signal Region (SR) is the lepton multiplicity; we require the presence of more than two leptons ($n_{leptons} > 2$). This condition ensures strictly orthogonal phase spaces between the control and signal regions, preventing any event overlap.

Apart from the lepton multiplicity, the kinematic cuts are chosen to closely mirror those of the Signal Region. This strategy ensures that the control region probes a phase space with similar kinematic characteristics to the signal region, thereby minimizing systematic uncertainties associated with extrapolating the normalization factors (scaling factors) from the CR to the SR. However, certain cuts are kept slightly looser than in the SR to retain higher statistics, reducing the statistical uncertainty on the derived scaling factors.

Specifically, the selection requires an Opposite Sign Same Flavor (OSSF) lepton pair consistent with the $Z$ boson mass ($80 < M_{\ell\ell} < 100$ GeV). To select a topology consistent with the high-$p_T$ $Z$ boson signal, a large missing transverse energy cut ($E_{miss}^T > 70$ GeV) and a boost requirement on the leptons ($\Delta R_{\ell\ell} < 2$) are applied.

Furthermore, a transverse mass cut is applied to the third lepton and the MET, denoted as $m_{TW} > 30$ GeV. This cut is introduced specifically to suppress the contribution from $Z$+jets events where a jet is misidentified as a lepton. By requiring a non-zero $m_{TW}$, we reduce the contamination from non-prompt leptons, ensuring high purity of the $WZ$ process.

\begin{table}[h!]
\centering
\begin{tabular}{|l|c|}
\hline
\textbf{Variable} & \textbf{Cut Value} \\ 
\hline 
$n_{leptons}$ & $> 2$ \\
\hline 
$M_{\ell\ell}$ & $80 < M_{\ell\ell} < 100$ GeV \\
\hline 
$E_{miss}^T$ & $> 70$ GeV \\
\hline 
$\Delta R_{\ell\ell}$ & $< 2$ \\ 
\hline 
$\Delta \varphi(E_{miss}^T, Z)$ & $> 2.2 $ \\
\hline 
$E_{miss}^T / H_T$ & $> 0.3$ \\
\hline 
Heavy Flavor & b-jet veto \\
\hline 
$m_{TW}$ & $> 30$ GeV \\ 
\hline
\end{tabular}
\caption{Definition of the inclusive 3-lepton Control Region (3l CR). The kinematic selections mimic the Signal Region to ensure phase-space compatibility, while the lepton multiplicity ensures orthogonality.}
\label{table:3lCR_definition_sim}
\end{table}
\FloatBarrier

\subsubsection{Composition and Yields}

The pre-fit event yields in the inclusive 3l Control Region are presented in Table~\ref{table:3lCR_unscaled}. The region is dominated by the $WZ$ process, which accounts for approximately 90\% of the total background expectation, confirming the high purity of this control region.

The data generally shows good agreement with the Monte Carlo predictions even before the simultaneous fit. The ratio of Data to MC is found to be $1.03 \pm 0.02$ for the inclusive channel. The contamination from the signal process ($ZZ \to 2\ell2\nu$) in this region is negligible ($\approx 0.12\%$ relative to the $WZ$ yield), ensuring that the normalization of the background can be constrained without significant bias from the signal.

\begin{table}[htbp]
\centering
\resizebox{\textwidth}{!}{%
\begin{tabular}{|l|c|c|c|c|c|}
\hline
\textbf{Process} & \textbf{All Channels} & \textbf{$\mu\mu\mu$} & \textbf{$\mu\mu e$} & \textbf{$\mu e e$} & \textbf{$eee$} \\ \hline
\textbf{Data} & \textbf{2409.0 $\pm$ 49.08} & 639.0 $\pm$ 25.28 & 607.0 $\pm$ 24.64 & 553.0 $\pm$ 23.52 & 610.0 $\pm$ 24.7 \\ \hline
\hline
EWK & 0.1 $\pm$ 0.01 & 0.01 $\pm$ 0.0 & 0.05 $\pm$ 0.01 & 0.04 $\pm$ 0.01 & 0.01 $\pm$ 0.0 \\ \hline
QCD & 2.4 $\pm$ 0.6 & 0.1 $\pm$ 0.13 & 1.47 $\pm$ 0.47 & 0.84 $\pm$ 0.31 & -0.01 $\pm$ 0.17 \\ \hline
\textbf{WZ} & \textbf{2106.3 $\pm$ 11.18} & 582.12 $\pm$ 5.83 & 536.64 $\pm$ 5.71 & 467.68 $\pm$ 5.55 & 519.85 $\pm$ 5.26 \\ \hline
Z+jets & 86.57 $\pm$ 11.0 & 20.56 $\pm$ 5.88 & 20.38 $\pm$ 5.77 & 28.37 $\pm$ 5.1 & 17.25 $\pm$ 5.21 \\ \hline
top, $t\bar{t}V$, $Wt$ & 57.42 $\pm$ 1.83 & 9.19 $\pm$ 0.68 & 16.2 $\pm$ 0.97 & 18.76 $\pm$ 1.0 & 13.27 $\pm$ 0.97 \\ \hline
WW & 0.65 $\pm$ 0.14 & 0.03 $\pm$ 0.03 & 0.21 $\pm$ 0.08 & 0.22 $\pm$ 0.08 & 0.2 $\pm$ 0.08 \\ \hline
Rare (4l, VVV, etc.) & 90.79 $\pm$ 1.35 & 31.86 $\pm$ 1.12 & 15.62 $\pm$ 0.35 & 14.73 $\pm$ 0.35 & 28.58 $\pm$ 0.58 \\ \hline
ZZ & 2.5 $\pm$ 0.6 & 0.11 $\pm$ 0.0 & 1.52 $\pm$ 0.01 & 0.88 $\pm$ 0.01 & -0.01 $\pm$ 0.0 \\ \hline
\hline
\textbf{Total Bkg} & \textbf{2341.73 $\pm$ 15.85} & 643.85 $\pm$ 8.38 & 590.53 $\pm$ 8.2 & 530.61 $\pm$ 7.62 & 579.14 $\pm$ 7.49 \\ \hline
\hline
Data/MC & 1.03 $\pm$ 0.02 & 0.99 $\pm$ 0.04 & 1.03 $\pm$ 0.04 & 1.04 $\pm$ 0.05 & 1.05 $\pm$ 0.04 \\ \hline
$\frac{\text{Data-nonWZ}}{\text{WZ}}$ & 1.03 $\pm$ 0.03 & 0.99 $\pm$ 0.05 & 1.03 $\pm$ 0.05 & 1.05 $\pm$ 0.06 & 1.06 $\pm$ 0.05 \\ \hline
Signal/WZ (\%) & 0.12 $\pm$ 0 & 0.02 $\pm$ 0 & 0.28 $\pm$ 0 & 0.19 $\pm$ 0 & -0.0 $\pm$ 0 \\ \hline
\end{tabular}%
}
\caption{Pre-fit event yields in the inclusive 3l Control Region. The "Rare" category includes $4\ell$, $\ell\ell qq$, $VVV$, $W$+jets, and $Z\tau\tau$. The region is highly pure in $WZ$ events.}
\label{table:3lCR_unscaled}
\end{table}
\FloatBarrier












\subsection{Z+jets Control Region (Z+jets CR)}
\label{sec:bkg_ZjetsCR}

The estimation of the $Z$+jets background presents a unique challenge in the $ZZ \to \ell\ell\nu\nu$ analysis. Unlike the irreducible diboson backgrounds, the missing transverse energy in $Z$+jets events is instrumental ("fake"), arising primarily from jet energy mismeasurement, pile-up interactions, and detector resolution effects rather than from genuine neutrinos. Consequently, the modeling of $E_{miss}^T$ in Monte Carlo simulations is highly sensitive to the accuracy of the detector response, jet energy corrections, and track reconstruction.

\subsubsection{Definition and Event Selection}

To constrain this background using data, a Control Region must be defined that is rich in $Z$+jets events while maintaining orthogonality to the Signal Region. This creates a tension in the selection logic:
\begin{itemize}
    \item \textbf{Similarity:} The CR should probe high $E_{miss}^T$ and high $E_{miss}^T/H_T$ values to resemble the kinematic phase space of the Signal Region.
    \item \textbf{Purity:} The CR must have minimal contamination from true $ZZ \to 2\ell2\nu$ signal events to ensure independent background estimation.
    \item \textbf{Orthogonality:} The specific cut values must ensure no overlap with the Signal Region.
\end{itemize}

To satisfy these requirements, the $Z$+jets CR is defined using a two-dimensional selection in the ($E_{miss}^T$, $E_{miss}^T/H_T$) plane, selecting events that fall just outside the signal boundaries. The specific selection logic is:
\begin{equation}
\left( 70 < E_{miss}^T < 80 \text{ GeV} \land \frac{E_{miss}^T}{H_T} > 0.3 \right) 
\lor 
\left( E_{miss}^T > 80 \text{ GeV} \land 0.3 < \frac{E_{miss}^T}{H_T} < 0.45 \right)
\end{equation}
This definition captures the "tail" of the $Z$+jets distribution that is kinematically closest to the signal. The geometric relationship between the Signal Region and the $Z$+jets Control Region in this phase space is illustrated in Figure~\ref{fig:Zjets_Control_Region}.

%\begin{figure}[htbp]
%  \centering
%  \includegraphics[width=0.6\textwidth]{figures/Sim_fit/Zjets_Control_Region.png}
%  \caption{Illustration of the inclusive phase-space in the ($E_{miss}^T$, $E_{miss}^T/H_T$) plane. The $Z$+jets Control Region is defined to be adjacent to the Signal Region to constrain the fake MET tail while maintaining orthogonality.}
%  \label{fig:Zjets_Control_Region}
%\end{figure}



All other kinematic selections, such as the dilepton mass window ($80 < M_{\ell\ell} < 100$ GeV), lepton separation ($\Delta R_{\ell\ell} < 1.8$), and azimuthal separation ($\Delta \varphi(E_{miss}^T, Z) > 2.2$), are kept identical to the Signal Region to minimize extrapolation uncertainties. A summary of the cuts is provided in Table~\ref{table:Zjets_definition}.


\begin{table}[htbp]
\centering
\begin{tabular}{|l|c|}
\hline
\textbf{Variable} & \textbf{Cut Value} \\ 
\hline 
Dilepton Mass & $80 < M_{\ell\ell} < 100$ GeV \\
\hline 
\multirow{2}{*}{MET / Significance} & ($70 < E_{miss}^T < 80$ GeV \textbf{and} $E_{miss}^T / H_T > 0.3$) \\
& \textbf{OR} ($E_{miss}^T > 80$ GeV \textbf{and} $0.3 < E_{miss}^T / H_T < 0.45$) \\
\hline 
Lepton separation & $\Delta R_{\ell\ell} < 1.8$ \\ 
\hline 
Azimuthal separation & $\Delta \varphi(E_{miss}^T, Z) > 2.2 $ \\
\hline 
Heavy Flavor & b-jet veto \\
\hline
\end{tabular}
\caption{Definition of the inclusive $Z$+jets Control Region. The region targets the phase space dominated by instrumental MET while remaining orthogonal to the signal region.}
\label{table:Zjets_definition}
\end{table}
\FloatBarrier

\subsubsection{Categorization by Jet Multiplicity}

Studies of the $Z$+jets modeling revealed significant shape differences between Data and Monte Carlo in key distributions used for unfolding. Since the fake MET originates largely from jet mismeasurements, the kinematics of the event are strongly correlated with the number of hadronic jets. 

It was determined that a single global scaling factor was insufficient to correct the $Z$+jets prediction across the entire phase space. Therefore, the control region is split into three sub-regions based on jet multiplicity ($n_{jets}$):
\begin{itemize}
    \item $n_{jets} = 0$
    \item $n_{jets} = 1$
    \item $n_{jets} \geq 2$
\end{itemize}
Independent scaling factors are derived for each of these bins during the simultaneous fit. While a finer binning was investigated (separating $n_{jets}=2$ and $n_{jets} \geq 3$), results indicated that the scaling behavior was consistent for higher jet multiplicities, justifying the merged $\geq 2$ bin.

\subsubsection{Composition and Yields}

The pre-fit event yields for the inclusive $Z$+jets Control Region are detailed in Table~\ref{table:Zjets_unscaled}. The region is dominated by $Z$+jets events, but shows a significant normalization discrepancy pre-fit, with Data exceeding the MC prediction by approximately 18\% in the inclusive selection. This reinforces the necessity of the data-driven scaling factors derived from this region. The signal contamination is controlled at the level of $\approx 4\%$.

\begin{table}[htbp]
\centering
\resizebox{\textwidth}{!}{%
\begin{tabular}{|l|c|c|c|}
\hline
\textbf{Process} & \textbf{ee and $\mu\mu$} & \textbf{ee} & \textbf{$\mu\mu$} \\ \hline
\textbf{Data} & \textbf{14193.0 $\pm$ 119.13} & 6841.0 $\pm$ 82.71 & 7352.0 $\pm$ 85.74 \\ \hline
\hline
EWK & 11.62 $\pm$ 0.1 & 5.7 $\pm$ 0.07 & 5.92 $\pm$ 0.07 \\ \hline
QCD & 397.1 $\pm$ 5.98 & 183.28 $\pm$ 3.99 & 213.82 $\pm$ 4.45 \\ \hline
WZ & 636.46 $\pm$ 7.49 & 301.33 $\pm$ 3.88 & 335.13 $\pm$ 6.41 \\ \hline
\textbf{Z+jets} & \textbf{10239.25 $\pm$ 176.43} & 4712.74 $\pm$ 120.55 & 5526.52 $\pm$ 128.83 \\ \hline
top, $t\bar{t}V$, $Wt$ & 535.95 $\pm$ 5.94 & 262.62 $\pm$ 4.16 & 273.33 $\pm$ 4.24 \\ \hline
WW & 113.04 $\pm$ 1.89 & 51.64 $\pm$ 1.28 & 61.4 $\pm$ 1.39 \\ \hline
Rare (4l, VVV, etc.) & 104.46 $\pm$ 4.38 & 54.65 $\pm$ 4.22 & 49.81 $\pm$ 1.19 \\ \hline
ZZ & 408.72 $\pm$ 5.98 & 188.98 $\pm$ 3.99 & 219.74 $\pm$ 3.99 \\ \hline
\hline
\textbf{Total Bkg} & \textbf{11629.16 $\pm$ 176.76} & 5382.98 $\pm$ 120.77 & 6246.19 $\pm$ 129.07 \\ \hline
\hline
Data/MC & 1.18 $\pm$ 0.02 & 1.23 $\pm$ 0.03 & 1.14 $\pm$ 0.03 \\ \hline
$\frac{\text{Data-OtherBkg}}{\text{Zjets}}$ & 1.21 $\pm$ 0.03 & 1.27 $\pm$ 0.05 & 1.16 $\pm$ 0.05 \\ \hline
Signal/Zjets (\%) & 3.99 $\pm$ 0 & 4.01 $\pm$ 0 & 3.98 $\pm$ 0 \\ \hline
\end{tabular}%
}
\caption{Pre-fit event yields in the inclusive $Z$+jets Control Region. The last row indicates the ratio of the Signal expectation to the dominant $Z$+jets background, confirming the region is background-enriched.}
\label{table:Zjets_unscaled}
\end{table}
\FloatBarrier

Validation of the simultaneous fit performance across these regions is detailed in Appendix~\ref{sec:TestScaled_3lCR}, while systematic uncertainties associated with the control region definitions and scaling factors are discussed in Section~\ref{sec:SystScaleFactorsCRvariations}.





\subsection{Minor Backgrounds}
\label{sec:bkg_minor}

Several background processes contribute to the signal region at a subleading level (typically $< 1\%$ of the total yield). Due to their small cross-sections or low acceptance in the analysis phase space, it is not feasible to define statistically significant dedicated control regions for these processes. Instead, their contributions are estimated purely from Monte Carlo simulation normalized to their theoretical cross-sections.

These minor backgrounds include:

\begin{itemize}
    \item \textbf{$ZZ \to 4\ell$:} This process can enter the signal region if two leptons are lost (e.g., falling outside detector acceptance or failing identification). While physically similar to the signal, the requirement of large $E_{miss}^T$ strongly suppresses this contribution.
    
    \item \textbf{Triboson Production ($VVV$):} Processes such as $WWW$, $WWZ$, and $WZZ$ have very small production cross-sections but can produce high lepton multiplicities and missing energy.
    
    \item \textbf{$t\bar{t}V$ ($t\bar{t}Z$, $t\bar{t}W$):} Associated production of top pairs with a boson. These events are largely rejected by the $b$-jet veto applied in the signal region.
    
    \item \textbf{$W$ + jets:} Events where a $W$ boson decays leptonically and a jet is misidentified as a second lepton. Strict lepton identification and isolation criteria employed in this analysis reduce this background to a negligible level.
\end{itemize}

Since these backgrounds are not constrained by data in the simultaneous fit, conservative systematic uncertainties are assigned to their theoretical cross-sections (typically on the order of $20\%$ to $30\%$) to account for potential modeling discrepancies.







