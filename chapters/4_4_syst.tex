\section{Systematic Uncertainties}
\label{sec:llvv_systematics}

This section summarizes the experimental and theoretical sources of systematic uncertainty affecting the event yields of the backgrounds and the signal. These sources impact the differential estimate of the signal cross-section after the unfolding procedure. 

The systematic uncertainties are assessed to account for residual discrepancies between data and simulation in the modelling of trigger efficiencies, lepton reconstruction, jet calibration, $b$-tagging, missing transverse energy ($E_{\text{T}}^{\text{miss}}$), luminosity, and the average number of proton--proton interactions per bunch crossing (pile-up).

Consistent with previous analyses of the $ZZjj$ final state, the measurement of the signal yields and resulting cross-sections is dominated by statistical uncertainty. Among the sources of systematic uncertainty, the largest contributions arise from jet reconstruction—specifically pile-up suppression and $\eta$-intercalibration—and from the unfolding bias. 
%The bias associated with the unfolding method is evaluated through the data-driven closure test discussed in Section~\ref{sec:unfolding_ddclosure}.

\subsection{Experimental Sources}
\label{subsec:exp_syst}

The experimental systematic uncertainties encompass detector effects and reconstruction efficiencies. For each source, the propagation to the final measurement is performed as follows:
\begin{enumerate}
    \item A detector-level Standard Model (SM) distribution is constructed using Monte Carlo (MC) predictions where the specific systematic parameter has been varied.
    \item The nominal background is subtracted from this varied distribution.
    \item The resulting background-subtracted distribution is unfolded using the \textit{nominal} response matrix.
    \item The systematic uncertainty is defined as the difference between this unfolded distribution and the nominal MC particle-level (truth) distribution.
\end{enumerate}

To ensure that upward and downward variations are statistically significant and symmetric, bootstrap replicas are utilized. In cases where the variations are symmetric within statistical uncertainties, the final systematic value is smoothed by averaging the absolute values of the variations.

The specific sources considered are listed below:

\begin{itemize}
    \item \textbf{Luminosity:} 
    The total integrated luminosity of the dataset is known to a precision of $0.83\%$. This uncertainty is determined using a combination of van der Meer beam separation scans in dedicated data-taking sessions and calibrated luminosity-sensitive detectors during regular data-taking periods~\cite{DAPR-2021-01}.
    
    \item \textbf{Pile-up Reweighting:} 
    The number of proton--proton interactions per bunch crossing is modelled in simulation by reweighting events to match the distribution observed in data. An uncertainty arises from the limited precision in the ratio of predicted to measured inelastic cross-sections within the ATLAS acceptance region~\cite{STDM-2015-05}. To account for this, a systematic variation is applied to the average number of interactions in the simulation via the nuisance parameter \texttt{PRW\_DATASF}.

    \item \textbf{Trigger Efficiency:} 
    Uncertainties in the trigger efficiency scale factors are separated by lepton flavor. For electrons, a single total nuisance parameter is used. For muons, the uncertainty is split into statistical and systematic components.

    \item \textbf{Lepton Efficiency (Reconstruction, ID, Isolation):} 
    Efficiencies for lepton reconstruction, identification, and isolation are corrected in simulation using scale factors derived from tag-and-probe control samples in data.
    \begin{itemize}
        \item \textbf{Electrons:} Modeled using a simplified de-correlation model containing 25 nuisance parameters for reconstruction, plus parameters for isolation and identification.
        \item \textbf{Muons:} Involves four uncertainty components covering statistical and systematic terms for both general and low-$p_{\text{T}}$ regimes, plus two parameters for isolation.
    \end{itemize}

    \item \textbf{Lepton Momentum and Scale:} 
    Differences between data and simulation in the lepton momentum scale and resolution are evaluated by applying additional scaling and smearing variations.
    \begin{itemize}
        \item \textbf{Electrons:} Two parameters, \texttt{EG\_RESOLUTION\_ALL} and \texttt{EG\_SCALE\_ALL}, account for resolution and scale uncertainties.
        \item \textbf{Muons:} Independent parameters are used for Inner Detector and Muon Spectrometer measurements, along with Sagitta correction uncertainties and momentum scale parameters.
    \end{itemize}

    \item \textbf{Jet Energy Scale (JES) and Resolution (JER):} 
    Given the hadronic nature of the VBS topology (two forward jets), jet systematics are a dominant source of uncertainty.
    \begin{itemize}
        \item \textbf{JES:} Derived using test-beam data, LHC collision data, and simulation. The \texttt{GlobalReduction} configuration is employed, corresponding to 23 distinct nuisance parameters.
        \item \textbf{JER:} Evaluated by applying additional smearing to the jet energy in simulated samples to match the resolution observed in data. The \texttt{FullJER} configuration (13 parameters) is used.
    \end{itemize}
    
    \item \textbf{Jet Vertex Tagger (JVT):} 
    Scale factor variations for the JVT are applied to account for uncertainties in pile-up jet suppression. Consistent with previous measurements, these variations result in a minor impact ($< 1\%$).

    \item \textbf{Flavor Tagging ($b$-tagging):}
    The uncertainty in the $b$-tagging efficiency, as well as the corresponding uncertainties in the $c$-jet and light-jet rejection scale factors, are taken into account~\cite{FTAG-2018-01}. These are propagated to the measurement through variations of the flavor-tagging scale factors.

    \item \textbf{Missing Transverse Energy ($E_{\text{T}}^{\text{miss}}$):}
    Uncertainties in $E_{\text{T}}^{\text{miss}}$ are evaluated by varying the soft term (signals not associated with high-$p_T$ objects) and by propagating the lepton and jet energy scale and resolution uncertainties to the $E_{\text{T}}^{\text{miss}}$ calculation.
\end{itemize}

\subsection{Theoretical Sources}
\label{subsec:theo_syst}

Theoretical variations typically induce only minor changes in the shapes of observables but can significantly impact predicted yields. These uncertainties are process-specific.

\begin{itemize}
    \item \textbf{QCD Scale Uncertainties:} 
    These are evaluated using on-the-fly weights corresponding to variations of the renormalization ($\mu_R$) and factorization ($\mu_F$) scales. Seven variations are considered: combinations of $\mu_R$ and $\mu_F$ at factors of 0.5, 1, and 2, excluding the extreme cases where one scale is halved and the other doubled. The envelope of these variations defines the uncertainty. Crucially, for the $ZZjj$ signal, the QCD-induced and EW-induced components are treated separately with \textbf{uncorrelated} scale uncertainties.
        
    \item \textbf{PDF and $\alpha_S$:} 
    Uncertainties are evaluated following the PDF4LHC prescription~\cite{Butterworth:2015oua}. This includes the RMS of 100 internal replicas of the NNPDF3.0 set. The effect of the strong coupling constant is assessed separately by varying $\alpha_s$ by $\pm 0.001$ and adding the result in quadrature.

    \item \textbf{MC Generator Modelling:} 
    To assess the uncertainty associated with the MC generator choice for the QCD-induced signal component, the nominal \textsc{Sherpa} sample is compared to an alternative sample generated with \textsc{Powheg}. The relative difference in the unfolded cross-section is taken as a systematic uncertainty.
    
    \item \textbf{Parton Shower:}     
    Uncertainties in the parton showering are calculated within \textsc{Sherpa} by varying the shower recoil scheme. Further details are documented in \ref{llvv_partonshower}.

    \item \textbf{$Z$+jets Background Normalization:}
    In the $ZZjj$ phase space, the $Z+\text{jets}$ background is estimated using MC simulation. A conservative normalization uncertainty of $50\%$ is assigned to this background to account for potential mismodelling in events with high jet multiplicity. This value is motivated by discrepancies observed in inclusive $Z+\geq 2$ jets control regions.
\end{itemize}

\subsection{Summary of Impact}

The experimental and theoretical uncertainties are incorporated into the analysis through variations in relevant nuisance parameters in simultaneous fits to data. The impact of these systematics on the measured fiducial cross-sections for the inclusive ($ZZ \to \ell\ell\nu\nu$) and VBS-enriched ($ZZjj \to \ell\ell\nu\nu jj$) phase spaces is summarized in Table~\ref{tab:uncertainties}.

\begin{table}[hbtp]
\caption{The breakdown of systematic uncertainties on the measured fiducial cross-sections $\sigma_{ZZ \to \ell \ell \nu\nu}$ and $\sigma_{ZZjj \to  \ell \ell \nu\nu jj}$. The values represent the relative uncertainty on the measured cross-section.}
\centering
\scriptsize
\renewcommand{\arraystretch}{1.2}
\begin{tabular}{lcc}
\toprule
\textbf{Systematic source} & \textbf{Impact on $\sigma_{ZZ \to \ell \ell \nu\nu}$} & \textbf{Impact on $\sigma_{ZZjj \to \ell \ell \nu\nu jj}$} \\
\midrule
Scale variation                          & 0.99\%  & 3.5\%   \\
PDF and $\alpha_s$          & 0.15\%  & 0.42\%  \\
Alternative signal generator & 0.38\%  & 0.93\%  \\
Z+jets modelling             & --      & 0.43\%  \\
\midrule
Jet energy scale and resolution & 1.7\%   & 8.7\%   \\
MC statistics                & 1.6\%   & 2.9\%   \\
Pile-up reweighting        & 0.39\%  & 4.1\%   \\
$E_\mathrm{T}^\text{miss}$  & 1.4\%   & 0.97\%  \\
Electrons     & 0.38\%  & 0.24\%  \\
Muons         & 0.35\%  & 0.42\%  \\
$b$-tagging                 & 0.28\%  & 1.2\%   \\
\midrule
Luminosity                  & 0.87\%  & 0.82\%  \\
\midrule
Total systematics             & 3.1\%   & 11\%   \\
\midrule
Statistics             & 3.6\%   & 14\%   \\
\midrule
\textbf{Total Uncertainty}            &  \textbf{4.8\%}  &  \textbf{18\%}   \\
\bottomrule
\end{tabular}
\label{tab:uncertainties}
\end{table}

