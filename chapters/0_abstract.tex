% !TeX root = ../main.tex

\ustcsetup{
  keywords  = {ATLAS, Z玻色子, 暗物质, 监测漂移管探测器},
  keywords* = {LHC, ATLAS, ZZ, Dark Matter, sMDT},
}

\begin{abstract}

  在Higgs玻色子发现之后, 标准模型的理论趋于完善, 验证标准模型、以及寻找超越标准模型之外的新物理是现在粒子物理的主要课题. W及Z玻色子是负责传递弱核力的基本粒子, 在其1983年被发现后一直是标准模型重要的核心组件, 而在大型核子对撞机上, ZZ玻色子对的产生是非常重要而且占比很大的一个末态, 对其进行观测的主要衰变道为四轻子衰变以及双轻子加不可见末态. 其中双轻子加不可见末态有以下几个特点: 拥有比四轻子高得多的截面, 有助于提高测量过程中的统计量, 提高测量精度; 末态简单, 只包含两个轻子; 物理过程简单, 分析过程. 因此双轻子加不可见末态也成为了对ZZ产生截面进行观测的良好末态. 
  
  在标准模型下Higgs玻色子会有0.001的比例衰变到两个Z玻色子再到四个中微子, 而在对暗物质的探测中, 在一些超越标准模型之外的理论预言了Higgs的不可见衰变, 对于ZH(inv)的末态则只会观测到一个可见的Z玻色子. 在这个衰变道下的可见末态同样是双轻子加不可见部分, 我们可以通过测量对该末态的比例来观测Higgs玻色子进行超越标准模型预言的衰变道的测量, 并给出其上限. 该部分会包含了spin-1和2HDM+a两个模型的预测, 并给出更为精确的观测上限. 
  
  在该文章中, 将使用ATLAS在2015-2018年收集的13TeV积分亮度为140$fb^{-1}$的对撞数据, 通过对双轻子加不可见末态的分析与测量, 进而得到对ZZ产生微分截面的测量. 该测量的结果, 其测量的总产生截面相比于使用ATLAS第二次运行早期的测量结果, 其拥有更高的精度, 同时与之前的测量结果相吻合. 同时也使用该收集的数据, 来探测该末态下存在Higgs进行超越标准模型衰变的可能性, 并得到现有超越标准模型理论预言的结果并不能显著观测到, 并重新设定Higgs玻色子不可见衰变的比例上限. 
  
  同时大型强子对撞机正在升级为高亮度大型强子对撞机, 更高的亮度会为新物理的发现带来更多机遇, 同时也对探测器的抗辐照特性, 计数率等方面带来更有挑战性的要求, 因而硬件方面的升级不可或缺. 该文章也收录了作者在ATLAS探测器升级过程中, 在小型监测漂移管探测器组装以及测试方面的工作, 这些新的探测器将用于ATLAS探测器的升级, 并为大型强子对撞机的下一步运行奠定良好的基础. 

\end{abstract}

%中文摘要是论文内容的总结概括,应简要说明论文的研究目的、基本研究内容、研究方法或过程、结果和结论,突出论文的创新之处。摘要应具有独立性和自明性,即不用阅读全文,就能获得论文必要的信息。摘要中不宜使用公式、图表,不引用文献。博士论文摘要一般控制在800-1500字
%中文关键词是为了文献标引工作从论文中选取出来用以表示全文主题内容信息的单词和术语,一般 3~8 个词,要求能够准确概括论文的核心内容。

\begin{abstract*}

This thesis undertakes studies based on data collected by the ATLAS experiment at the Large Hadron Collider, corresponding to an integrated luminosity of $140$ fb$^{-1}$ at a proton-proton collision energy of $\sqrt{s}=13$ TeV.
The analyses focus on the $\ell^+\ell^- E_T^{miss}$ final state, where $\ell^+\ell^-$ represents either an electron pair ($e^+e^-$) or a muon pair ($\mu^+\mu^-$) decay from the $Z$ boson, and $E_T^{miss}$ denotes the missing transverse energy. This final state originates from the decay channel of the vector boson pair $ZZ$, a Standard Model (SM) production process described by $ZZ\rightarrow \ell^+\ell^- \nu \bar\nu$.
This final state may also arise from the hypothetical signal of dark matter particles decaying from the Higgs boson in the production process of $pp\rightarrow ZH \rightarrow \ell^+\ell^- \chi\bar\chi$. In both cases, the neutrino $\nu$ and potentially the dark matter particle $\chi$ remain undetected, resulting in the experimental signature of missing transverse energy $E_T^{miss}$.

Two significant physics analyses and results are presented in this thesis in detail.
The first focuses on precisely measuring the Standard Model (SM) $ZZ$ production cross-section. This measurement scrutinizes the SM electroweak theory at the highest energy frontier. Additionally, it aims to probe the potential breakdown of the SM by investigating higher-order quantum loop corrections. Particularly, the analysis delves into measuring the anomalous neutral vector boson self-interactions, which could provide insights into deviations from the SM predictions.
The second analysis is dedicated to the search for dark matter particles via the $ZH$ production process. Here, the objective is to establish constraints on the invisible Higgs decay branching fraction, offering crucial insights into the possible existence of dark matter.

In the quest to search for dark matter particles through the $pp\rightarrow ZH \rightarrow \ell^+\ell^- \chi\bar\chi$ process, events stemming from the SM $ZZ\rightarrow \ell^+\ell^- \nu \bar\nu$ process represent the irreducible background.
Furthermore, both analyses contend with additional significant backgrounds originating from SM processes such as $WW$, $WZ$, $t\bar t$, $tW$, and $ttV$ productions, as well as the $Z$+jets process. These backgrounds are estimated using data-driven techniques.
To enhance sensitivity and diminish background interference in both analyses, multi-variate-analysis methods are employed. These methods serve to refine signal-to-background discrimination.
Ultimately, the extraction of final physics results hinges on fitting the observed data with the underlying physics models. This fitting process enables the extraction of meaningful insights into the presence or absence of dark matter particles and other phenomena under investigation.

%
%  Following the discovery of the Higgs boson, the Standard Model(SM) of particle physics has become increasingly robust. However, verifying the SM and exploring the beyond Standard Model(BSM) theories remains the primary focus in particle physics research. W and Z bosons are fundamental particles responsible for mediating the weak force, and they have been pivotal components of the Standard Model since their discovery in 1983. Particularly, the production of ZZ boson pairs is a significant and prevalent channel in the Large Hadron Collider (LHC). Its primary visible decay channels, including four leptons or two leptons accompanied by 2 invisible neutrinos, provide valuable insights for the verification of SM. Specifically, the di-lepton plus invisible final state has a significantly higher cross-section compared to the four-lepton channel, enhancing statistical significance and measurement precision. Moreover, its simplicity in the final state and physics process could facilitate the analysis.

% In the Standard Model theory, the Higgs boson has a small branching ratio (~0.001) to decay into two Z bosons, then to four neutrinos. However, some BSM(beyond the Standard Model) theories predict other invisible decays of the Higgs boson, which can be used for dark matter search. Therefore the ZH production in such scenarios, can only observe one visible Z boson(decays to visible state $ll$) in the final state, accompanied by an invisible part. By measuring the cross-section of events in this final state, we can probe the existence of Higgs boson decays with BSM predictions and set limitations on their branching ratios. 

%This thesis utilizes data collected by the ATLAS detector at $\sqrt{s}=13 TeV$, corresponding to an integrated luminosity of $140 fb^{-1}$ over the period 2015-2018. Through analysis and measurement of the $ll+E^{miss}_{T}$ final state, the differential cross-section for ZZ boson pair production is measured with the latest data. The obtained results show higher precision compared to previous measurements with early run-2 data corresponding to an integrated luminosity of $36.1 fb^{-1}$. The second analysis investigates the invisible decays of Higgs boson beyond the Standard Model predictions,  no significant excesses were observed above the SM expectation, and limits are set on the signal models.
%

In addition to the detailed physics analyses presented, this thesis also highlights a significant hardware project related to the ATLAS muon detector upgrade, a critical task in preparation for the high-luminosity phase of the Large Hadron Collider (HL-LHC) expected in 2029. 
The focus of this work lies in the production and testing of the small-diameter Monitored Drift Tube (sMDT) chambers. The author's contributions span various facets of this undertaking, including the development and implementation of infrastructure for detector construction, the meticulous assembly of precision tubes and chambers, and the rigorous execution of quality control tests. These efforts are integral to ensuring the ATLAS experiment's readiness to capitalize on the enhanced physics potential afforded by the HL-LHC era.
 
\end{abstract*}
