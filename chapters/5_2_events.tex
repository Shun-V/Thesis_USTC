\section{Event Selection}
\label{sec:monoZ_selection}

The analysis targets events with a leptonic $Z$ boson candidate and large missing transverse momentum. The selection is optimized to maximize the significance of DM signals against the Standard Model background, primarily $Z$+jets and diboson processes.

\subsection{Object Selection}
\label{subsec:object_sel}

Candidate events are required to have a primary vertex with at least two associated tracks with $p_{\mathrm{T}} > 1$ GeV. The physics objects used in this analysis—electrons, muons, jets, and missing transverse momentum—are defined as follows.

\paragraph{Electrons:}
Electrons are reconstructed from energy clusters in the electromagnetic calorimeter matched to tracks in the inner detector~\cite{PERF-2017-01}. They are required to be within the acceptance region $|\eta| < 2.47$ and satisfy $p_{\mathrm{T}} > 7$ GeV. For the final signal selection, electrons must satisfy "Medium" likelihood-based identification criteria and "FixedCutPflowLoose" isolation requirements.

\paragraph{Muons:}
Muons are reconstructed by combining tracks in the inner detector with tracks in the muon spectrometer~\cite{PERF-2015-10}. They are required to satisfy $p_{\mathrm{T}} > 7$ GeV and $|\eta| < 2.5$. Signal muons must satisfy "Medium" identification criteria and "FixedCutPflowLoose" isolation requirements.

\paragraph{Jets:}
Jets are reconstructed using the anti-$k_t$ algorithm with a radius parameter $R=0.4$~\cite{PERF-2016-04}. They are required to have $p_{\mathrm{T}} > 30$ GeV and $|\eta| < 4.5$. To suppress jets originating from pile-up interactions, jets with $p_{\mathrm{T}} < 60$ GeV and $|\eta| < 2.4$ must satisfy the Jet Vertex Tagger (JVT) requirement (JVT $> 0.5$).

\paragraph{Missing Transverse Momentum:}
The missing transverse momentum, $\vec{E}_{\mathrm{T}}^{\text{miss}}$, is calculated as the negative vector sum of the transverse momenta of all calibrated physics objects (electrons, muons, jets) and a "soft term" calculated from tracks associated with the primary vertex but not matched to any reconstructed object~\cite{ATL-PHYS-PUB-2015-027}. The magnitude is denoted as $E_{\mathrm{T}}^{\text{miss}}$. This analysis uses the "Tight" $E_{\mathrm{T}}^{\text{miss}}$ working point to improve the rejection of fake $E_{\mathrm{T}}^{\text{miss}}$ from pile-up jets, as recommended by the Jet/EtMiss group.

\subsection{Event Pre-selection}
\label{subsec:pre_selection}

Events are selected using single-electron or single-muon triggers. The baseline pre-selection criteria are designed to identify a $Z$ boson candidate while ensuring high efficiency:

\begin{itemize}
    \item \textbf{Leptons:} Exactly two same-flavor, opposite-sign (SFOS) leptons ($\ell = e, \mu$) are required. The leading lepton must have $p_{\mathrm{T}} > 30$ GeV, and the sub-leading lepton must have $p_{\mathrm{T}} > 20$ GeV.
    \item \textbf{Z-mass Window:} The invariant mass of the lepton pair must fall within $76 < m_{\ell\ell} < 106$ GeV, consistent with the $Z$ boson mass to suppress non-resonant backgrounds.
    \item \textbf{Third Lepton Veto:} Events containing a third lepton with $p_{\mathrm{T}} > 7$ GeV (satisfying "Loose" identification) are vetoed to suppress the $WZ$ background.
    \item \textbf{$b$-jet Veto:} Events containing any $b$-tagged jets ($p_{\mathrm{T}} > 20$ GeV, $|\eta| < 2.5$) are vetoed to suppress top-quark backgrounds ($t\bar{t}$, $Wt$). The MV2c10 tagger is used at the 85\% efficiency working point.
    \item \textbf{$E_{\mathrm{T}}^{\text{miss}}$ Threshold:} A loose requirement of $E_{\mathrm{T}}^{\text{miss}} > 70$ GeV is applied at the pre-selection stage.
\end{itemize}

\subsection{Signal Region Definition}
\label{subsec:sr_def}

To further enhance the sensitivity to Dark Matter signals, specific kinematic requirements are applied to define the Signal Region (SR). The optimization was performed using an iterative BDT approach and significance calculations:

\begin{itemize}
    \item \textbf{$E_{\mathrm{T}}^{\text{miss}} > 90$ GeV:} A tighter $E_{\mathrm{T}}^{\text{miss}}$ cut is applied to reduce the $Z+\text{jets}$ background, where $E_{\mathrm{T}}^{\text{miss}}$ often arises from instrumental mis-measurement.
    \item \textbf{$E_{\mathrm{T}}^{\text{miss}}$-significance $> 9$:} The object-based $E_{\mathrm{T}}^{\text{miss}}$-significance ($S$) is a powerful discriminant against fake missing energy. It assesses whether the observed $E_{\mathrm{T}}^{\text{miss}}$ is consistent with the resolution of the reconstructed objects. A cut of $S > 9$ significantly suppresses the $Z+\text{jets}$ contribution. The significance is defined as~\cite{ATL-PHYS-PUB-2020-025}:
    \begin{equation}
        Z = \sqrt{2\left((s+b)\ln\left[\frac{(s+b)(b+\sigma_{b}^{2})}{b^{2}+(s+b)\sigma_{b}^{2}}\right] - \frac{b^{2}}{\sigma_{b}^{2}}\ln\left[1+\frac{s\sigma_{b}^{2}}{b(b+\sigma_{b}^{2})}\right]\right)}
    \end{equation}
    \item \textbf{$\Delta R_{\ell\ell} < 1.8$:} The angular separation between the two leptons is required to be small. This selects events where the $Z$ boson is boosted, recoiling against the invisible DM system.
\end{itemize}

For the final statistical interpretation of the Mono-$Z$ channel, the transverse mass of the $ZZ$ system, $m_T^{ZZ}$, is used as the discriminant variable. It is defined as:
\begin{equation}
    m_{T}^{ZZ} = \sqrt{\left(\sqrt{m_{Z}^{2} + |p_{T}^{\ell\ell}|^{2}} + \sqrt{m_{Z}^{2} + |E_{T}^{\text{miss}}|^{2}}\right)^{2} - \left|\vec{p}_{T}^{\,\ell\ell} + \vec{E}_{T}^{\text{miss}}\right|^{2}}
\end{equation}
This variable effectively separates the signal (which peaks at higher values correlated with the mediator mass) from the SM backgrounds. Only events with $m_T^{ZZ} > 200$ GeV are included in the fit.