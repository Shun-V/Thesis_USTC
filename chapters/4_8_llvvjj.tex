\section{Electroweak $ZZjj$ Production and Search for aQGCs}
\label{sec:llvvjj}

While the inclusive $ZZ$ measurement provides a precision test of QCD and trilinear gauge couplings, the study of $ZZ$ production in association with two jets ($ZZjj$) offers a unique window into the electroweak sector. This final state is sensitive to Vector Boson Scattering (VBS), a process directly probing the quartic gauge boson self-interactions ($VVVV$) predicted by the Standard Model. This section presents the measurement of the $ZZjj \to \ell\ell\nu\nu jj$ cross-section and the subsequent search for Anomalous Quartic Gauge Couplings (aQGCs).

\subsection{Event Selection and Optimization}
\label{subsec:zzjj_selection}

To target the VBS topology, the event selection is modified from the inclusive $\ell\ell\nu\nu$ baseline described in Section~\ref{sec:llvv_event_selection}. The selection is optimized to suppress the dominant QCD-induced $ZZ$ background and enhance the electroweak signal-to-background ratio.

The definitions for leptons and the $Z$ boson candidate remain identical to the inclusive analysis. However, specific requirements are imposed on the hadronic activity and missing transverse momentum:

\begin{itemize}
    \item \textbf{Jet Multiplicity:} Events must contain at least two hadronic jets with $p_{\mathrm{T}} > 30$~GeV and $|\eta| < 4.5$. This ensures the presence of the "tagging jets" characteristic of VBS processes.
    \item \textbf{Missing Transverse Momentum:} The threshold is raised significantly to $E_{\mathrm{T}}^{\text{miss}} > 150$~GeV (compared to $110$~GeV in the inclusive channel). This stricter cut is essential to suppress the $Z+\text{jets}$ background, which is more abundant in high jet-multiplicity regions.
\end{itemize}

Standard VBS analyses often apply cuts on the dijet invariant mass ($m_{jj}$) or pseudorapidity separation ($\Delta y_{jj}$). However, in this fiducial measurement, these kinematic variables are inspected differentially rather than being used as cut thresholds, allowing for a broader phase space measurement.

\subsection{Fiducial Cross-Section Measurement}
\label{subsec:zzjj_xsec}

The signal yield is extracted using a profile-likelihood fit to the $\Delta\phi(\ell\ell, E_{\mathrm{T}}^{\text{miss}})$ distribution, similar to the inclusive strategy. The fit includes dedicated control regions for the $WZjj$ and non-resonant backgrounds.

The observed signal strength for the $ZZjj$ process is:
\begin{equation}
    \mu_{ZZjj} = 1.02^{+0.19}_{-0.17}.
\end{equation}
The background normalization factors are determined to be $0.97^{+0.24}_{-0.20}$ for $WZ$ and $1.08^{+0.19}_{-0.15}$ for the non-resonant $WW/t\bar{t}$ component.

The corresponding measured fiducial cross-section is:
\begin{equation}
    \sigma_{ZZjj \to \ell\ell\nu\nu jj}^{\text{fid}} = 0.96^{+0.18}_{-0.16}~\text{fb}.
\end{equation}
This result is in good agreement with the Standard Model prediction of $0.94 \pm 0.20$~fb, which combines the QCD-induced production (simulated by \textsc{Sherpa}) and the electroweak-induced production (simulated by \textsc{MadGraph5}). The uncertainty is dominated by data statistics due to the rarity of the process.

Differential cross-sections were also measured for seven kinematic observables, including the dijet invariant mass $m_{jj}$ and the transverse mass of the $ZZ$ system ($m_{\mathrm{T}}^{ZZ}$). As shown in Figure~\ref{fig:postfit_observed_zzjj} and Figure~\ref{fig:diff_xs_ZZjj}, the data is well described by the SM predictions within the larger uncertainties of this channel.

\begin{figure}[!htbp]
    \centering
    \subfloat[]{
        \includegraphics[width=0.32\linewidth]{figures/llvv/fig_postfit_d.pdf}
        \label{fig:postfit_leading_jet_pt}
    }
    \hfill
    \subfloat[]{
        \includegraphics[width=0.32\linewidth]{figures/llvv/fig_postfit_e.pdf}
        \label{fig:postfit_mjj}
    }
    \hfill
    \subfloat[]{
        \includegraphics[width=0.32\linewidth]{figures/llvv/fig_postfit_f.pdf}
        \label{fig:postfit_mtzz}
    }
    \caption{Post-fit kinematic distributions for the $ZZjj$ SR: 
    (a) the leading jet $p_\text{T}^\text{leading jet}$, 
    (b) the invariant mass of the dijet system $m_\text{jj}$, and 
    (c) the transverse mass of the $ZZ$ system $m_\text{T}^\text{ZZ}$. 
    The data points are shown with statistical error bars, while the shaded band represents the total post-fit uncertainty in the prediction, combining statistical and systematic uncertainties. Open markers indicate data points lying outside the vertical range of the plot.}
    \label{fig:postfit_observed_zzjj}
\end{figure}

\begin{figure}[!hbtp]
    \centering
    % Row 1: 3 images
    \subfloat[]{
        \includegraphics[width=0.32\linewidth]{figures/llvv/fig_diff_llvvjj_a.pdf}
        \label{fig:diff_llvvjj_z_pt}
    }
    \hfill
    \subfloat[]{
        \includegraphics[width=0.32\linewidth]{figures/llvv/fig_diff_llvvjj_b.pdf}
        \label{fig:diff_llvvjj_dphi}
    }
    \hfill
    \subfloat[]{
        \includegraphics[width=0.32\linewidth]{figures/llvv/fig_diff_llvvjj_c.pdf}
        \label{fig:diff_llvvjj_y_z}
    }
    \\ % Row 2: 2 images (slightly wider for visibility)
    \subfloat[]{
        \includegraphics[width=0.45\linewidth]{figures/llvv/fig_diff_llvvjj_d.pdf}
        \label{fig:diff_llvvjj_pt_zz}
    }
    \hfill
    \subfloat[]{
        \includegraphics[width=0.45\linewidth]{figures/llvv/fig_diff_llvvjj_e.pdf}
        \label{fig:diff_llvvjj_mt_zz}
    }
    \\ % Row 3: 2 images
    \subfloat[]{
        \includegraphics[width=0.45\linewidth]{figures/llvv/fig_diff_llvvjj_f.pdf}
        \label{fig:diff_llvvjj_jet_pt}
    }
    \hfill
    \subfloat[]{
        \includegraphics[width=0.45\linewidth]{figures/llvv/fig_diff_llvvjj_g.pdf}
        \label{fig:diff_llvvjj_mjj}
    }

    \caption{Differential cross-section measurements in the $ZZjj \rightarrow \ell\ell\nu\nu jj$ fiducial phase space, compared with SM predictions for a set of kinematic variables:
    (a) $p_\text{T}^\text{Z}$,
    (b) $\Delta\phi(\ell, \ell)$,
    (c) $|y^Z|$,
    (d) $p_\text{T}^\text{ZZ}$,
    (e) $m_\text{T}^\text{ZZ}$,
    (f) $p_\text{T}^\text{leading jet}$, and 
    (g) $m_\text{jj}$.
    The SM predictions are obtained using \textsc{Sherpa} for strong production and \textsc{MadGraph5} for electroweak production. The statistical and total uncertainties in the data points are displayed in both the differential cross-sections and the corresponding ratios. The total theory uncertainties in the MC predictions are represented as shaded bands in the cross-section panels and as error bars in the ratio panels.}
    \label{fig:diff_xs_ZZjj}
\end{figure}

\subsection{Constraints on Anomalous Quartic Gauge Couplings}
\label{subsec:aqgc_limits}

The $ZZjj$ channel provides direct sensitivity to the $ZZWW$ and $ZZZZ$ quartic vertices. In the Standard Model, the cross-section contribution from these vertices is small. However, physics Beyond the Standard Model (BSM) could manifest as Anomalous Quartic Gauge Couplings (aQGCs), leading to a significant enhancement of the cross-section at high energy scales.

\subsubsection{EFT Interpretation}

The aQGCs are parameterized using the Standard Model Effective Field Theory (SMEFT) framework. This analysis considers dimension-8 operators, specifically the transversal operators $T0$ through $T9$, which induce quartic couplings without modifying triple gauge couplings.

The most sensitive observable for detecting aQGCs is the transverse mass of the $ZZ$ system, defined as:
\begin{equation}
    m_{\mathrm{T}}^{ZZ} = \sqrt{ \left( \sqrt{m_Z^2 + (p_{\mathrm{T}}^{\ell\ell})^2} + \sqrt{m_Z^2 + (E_{\mathrm{T}}^{\text{miss}})^2} \right)^2 - \left| \vec{p}_{\mathrm{T}}^{\ell\ell} + \vec{E}_{\mathrm{T}}^{\text{miss}} \right|^2 }.
\end{equation}
The presence of aQGCs would appear as an excess of events in the high-$m_{\mathrm{T}}^{ZZ}$ tail.

\subsubsection{Limits on Wilson Coefficients}

A binned likelihood fit is performed on the $m_{\mathrm{T}}^{ZZ}$ distribution in the $ZZjj$ signal region. No significant deviation from the SM background expectation is observed. Consequently, 95\% confidence level (CL) limits are set on the Wilson coefficients $f_{T,i}/\Lambda^4$.

The observed limits are consistent with zero for all operators. To ensure the validity of the EFT expansion, unitarity bounds are calculated as a function of the energy cut-off scale $E_c$. The evolution of the limits for the $f_{T0}$ - $f_{T3}$ operator is shown in Figure~\ref{fig:aqgc_limits_fT0_fT3}.

Notably, the constraints obtained in this $\ell\ell\nu\nu jj$ analysis are significantly tighter than those from the fully leptonic $ZZ \to 4\ell jj$ channel, improving the sensitivity by more than a factor of two. This improvement is driven primarily by the larger branching ratio of the $Z \to \nu\nu$ decay compared to $Z \to \ell\ell$, which preserves statistics in the high-$m_{\mathrm{T}}^{ZZ}$ tail where the aQGC sensitivity is highest.

\begin{figure}[htbp]
    \centering
    % Top Row: fT0 and fT1
    \subfloat[]{
        \includegraphics[width=0.45\linewidth]{figures/llvv/fig_aqgc_limit_fT0.pdf}
        \label{fig:aqgc_fT0}
    }
    \hfill
    \subfloat[]{
        \includegraphics[width=0.45\linewidth]{figures/llvv/fig_aqgc_limit_fT1.pdf}
        \label{fig:aqgc_fT1}
    }
    \\ % Line break for Bottom Row
    % Bottom Row: fT2 and fT3
    \subfloat[]{
        \includegraphics[width=0.45\linewidth]{figures/llvv/fig_aqgc_limit_fT2.pdf}
        \label{fig:aqgc_fT2}
    }
    \hfill
    \subfloat[]{
        \includegraphics[width=0.45\linewidth]{figures/llvv/fig_aqgc_limit_fT3.pdf}
        \label{fig:aqgc_fT3}
    }
    
    \caption{Expected and observed 95\% CL intervals for the Wilson coefficients (a) $f_{\text{T0}}$, (b) $f_{\text{T1}}$, (c) $f_{\text{T2}}$, and (d) $f_{\text{T3}}$ as a function of the EFT cut-off scale $E_\text{c}$. 
    The cut-off scale imposes a restriction such that the BSM amplitudes are set to zero when $m_{ZZ} > E_\text{c}$. 
    The dashed (solid) lines show the expected (observed) limits, and the solid green curve corresponds to the unitarity bounds.}
    \label{fig:aqgc_limits_fT0_fT3}
\end{figure}


% \begin{figure}[htbp]
%     \centering
%     % Row 1
%     \subfloat[]{
%         \includegraphics[width=0.32\textwidth]{figures/llvv/fig_aqgc_limit_fT4.pdf}
%         \label{fig:aqgc_fT4}
%     }
%     \hfill
%     \subfloat[]{
%         \includegraphics[width=0.32\textwidth]{figures/llvv/fig_aqgc_limit_fT5.pdf}
%         \label{fig:aqgc_fT5}
%     }
%     \hfill
%     \subfloat[]{
%         \includegraphics[width=0.32\textwidth]{figures/llvv/fig_aqgc_limit_fT6.pdf}
%         \label{fig:aqgc_fT6}
%     }
%     \\ % Line break
%     % Row 2
%     \subfloat[]{
%         \includegraphics[width=0.32\textwidth]{figures/llvv/fig_aqgc_limit_fT7.pdf}
%         \label{fig:aqgc_fT7}
%     }
%     \hfill
%     \subfloat[]{
%         \includegraphics[width=0.32\textwidth]{figures/llvv/fig_aqgc_limit_fT8.pdf}
%         \label{fig:aqgc_fT8}
%     }
%     \hfill
%     \subfloat[]{
%         \includegraphics[width=0.32\textwidth]{figures/llvv/fig_aqgc_limit_fT9.pdf}
%         \label{fig:aqgc_fT9}
%     }
%     \caption{Expected and observed 95\% CL intervals for the Wilson coefficients $f_{\text{T4}}$ through $f_{\text{T9}}$ as a function of the EFT cut-off scale $E_\text{c}$. The dashed (solid) lines show the expected (observed) limits, and the solid green curve corresponds to the unitarity bounds.}
%     \label{fig:aqgc_limits_fT4_fT9}
% \end{figure}