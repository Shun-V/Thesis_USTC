% !TeX root = ../main.tex

\chapter{\texorpdfstring{Measurement of ZZ Production with $ll\nu\nu$ Final State}{Measurement of ZZ Production with llnunu Final State}}


\section{Introduction}
\label{sec:ZZ_intro}

The study of diboson production at the Large Hadron Collider (LHC) is a cornerstone of the experimental program to test the Standard Model (SM) of particle physics. Among these processes, the production of $Z$ boson pairs ($ZZ$) is of particular importance. It provides a direct probe of the electroweak gauge structure and the non-Abelian self-interactions of the gauge bosons. Precise measurements of the $ZZ$ production cross-section constitute a stringent test of the SM predictions at the highest available energies and serve as a sensitive probe for new physics phenomena that may manifest as anomalous Triple Gauge Couplings (aTGCs).

This chapter presents a measurement of the $ZZ$ production cross-section in proton-proton collisions at a center-of-mass energy of $\sqrt{s} = 13$ TeV, using the full Run 2 dataset collected by the ATLAS detector, corresponding to an integrated luminosity of $140~\text{fb}^{-1}$. The analysis focuses on the final state where one $Z$ boson decays into a pair of charged leptons ($\ell^+\ell^-$, where $\ell = e, \mu$) and the other decays into a pair of neutrinos ($\nu\bar{\nu}$).

\subsection{Motivation for the $\ell\ell\nu\nu$ Channel}

While the "golden channel" $ZZ \to 4\ell$ offers a fully reconstructed final state with very high signal purity, it suffers from a small branching ratio. The branching fraction of the $Z$ boson to neutrinos ($\mathcal{B}(Z \to \nu\bar{\nu}) \approx 20\%$) is approximately six times larger than that to charged leptons ($\mathcal{B}(Z \to \ell^+\ell^-) \approx 3.3\%$). Consequently, the $ZZ \to \ell\ell\nu\nu$ channel benefits from significantly higher statistics.

This statistical advantage is particularly crucial in the high transverse momentum ($p_{\mathrm{T}}$) regime. Since deviations from the SM due to effective field theory (EFT) operators or new heavy resonances are expected to be enhanced at high energies, the $\ell\ell\nu\nu$ channel provides superior sensitivity for constraining aTGCs compared to the $4\ell$ channel. Furthermore, understanding the $ZZ \to \ell\ell\nu\nu$ process is essential as it constitutes a major irreducible background for searches for Dark Matter (Mono-$Z$) and high-mass Higgs boson decays.

\subsection{Analysis Strategy and Outline}

The measurement is performed in a fiducial phase space defined to match the experimental acceptance. The analysis relies on identifying events with a dilepton pair consistent with a $Z$ boson mass and significant missing transverse momentum ($E_{\mathrm{T}}^{\text{miss}}$) arising from the neutrinos.

The structure of this chapter is as follows:
\begin{itemize}
    \item Section~\ref{sec:llvv_samples} describes the data and Monte Carlo (MC) simulation samples used to model the signal and background processes.
    \item Section~\ref{sec:llvv_selection} details the reconstruction of physics objects and the selection criteria defining the Signal Regions (SR).
    \item Section~\ref{sec:llvv_background} discusses the estimation of major backgrounds, particularly the dominant $WZ$ and non-resonant backgrounds, using data-driven control regions.
    \item Section~\ref{sec:llvv_systematics} outlines the sources of experimental and theoretical systematic uncertainties.
    \item Section~\ref{sec:llvv_unfolding} presents the unfolding procedure used to correct the detector-level distributions to the particle level.
    \item Section~\ref{sec:llvv_dnn} describes the optimization strategies, including the use of Deep Neural Networks (DNN), to enhance the separation between signal and background.
    \item Section~\ref{sec:llvv_results} reports the measured integrated and differential cross-sections for the inclusive $ZZ$ production and the constraints on aTGCs.
    \item Finally, Section~\ref{sec:llvvjj} extends the analysis to the electroweak production of $ZZ$ in association with two jets ($ZZjj$), targeting the Vector Boson Scattering (VBS) topology and Anomalous Quartic Gauge Couplings (aQGCs).
\end{itemize}


\clearpage
%\section{Data and Monte Carlo Samples}
\section{Monte Carlo and Data Samples}

This chapter details the datasets and simulated samples employed in the analysis. The measurement is performed on the full proton-proton ($pp$) collision dataset recorded by the ATLAS experiment during Run~2 of the Large Hadron Collider (LHC) at a centre-of-mass energy of $\sqrt{s} = 13$~TeV.

Monte Carlo (MC) simulations are essential for modeling the kinematic and topological properties of the signal process and for estimating the contributions from all significant Standard Model backgrounds. A cornerstone of the analysis methodology is the consistent application of object reconstruction, calibration, and event selection criteria to both the data and the simulated samples. This unified treatment ensures that a direct and unbiased comparison can be made between the observed data and the theoretical predictions, allowing for the extraction of the physics results.

The following sections will first provide a comprehensive overview of the MC samples generated for the signal and various background processes. Subsequently, the data samples, data quality requirements, and the trigger strategy employed to select events for this analysis will be detailed.


\subsection{Monte Carlo Samples}

The estimation of signal efficiencies and background contributions relies on a diverse set of Monte Carlo (MC) event samples. These samples are generated to correspond to the 2015--2018 data-taking periods and are processed through the same reconstruction software and calibration procedures as the collision data. This ensures a consistent treatment of physics objects and allows for a direct comparison between observation and prediction. The following sections describe the general simulation framework before detailing the specific samples used to model signal and background processes.

\subsubsection{The ATLAS Simulation Framework}
The production of simulated events in ATLAS follows a well-established multi-step procedure, ensuring a high-fidelity representation of proton-proton collisions and the subsequent detector response. This simulation chain consists of three main stages:

\begin{enumerate}
    \item \textbf{Event Generation:} The initial hard-scatter interaction is simulated using a variety of event generators. These generators compute the matrix element (ME) for a specific physics process, often to Next-to-Leading Order (NLO) or Leading Order (LO) in perturbative QCD. The ME calculation is then interfaced with a parton shower (PS) algorithm, such as those provided by \texttt{PYTHIA~8}~\cite{Sjostrand:2014zea} or the one internal to \texttt{Sherpa}~\cite{Bothmann_2019}, which models initial- and final-state radiation, multiple parton interactions, hadronisation, and particle decays. The specific generator, perturbative order, and Parton Distribution Function (PDF) set used for each sample are chosen to provide the most accurate description of the process.

    \item \textbf{Detector Simulation:} The stable particles generated in the first step are propagated through a detailed model of the ATLAS detector geometry and material composition using the \texttt{Geant4} toolkit~\cite{Agostinelli:2002hh}. This stage simulates the interactions of particles with the active and passive elements of the detector, resulting in simulated energy deposits and "hits" in the various sub-systems.

    \item \textbf{Digitisation and Reconstruction:} The simulated hits are converted into digitised electronic signals that emulate the real detector readout. To account for the high-luminosity environment of the LHC, multiple inelastic $pp$ collisions, known as pile-up, are overlaid on each hard-scatter event. These pile-up events are simulated with \texttt{PYTHIA~8} using the A3 tune~\cite{ATL-PHYS-PUB-2016-017} and the \texttt{NNPDF2.3LO} PDF set~\cite{Ball:2012cx}. The distribution of the number of pile-up interactions is weighted to match that observed in the data. Finally, the same reconstruction algorithms used for collision data are applied to the digitised output to reconstruct high-level physics objects such as electrons, muons, jets, and missing transverse momentum ($E_{\text{T}}^{\text{miss}}$).
\end{enumerate}

Correction factors, or scale factors, are applied to all simulated samples to account for small differences in trigger, reconstruction, and identification efficiencies between data and MC simulation.

\subsubsection{Signal Samples}
The analysis targets the measurement of the SM $ZZ$ production cross-section in the $\ell^+\ell^-\nu\bar{\nu}$ final state and searches for BSM physics through anomalous gauge boson couplings.

\paragraph{Standard Model \textit{ZZ} Production}
The primary signal process is the production of a pair of $Z$ bosons decaying to $\ell^+\ell^-\nu\bar{\nu}$. This includes both quark-antiquark initiated and loop-induced gluon-gluon fusion processes, which are simulated with \texttt{Sherpa 2.2.2}. Electroweak $ZZ$ production in association with two jets is modelled separately with \texttt{MadGraph5}. Details of these samples are provided in Table~\ref{tab:ZZ_signal_samples}.

\paragraph{Anomalous Gauge Couplings}
The search for new physics is performed by probing for anomalous neutral triple gauge couplings (nTGC) and anomalous quartic gauge couplings (aQGC) using an Effective Field Theory (EFT) framework. A dedicated set of MC samples has been generated to model the kinematic effects of these BSM operators. The samples used for the nTGC and aQGC searches are listed in Tables~\ref{tab:ZZEFTsamples} and~\ref{tab:ZZaQGCsamples}, respectively.

\begin{table}[htbp]
\centering
\renewcommand\arraystretch{1.5}
\caption{Summary of Monte Carlo samples used to model the SM $ZZ \to \ell^+\ell^-\nu\bar{\nu}$ signal components. Cross-sections ($\sigma$) are provided by the generator, and k-factors are applied to correct to higher-order predictions.}
\label{tab:ZZ_signal_samples}
\scriptsize
\begin{tabularx}{450pt}{|X||c|c|c|c|}
\hline
\centering \textbf{Process} & \textbf{DSID} & \textbf{Generator} & \textbf{$\sigma$ [pb]} & \textbf{k-factor}\\
\hline
\centering $gg\rightarrow ZZ \to \ell^+\ell^-\nu\bar{\nu}$ & 345723 & Sherpa 2.2.2 & 0.0071108 & 1.7 \\ \hline
\centering $q\bar{q}\rightarrow ZZ/\gamma^* \to \ell^+\ell^-\nu\bar{\nu}$ & 345666 & Sherpa 2.2.2 & 0.49908 & 1.5 \\ \hline
\centering Electroweak $ZZjj \to \ell^+\ell^-\nu\bar{\nu}jj$ & 363724 &  MadGraph5 & 0.0013529 & - \\ \hline
\end{tabularx}
\end{table}

\begin{table}[htbp]
\centering
\renewcommand\arraystretch{1.5}
\caption{List of samples generated for the study of anomalous Neutral Triple Gauge Couplings (nTGCs).}
\label{tab:ZZEFTsamples}
\scriptsize
\begin{tabularx}{450pt}{|X|c|}
\hline
\centering \textbf{Process} & \textbf{DSID} \\
\hline
nunull\_f4gamma\_plus0001 & 367911 \\ \hline
nunull\_f4Z\_plus0001 & 367912 \\ \hline
nunull\_f5gamma\_plus0001 & 367913 \\ \hline
nunull\_f5Z\_plus0001 & 367914 \\ \hline
nunull\_sm & 367915 \\ \hline
\end{tabularx}
\end{table}

\begin{table}[htbp]
\centering
\renewcommand\arraystretch{1.5}
\caption{List of samples generated for the study of anomalous Quartic Gauge Couplings (aQGCs).}
\label{tab:ZZaQGCsamples}
\scriptsize
\begin{tabularx}{450pt}{|X|c|}
\hline
\centering \textbf{Process} & \textbf{DSID} \\
\hline
aQGCFT0\_INT\_05\_ZZ\_llvv & 515527 \\ \hline
aQGCFT0\_QUAD\_05\_ZZ\_llvv & 515528 \\ \hline
aQGCFT1\_INT\_1\_ZZ\_llvv & 515529 \\ \hline
aQGCFT1\_QUAD\_1\_ZZ\_llvv & 515530 \\ \hline
aQGCFT2\_INT\_1\_ZZ\_llvv & 515531 \\ \hline
aQGCFT2\_QUAD\_1\_ZZ\_llvv & 515532 \\ \hline
aQGCFT5\_INT\_1\_ZZ\_llvv & 515533 \\ \hline
aQGCFT5\_QUAD\_1\_ZZ\_llvv & 515534 \\ \hline
aQGCFT6\_INT\_1\_ZZ\_llvv & 515535 \\ \hline
aQGCFT6\_QUAD\_1\_ZZ\_llvv & 515536 \\ \hline
aQGCFT7\_INT\_1\_ZZ\_llvv & 515537 \\ \hline
aQGCFT7\_QUAD\_1\_ZZ\_llvv & 515538 \\ \hline
aQGCFT8\_INT\_1\_ZZ\_llvv & 515539 \\ \hline
aQGCFT8\_QUAD\_1\_ZZ\_llvv & 515540 \\ \hline
aQGCFT9\_INT\_2\_ZZ\_llvv & 515541 \\ \hline
aQGCFT9\_QUAD\_2\_ZZ\_llvv & 515542 \\ \hline
\end{tabularx}
\end{table}

\FloatBarrier
\subsubsection{Background Samples}
Backgrounds to the $\ell^+\ell^- + E_{\text{T}}^{\text{miss}}$ final state are broadly categorised as either featuring prompt leptons and genuine $E_{\text{T}}^{\text{miss}}$ (irreducible) or arising from misidentified objects or detector mismeasurement (reducible). The MC samples used to model these processes are detailed in the following tables.

\paragraph{Diboson and Triboson Backgrounds}
The production of $WZ$, $WW$, and $ZZ \to \ell^+\ell^-\ell^+\ell^-$ represent the most significant irreducible backgrounds. Smaller contributions from triboson processes ($WWW$, $WWZ$, $WZZ$, $ZZZ$) are also considered. These processes are primarily simulated with \texttt{Sherpa} and \texttt{Powheg+Pythia8}. Summaries are provided in Tables~\ref{tab:ZZbkg_samples},~\ref{tab:WZandWWsamples}, and~\ref{tab:VVVsamples}.

\paragraph{Top Quark Backgrounds}
Events from top-antitop ($t\bar{t}$) and single-top production are a significant source of background, primarily from their dileptonic decay channels. Associated production with a vector boson ($t\bar{t}V$) also contributes. These are simulated with \texttt{Powheg+Pythia8} and \texttt{MadGraph5\_aMC@NLO}, as detailed in Tables~\ref{tab:Topsamples} and~\ref{tab:ttVsamples}.

\paragraph{$Z/\gamma^*$+jets Background}
This is the dominant reducible background, arising from Drell-Yan events with large, mismeasured $E_{\text{T}}^{\text{miss}}$. It is modelled using \texttt{Sherpa~2.2.1} with NLO-accurate matrix elements. Samples are generated in slices of the partonic transverse momentum sum and separated by lepton flavour, as shown in Tables~\ref{tab:ZjetsSamplesSherpa1},~\ref{tab:ZjetsSamplesSherpa2}, and~\ref{tab:ZjetsSamplesSherpa3}.

\paragraph{Higgs Boson Backgrounds}
Processes involving the production of a Higgs boson can also contribute to the selected final state. These minor backgrounds are simulated with \texttt{Powheg+Pythia8} and are listed in Table~\ref{tab:Higgssamples}.

% --- Diboson BG Tables ---
\begin{table}[htbp]
\centering
\renewcommand\arraystretch{1.5}
\caption{Summary of $ZZ$ background samples, including the four-lepton final state.}
\label{tab:ZZbkg_samples}
\scriptsize
\begin{tabularx}{450pt}{|X||c|c|c|c|}
\hline
\centering \textbf{Process} & \textbf{DSID} & \textbf{Generator} & \textbf{$\sigma$ [pb]} & \textbf{k-factor}\\
\hline
\centering $gg\rightarrow ZZ \to 4\ell$ & 345706 & Sherpa 2.2.2 & 0.010091 & 1.7 \\ \hline 
\centering $q\bar{q}\rightarrow ZZ/\gamma^* \to 4\ell$ & 364250 & Sherpa 2.2.2 & 1.2522 & - \\ \hline 
\centering $q\bar{q}\rightarrow ZZ \rightarrow q\bar{q} \ell^+\ell^-$ & 363356 & Sherpa 2.2.1 & 15.565 & 0.1419 \\ \hline 
\end{tabularx}
\end{table}

\begin{table}[htbp]
\centering
\renewcommand\arraystretch{1.5}
\caption{Summary of $WZ$ and $WW$ background samples.}
\label{tab:WZandWWsamples}
\scriptsize
\begin{tabularx}{450pt}{|X||c|c|c|c|}
\hline
\centering \textbf{Process} & \textbf{DSID} & \textbf{Generator} & \textbf{$\sigma$ [pb]} & \textbf{k-factor}\\
\hline
\centering $WZ \to \ell\nu\ell^+\ell^-$ & 364253 & Sherpa 2.2.2 & 4.5718 & - \\ \hline
\centering $WZjj \to \ell\nu\ell^+\ell^-jj$ & 364284 & Sherpa 2.2.2 & 0.047385 & - \\ \hline
\centering $WZ \rightarrow q\bar{q}\ell^{+} \ell^{-}$ & 363358 & Sherpa 2.2.1 & 3.4328 & - \\ \hline
\centering $q\bar{q} \rightarrow WW \to \ell\nu\ell\nu$ & 361600 & Powheg+Pythia8 & 10.631 & - \\ \hline
\centering $q\bar{q}\rightarrow WW \rightarrow q\bar{q} \ell\nu $ & 361606 & Powheg+Pythia8 & 44.18 & - \\ \hline
\centering $gg \rightarrow WW \to \ell\nu\ell\nu$ & 345718 & Sherpa 2.2.2 & 0.4823 & - \\ \hline
\end{tabularx}
\end{table}

% --- Triboson BG Table ---
\begin{table}[htbp]
\centering
\renewcommand\arraystretch{1.5}
\caption{Summary of triboson ($VVV$) background samples.}
\label{tab:VVVsamples}
\scriptsize
\begin{tabularx}{450pt}{|X||c|c|c|c|}
\hline
\centering \textbf{Process} & \textbf{DSID} & \textbf{Generator} & \textbf{$\sigma$ [pb]} & \textbf{k-factor}\\
\hline
\centering $WWW \to 3\ell3\nu$ & 364242 & Sherpa 2.2.2 & 0.0071997 & - \\ \hline
\centering $WWZ \to 4\ell2\nu$ & 364243 & Sherpa 2.2.2 & 0.0017973 & - \\ \hline 
\centering $WWZ \to 2\ell4\nu$ & 364244 & Sherpa 2.2.2 & 0.0035481 & - \\ \hline 
\centering $WZZ \to 5\ell1\nu$ & 364245 & Sherpa 2.2.2 & 0.00018812 & - \\ \hline 
\centering $WZZ \to 3\ell3\nu$ & 364246 & Sherpa 2.2.2 & 0.0016664 & 0.44594 \\ \hline 
\centering $ZZZ \to 6\ell$ & 364247 & Sherpa 2.2.2 & 1.4458e-05 & - \\ \hline 
\centering $ZZZ \to 4\ell2\nu$ & 364248 & Sherpa 2.2.2 & 0.00038556 & - \\ \hline
\centering $ZZZ \to 2\ell4\nu$ & 364249 & Sherpa 2.2.2 & 0.00038491 & 0.44479 \\ \hline 
\end{tabularx}
\end{table}

\FloatBarrier
% --- Top BG Tables ---
\begin{table}[htbp]
\centering
\renewcommand\arraystretch{1.5}
\caption{Summary of $t\bar{t}$ and single-top background samples.}
\label{tab:Topsamples}
\scriptsize
\begin{tabularx}{450pt}{|X||c|c|c|c|}
\hline
\centering \textbf{Process} & \textbf{DSID} & \textbf{Generator} & \textbf{$\sigma$ [pb]} & \textbf{k-factor}\\
\hline
\centering $t\bar{t}$ & 410472 & Powheg+Pythia8 & 729.77 & 0.12020 \\ \hline 
\centering single top (s-channel) & 410644 & Powheg+Pythia8 & 2.027 & - \\ \hline
\centering single anti-top (s-channel) & 410645 & Powheg+Pythia8 & 1.2674 & - \\ \hline 
\centering single top (t-channel) & 410658 & Powheg+Pythia8 & 36.996 & - \\ \hline 
\centering single anti-top (t-channel) & 410659 & Powheg+Pythia8 & 22.175 & - \\ \hline 
\centering $Wt$ (dilepton) & 410648 & Powheg+Pythia8 & 3.997 & - \\ \hline 
\centering $W\bar{t}$ (dilepton) & 410649 & Powheg+Pythia8 & 3.993 & - \\ \hline 
\end{tabularx}
\end{table}

\begin{table}[htbp]
\centering
\renewcommand\arraystretch{1.5}
\caption{Summary of background samples for associated production of top quarks with vector bosons.}
\label{tab:ttVsamples}
\scriptsize
\begin{tabularx}{450pt}{|X||c|c|c|c|}
\hline
\centering \textbf{Process} & \textbf{DSID} & \textbf{Generator} & \textbf{$\sigma$ [pb]} & \textbf{k-factor}\\
\hline
\centering $t\bar{t}Z, Z\rightarrow \nu \nu$  & 410156 & MG5\_aMC@NLO+Pythia8 & 0.15497 & - \\ \hline 
\centering $t\bar{t}Z, Z\rightarrow q\bar{q}$ & 410157 & MG5\_aMC@NLO+Pythia8 & 0.52821 & - \\ \hline 
\centering $t\bar{t}W$ & 410155 & MG5\_aMC@NLO+Pythia8 & 0.5483 & 1.1 \\ \hline 
\centering $t\bar{t}WW$ & 410081 & MadGraph+Pythia8 & 0.0080975 & 1.2231 \\ \hline
\end{tabularx}
\end{table}

\FloatBarrier
% --- Z+jets BG Tables ---
\begin{table}[htbp]
\centering
\renewcommand\arraystretch{1.5}
\caption{Summary of the $Z/\gamma^* \to e^+e^-$+jets background samples generated with \texttt{Sherpa~2.2.1}.}
\label{tab:ZjetsSamplesSherpa1}
\scriptsize
\begin{tabularx}{450pt}{|X||c|c|c|}
\hline
\centering \textbf{Process Slice} & \textbf{DSID} & \textbf{$\sigma$ [pb]} & \textbf{k-factor}\\
\hline
$Z\rightarrow ee$, $H_{T,parton} < 70$ GeV, CVetoBVeto & 364114 & 1981.8 & 0.8006 \\ \hline
$Z\rightarrow ee$, $H_{T,parton} < 70$ GeV, CFilterBVeto & 364115 & 1980.8 & 0.1101 \\ \hline
$Z\rightarrow ee$, $H_{T,parton} < 70$ GeV, BFilter & 364116 & 1981.7 & 0.0622 \\ \hline
$Z\rightarrow ee$, $70 < H_{T,parton} < 140$ GeV, CVetoBVeto & 364117 & 110.5 & 0.6732 \\ \hline
$Z\rightarrow ee$, $70 < H_{T,parton} < 140$ GeV, CFilterBVeto & 364118 & 110.63 & 0.1792 \\ \hline
$Z\rightarrow ee$, $70 < H_{T,parton} < 140$ GeV, BFilter & 364119 & 110.31 & 0.1116 \\ \hline
$Z\rightarrow ee$, $140 < H_{T,parton} < 280$ GeV, CVetoBVeto & 364120 & 40.731 & 0.5992 \\ \hline
$Z\rightarrow ee$, $140 < H_{T,parton} < 280$ GeV, CFilterBVeto & 364121 & 40.67 & 0.2247 \\ \hline
$Z\rightarrow ee$, $140 < H_{T,parton} < 280$ GeV, BFilter & 364122 & 40.643 & 0.1459 \\ \hline
$Z\rightarrow ee$, $280 < H_{T,parton} < 500$ GeV, CVetoBVeto & 364123 & 8.6743 & 0.5474 \\ \hline
$Z\rightarrow ee$, $280 < H_{T,parton} < 500$ GeV, CFilterBVeto & 364124 & 8.6711 & 0.2564 \\ \hline
$Z\rightarrow ee$, $280 < H_{T,parton} < 500$ GeV, BFilter & 364125 & 8.6766 & 0.1679 \\ \hline
$Z\rightarrow ee$, $500 < H_{T,parton} < 1000$ GeV & 364126 & 1.8081 & 0.9751 \\ \hline
$Z\rightarrow ee$, $H_{T,parton} > 1000$ GeV & 364127 & 0.14857 & 0.9751 \\ \hline
\end{tabularx}
\end{table}

\begin{table}[htbp]
\centering
\renewcommand\arraystretch{1.5}
\caption{Summary of the $Z/\gamma^* \to \mu^+\mu^-$+jets background samples generated with \texttt{Sherpa~2.2.1}.}
\label{tab:ZjetsSamplesSherpa2}
\scriptsize
\begin{tabularx}{450pt}{|X||c|c|c|}
\hline
\centering \textbf{Process Slice} & \textbf{DSID} & \textbf{$\sigma$ [pb]} & \textbf{k-factor}\\
\hline
$Z\rightarrow \mu\mu$, $H_{T,parton} < 70$ GeV, CVetoBVeto & 364100 & 1983.0 & 0.8016 \\ \hline
$Z\rightarrow \mu\mu$, $H_{T,parton} < 70$ GeV, CFilterBVeto & 364101 & 1978.4 & 0.1103 \\ \hline
$Z\rightarrow \mu\mu$, $H_{T,parton} < 70$ GeV, BFilter & 364102 & 1982.2 & 0.0626 \\ \hline
$Z\rightarrow \mu\mu$, $70 < H_{T,parton} < 140$ GeV, CVetoBVeto & 364103 & 108.92 & 0.6716 \\ \hline
$Z\rightarrow \mu\mu$, $70 < H_{T,parton} < 140$ GeV, CFilterBVeto & 364104 & 109.42 & 0.1813 \\ \hline
$Z\rightarrow \mu\mu$, $70 < H_{T,parton} < 140$ GeV, BFilter & 364105 & 108.91 & 0.1109 \\ \hline
$Z\rightarrow \mu\mu$, $140 < H_{T,parton} < 280$ GeV, CVetoBVeto & 364106 & 39.878 & 0.5938 \\ \hline
$Z\rightarrow \mu\mu$, $140 < H_{T,parton} < 280$ GeV, CFilterBVeto & 364107 & 39.795 & 0.2273 \\ \hline
$Z\rightarrow \mu\mu$, $140 < H_{T,parton} < 280$ GeV, BFilter & 364108 & 39.908 & 0.1425 \\ \hline
$Z\rightarrow \mu\mu$, $280 < H_{T,parton} < 500$ GeV, CVetoBVeto & 364109 & 8.5375 & 0.5451 \\ \hline
$Z\rightarrow \mu\mu$, $280 < H_{T,parton} < 500$ GeV, CFilterBVeto & 364110 & 8.5403 & 0.2587 \\ \hline
$Z\rightarrow \mu\mu$, $280 < H_{T,parton} < 500$ GeV, BFilter & 364111 & 8.4932 & 0.1712 \\ \hline
$Z\rightarrow \mu\mu$, $500 < H_{T,parton} < 1000$ GeV & 364112 & 1.7881 & 0.9751 \\ \hline
$Z\rightarrow \mu\mu$, $H_{T,parton} > 1000$ GeV & 364113 & 0.14769 & 0.9751 \\ \hline
\end{tabularx}
\end{table}

\begin{table}[htbp]
\centering
\renewcommand\arraystretch{1.5}
\caption{Summary of the $Z/\gamma^* \to \tau^+\tau^-$+jets background samples generated with \texttt{Sherpa~2.2.1}.}
\label{tab:ZjetsSamplesSherpa3}
\scriptsize
\begin{tabularx}{450pt}{|X||c|c|c|}
\hline
\centering \textbf{Process Slice} & \textbf{DSID} & \textbf{$\sigma$ [pb]} & \textbf{k-factor}\\
\hline
$Z\rightarrow \tau\tau$, $H_{T,parton} < 70$ GeV, CVetoBVeto & 364128 & 1981.6 & 0.8010 \\ \hline
$Z\rightarrow \tau\tau$, $H_{T,parton} < 70$ GeV, CFilterBVeto & 364129 & 1978.8 & 0.1103 \\ \hline
$Z\rightarrow \tau\tau$, $H_{T,parton} < 70$ GeV, BFilter & 364130 & 1981.8 & 0.0628 \\ \hline
$Z\rightarrow \tau\tau$, $70 < H_{T,parton} < 140$ GeV, CVetoBVeto & 364131 & 110.37 & 0.6717 \\ \hline
$Z\rightarrow \tau\tau$, $70 < H_{T,parton} < 140$ GeV, CFilterBVeto & 364132 & 110.51 & 0.1783 \\ \hline
$Z\rightarrow \tau\tau$, $70 < H_{T,parton} < 140$ GeV, BFilter & 364133 & 110.87 & 0.1081 \\ \hline
$Z\rightarrow \tau\tau$, $140 < H_{T,parton} < 280$ GeV, CVetoBVeto & 364134 & 40.781 & 0.5931 \\ \hline
$Z\rightarrow \tau\tau$, $140 < H_{T,parton} < 280$ GeV, CFilterBVeto & 364135 & 40.74 & 0.2233 \\ \hline
$Z\rightarrow \tau\tau$, $140 < H_{T,parton} < 280$ GeV, BFilter & 364136 & 40.761 & 0.1311 \\ \hline
$Z\rightarrow \tau\tau$, $280 < H_{T,parton} < 500$ GeV, CVetoBVeto & 364137 & 8.5502 & 0.5464 \\ \hline
$Z\rightarrow \tau\tau$, $280 < H_{T,parton} < 500$ GeV, CFilterBVeto & 364138 & 8.6707 & 0.2559 \\ \hline
$Z\rightarrow \tau\tau$, $280 < H_{T,parton} < 500$ GeV, BFilter & 364139 & 8.6804 & 0.1688 \\ \hline
$Z\rightarrow \tau\tau$, $500 < H_{T,parton} < 1000$ GeV & 364140 & 1.8096 & 0.9751 \\ \hline
$Z\rightarrow \tau\tau$, $H_{T,parton} > 1000$ GeV & 364141 & 0.14834 & 0.9751 \\ \hline
\end{tabularx}
\end{table}

% --- Higgs BG Table ---
\begin{table}[htbp]
\centering
\renewcommand\arraystretch{1.5}
\caption{List of background samples from Higgs boson production processes.}
\label{tab:Higgssamples}
\scriptsize
\begin{tabularx}{450pt}{|X||c|c|}
\hline
\centering \textbf{Process} & \textbf{DSID} & \textbf{Generator} \\
\hline
\centering $WH, H\rightarrow WW, W \to q\bar{q}', H \to \ell\nu\ell\nu$  & 346560 & Powheg+Pythia8 \\ \hline 
\centering $WH, H\rightarrow WW, W \to \ell\nu, H \to \ell\nu q\bar{q}'$  & 346561 & Powheg+Pythia8 \\ \hline 
\end{tabularx}
\end{table}

\FloatBarrier

\subsection{Data Samples}

The Run II pp collision data collection by the ATLAS experiment starts from 2015, and ends in 2018. This analysis uses the full Run II pp collision data using Release 21 reconstruction, which pass the final Good Run List (GRL) released by the Data Quality group for 2015-2018. 

\begin{table}[htbp]
\centering
\caption{Good Run Lists (GRLs) used in the analysis.}
\label{tab:grls}
\begin{tabular}{@{}llp{8.5cm}@{}}
\toprule
\textbf{Data Period} & \textbf{Run Range} & \textbf{GRL File Path} \\
\midrule
data15\_13TeV & 276262--284484 & \url{GoodRunsLists/data15_13TeV/20170619/PHYS_StandardGRL_All_Good_25ns_276262-284484_OflLumi-13TeV-008.root} \\
\addlinespace % Adds a bit of vertical space for readability
data16\_13TeV & 297730--311481 & \url{GoodRunsLists/data16_13TeV/20180129/PHYS_StandardGRL_All_Good_25ns_297730-311481_OflLumi-13TeV-009.root} \\
\addlinespace
data17\_13TeV & ---            & \url{GoodRunsLists/data17_13TeV/20180619/physics_25ns_Triggerno17e33prim.lumicalc.OflLumi-13TeV-010.root} \\
\addlinespace
data18\_13TeV & ---            & \url{GoodRunsLists/data18_13TeV/20180924/physics_25ns_Triggerno17e33prim.lumicalc.OflLumi-13TeV-001.root} \\
\bottomrule
\end{tabular}
\end{table}

To ensure a high and robust acceptance for events containing one or more energetic leptons, a selection based on a logical disjunction (OR) of multiple trigger paths is applied. The trigger strategy combines both single-lepton and di-lepton-seeded triggers to maximize efficiency across a broad kinematic phase space. 

This approach is crucial for retaining signal events where individual lepton transverse momenta may fall below the thresholds of the highest-$p_T$ single-lepton triggers. The specific trigger menus are tailored to each data-taking period to accommodate the evolving LHC running conditions and the corresponding adjustments to the ATLAS trigger system. The comprehensive list of electron and muon triggers utilized in this analysis is enumerated in Table~\ref{tab:lepton_triggers}

\begin{table}[h!]
\centering
\caption{Single-lepton triggers used for this analysis by data period and lepton flavor.}
\label{tab:lepton_triggers}
\begin{tabular}{ l p{6.5cm} p{4cm} }
    \hline
    \textbf{Year} & \textbf{Electron Triggers} & \textbf{Muon Triggers} \\
    \hline
    \\[-1.5ex] 
    
    2015 & 
      \begin{tabular}[t]{@{}l@{}}
        HLT\_e24\_lhmedium\_L1EM20VH\_OR \\
        \_e60\_lhmedium\_OR\_e120\_lhloose
      \end{tabular} 
      & 
      \begin{tabular}[t]{@{}l@{}}
        HLT\_mu20\_iloose\_L1MU15\_OR \\
        \_mu50
      \end{tabular} \\
    \\[2ex] 
    
    2016 & 
      \begin{tabular}[t]{@{}l@{}}
        HLT\_e26\_lhtight\_nod0\_ivarloose\_OR \\
        \_e60\_lhmedium\_nod0\_OR \\
        \_e140\_lhloose\_nod0 \\
        HLT\_e24\_lhmedium\_nod0\_L1EM20VH
      \end{tabular} 
      & HLT\_mu24\_ivarmedium\_OR\_mu50 \\
    
    \\[2ex]
    
    2017 & 
      \begin{tabular}[t]{@{}l@{}}
        HLT\_e26\_lhtight\_nod0\_ivarloose\_OR \\
        \_e60\_lhmedium\_nod0\_OR \\
        \_e140\_lhloose\_nod0
      \end{tabular}
      & HLT\_mu26\_ivarmedium\_OR\_mu50 \\
    
    \\[2ex]
    
    2018 & 
      \begin{tabular}[t]{@{}l@{}}
        HLT\_e26\_lhtight\_nod0\_ivarloose\_OR \\
        \_e60\_lhmedium\_nod0\_OR \\
        \_e140\_lhloose\_nod0
      \end{tabular} 
      & HLT\_mu26\_ivarmedium\_OR\_mu50 \\
    \hline
\end{tabular}
\end{table}


\clearpage
%\section{Event Selection}
\section{Object and Event Selection} 
\label{sec:llvv_selection}

To isolate the rare signal process from the vast number of particles produced in proton-proton collisions, a rigorous procedure is required to identify and select the final-state objects of interest. This procedure begins with the reconstruction of physics objects---including electrons, muons, and jets---from the electronic signals collected by the ATLAS detector. Following reconstruction, a stringent set of selection criteria is applied to ensure the high quality of these objects and to suppress contributions from misidentified particles and background processes, thereby enhancing the purity of the event sample.

This chapter details the complete selection strategy, which is performed in two sequential stages. First, the criteria for selecting individual physics objects are defined, forming a clean and calibrated set of inputs for the analysis. Second, these objects are used to construct an event-level selection designed to isolate the distinct signal topology and define the final signal regions.

The specific implementation of these selections adheres to the official recommendations of the ATLAS collaboration and is performed using AnalysisBase release 21.2.164. The analysis is conducted on STDM3 DAOD (Derived Analysis Object Data) samples, with r-tags r9364/r10201/r10724 corresponding to the MC16a/d/e campaigns.

% ======================================================================
\subsection{Object Selection}
\label{sec:llvv_object_selection}

The successful identification of the signal final state relies on the precise reconstruction and selection of its constituent physics objects from the complex collision environment. This section details the criteria applied to select the muons, electrons, and jets that form the basis of the analysis. A final step resolves any spatial ambiguities between these selected objects to ensure each is uniquely defined. All selections adhere to the recommendations provided by the ATLAS Combined Performance groups.

\subsubsection{Muons}
Muons are reconstructed as \textit{combined muons}, which involves matching a track identified in the inner detector (ID) with a corresponding track in the muon spectrometer. This combined fit provides a high-purity and well-measured muon candidate.

Signal muons selected for this analysis are required to satisfy a stringent set of criteria to ensure they are prompt (originating from the primary interaction) and well-isolated.
\begin{itemize}
    \item \textbf{Kinematics:} Muons must have a transverse momentum $p_T > 20$ GeV and be within the detector acceptance of $|\eta| < 2.5$.
    \item \textbf{Identification:} The `Medium` identification working point is required. This selection imposes quality requirements on the track fit and the compatibility between the ID and muon spectrometer measurements, effectively suppressing contamination from misidentified hadrons.
    \item \textbf{Purity:} To reject non-prompt muons from heavy-flavour decays and cosmic-ray muons, requirements are placed on the track's impact parameters with respect to the primary vertex: the transverse impact parameter significance $|d_0/\sigma(d_0)| < 3$ and the longitudinal impact parameter $|z_0 \cdot \sin(\theta)| < 0.5$~mm.
    \item \textbf{Isolation:} To ensure the muon is not part of a jet, an isolation requirement is applied. This is based on the sum of track and calorimeter energy deposits in a cone of $\Delta R = \sqrt{(\Delta\eta)^2+(\Delta\phi)^2} = 0.2$ around the muon. The `FixedCutPflowLoose` working point is used, which corresponds to an efficiency of approximately 99\% for high-$p_T$ muons.
\end{itemize}
To account for differences between data and simulation, dedicated scale and smearing corrections are applied to the muon momentum, and efficiency scale factors are used to correct the Monte Carlo event yields. In addition, a looser "baseline" selection ($p_T > 7$ GeV, `Loose` identification) is used to define muons for event vetoes.

\subsubsection{Electrons}
Electrons are reconstructed by matching a track in the inner detector to an energy cluster in the electromagnetic calorimeter. Signal electrons are selected based on the following criteria:
\begin{itemize}
    \item \textbf{Kinematics:} Electrons must have $p_T > 20$ GeV and be reconstructed within the fiducial pseudorapidity range of $|\eta| < 2.47$.
    \item \textbf{Identification:} A likelihood-based discriminant, which combines information from shower shapes, track quality, and track-cluster matching, is used for identification. The `Medium` working point is chosen to provide a high signal efficiency with strong rejection of jets misidentified as electrons.
    \item \textbf{Purity:} Similar to muons, cuts on the impact parameters are applied to ensure electrons originate from the primary vertex: $|d_0/\sigma(d_0)| < 5$ and $|z_0 \cdot \sin(\theta)| < 0.5$~mm.
    \item \textbf{Isolation:} The `FixedCutPflowLoose` isolation working point, defined in a cone of $\Delta R = 0.2$, is required to reject non-prompt electrons and ensure they are well-separated from other particles.
    \item \textbf{Quality Veto:} For the 2015-2016 data-taking period, events containing an electron in a known problematic region of the electromagnetic calorimeter crack ($1.37 < |\eta| < 1.52$) are vetoed to prevent mismeasurement of the event's missing transverse momentum.
\end{itemize}
Electron energy scale and resolution corrections, as well as efficiency scale factors, are applied to the simulation to ensure accurate modeling of the detector performance. A "baseline" electron selection ($p_T > 7$ GeV, `LooseLHB` identification) is used for veto purposes.

\subsubsection{Jets}
Jets provide crucial information about the hadronic activity in the event and are key to defining the VBS-like signal region.
\begin{itemize}
    \item \textbf{Reconstruction:} Jets are reconstructed from particle-flow objects using the anti-$k_T$ algorithm with a radius parameter of $R=0.4$.
    \item \textbf{Kinematics and Cleaning:} Jets are selected if they have $p_T > 30$ GeV and $|\eta| < 4.5$. They must also pass the `Loose` jet cleaning criteria, which are designed to remove spurious jets arising from detector noise or non-collision backgrounds.
    \item \textbf{Pileup Suppression:} To mitigate the impact of additional proton-proton interactions (pileup), jets within the tracker acceptance ($|\eta| < 2.4$) and with $p_T < 60$ GeV must satisfy a requirement on the Jet Vertex Tagger (JVT) discriminant.
    \item \textbf{B-Jet Veto:} To suppress the large background from top-quark pair production, events containing b-jets are vetoed. Jets are identified as b-jets using the `DL1r` multivariate tagging algorithm. Any event containing a jet within $|\eta| < 2.5$ that passes the 85\% efficiency working point is rejected.
\end{itemize}

\subsubsection{Overlap Removal}
A single detector signature can sometimes be reconstructed as multiple physics objects. To resolve these ambiguities and prevent double-counting, a sequential overlap removal procedure is applied to the baseline object collections. The procedure follows a prescribed hierarchy based on the most likely identity of the shared signature.
\begin{enumerate}
    \item Jets within $\Delta R < 0.2$ of a selected electron are removed.
    \item Electrons that share an ID track with a selected muon are removed.
    \item Jets within $\Delta R < 0.2$ of a selected muon are removed if they have few associated tracks, which is characteristic of muons depositing energy in the calorimeter.
    \item Finally, any electrons or muons found within a wider cone of $\Delta R < 0.4$ around any surviving jet are removed. This step effectively rejects non-prompt leptons originating from heavy-flavour hadron decays inside jets.
\end{enumerate}
After this procedure, the final signal selections for leptons and jets are applied to the surviving, uniquely identified objects.

% ======================================================================
\subsection{Event Selection}
\label{sec:llvv_event_selection}

With a foundation of calibrated physics objects, the analysis proceeds to the crucial stage of event selection. The goal is to isolate the distinctive signature of \(Z \rightarrow \ell\ell\nu\nu\) production from an overwhelming background of other Standard Model processes. The final state is characterized by two same-flavour, opposite-sign leptons consistent with the decay of a Z boson, accompanied by significant missing transverse momentum ($E_T^\text{miss}$) due to the two neutrinos from the second Z boson's decay, which escape detection.

The primary challenge in this channel is to distinguish the large, genuine $E_T^\text{miss}$ originating from the invisible Z decay from the fake $E_T^\text{miss}$ that arises from jet energy mismeasurements and instrumental effects, which is prevalent in the dominant Drell-Yan (Z+jets) background. Therefore, the event selection strategy is centered around stringent requirements on the $E_T^\text{miss}$ magnitude and its significance. This is complemented by topological requirements that exploit the expected event kinematics, such as the angular separation between the visible Z boson and the $E_T^\text{miss}$ vector. Furthermore, dedicated vetoes are employed to suppress other key backgrounds, including a b-jet veto to reject top-quark events and a veto on additional leptons to reduce contributions from WZ and \(ZZ \rightarrow 4\ell\) processes.

The strategy is to build the selection incrementally, applying successive requirements that target the distinct features of the signal while systematically rejecting specific backgrounds. This process culminates in the definition of two primary signal regions (SRs): an \textit{inclusive} SR for the overall cross-section measurement, and a \textit{VBS-like} SR targeting the electroweak production mode.

\subsubsection{Baseline Event Requirements and Z Boson Candidate}
The initial step in the selection process is to identify a viable candidate for the leptonic Z boson decay, \(Z \rightarrow \ell\ell\). This forms the cornerstone of the event signature.
\begin{itemize}
    \item \textbf{Data Quality and Trigger:} Events are first required to pass standard data quality checks to ensure all detector components were functioning correctly. They must also have fired a single-lepton trigger, which provides the initial, high-efficiency selection of events containing at least one high-$p_T$ electron or muon.
    \item \textbf{Primary Vertex:} A primary vertex with at least two associated tracks is required, ensuring that the event originates from a genuine proton-proton collision.
    \item \textbf{Dilepton Final State:} The core of the selection requires the presence of exactly two signal leptons (as defined in \ref{sec:llvv_object_selection}) of the same flavour and opposite charge (SFOS). To ensure trigger efficiency is high, the leading and subleading leptons are required to have $p_T > 30$ GeV and $p_T > 20$ GeV, respectively. Events containing any additional "baseline" leptons ($p_T > 7$ GeV) are vetoed. This veto is crucial for suppressing backgrounds with three or more real leptons, primarily from WZ and \(ZZ \rightarrow 4\ell\) production.
    \item \textbf{On-Shell Z Boson:} To select events consistent with a \(Z \rightarrow \ell\ell\) decay, the invariant mass of the dilepton pair is required to be within a window around the Z boson mass: $80 < m_{\ell\ell} < 100$ GeV. This is a powerful cut that significantly reduces non-resonant backgrounds such as \(t\bar{t}\) and WW production.
\end{itemize}

\subsubsection{Targeting the Invisible Z Decay and Suppressing Drell-Yan}
After identifying a clean, on-shell Z boson candidate, the selection must target the signature of the second Z boson decaying to neutrinos: large missing transverse momentum ($E_T^\text{miss}$). The primary challenge here is rejecting the dominant Drell-Yan (Z+jets) background, where large, fake $E_T^\text{miss}$ can arise from the mismeasurement of jet energies. A series of topological and kinematic cuts are designed specifically for this purpose.
\begin{itemize}
    \item \textbf{Missing Transverse Momentum:} A significant amount of genuine $E_T^\text{miss}$ is the key feature of the signal. Therefore, a high threshold is placed on its magnitude.
    \item \textbf{Lepton Collimation:} In the signal process, the \(Z\rightarrow\ell\ell\) boson often has a high transverse momentum, recoiling against the invisible Z boson. This boost causes its decay products to be more collimated. A requirement of $\Delta R(\ell,\ell) < 1.8$ is applied to exploit this feature, which preferentially rejects Drell-Yan events where the Z boson typically has lower $p_T$.
    \item \textbf{Topological Correlation:} In signal events, the visible Z boson and the invisible neutrinos are expected to be produced back-to-back in the transverse plane. This results in a large azimuthal separation between the dilepton momentum vector and the $E_T^\text{miss}$ vector. A cut of $\Delta\Phi(Z, E_T^\text{miss}) > 2.2$ is imposed to select events with this topology, strongly suppressing Z+jets events where fake $E_T^\text{miss}$ from a mismeasured jet is often not aligned opposite to the Z boson.
    \item \textbf{$E_T^\text{miss}$ Significance:} To further distinguish genuine $E_T^\text{miss}$ from instrumental effects, the ratio of the missing transverse momentum to the scalar sum of the transverse momenta of all selected objects, $H_T$, is used. For signal events, a large fraction of the total energy is invisible. A requirement of $E_T^\text{miss} / H_T > 0.65$ effectively rejects events with high hadronic activity where the $E_T^\text{miss}$ is small relative to the total visible energy.
    \item \textbf{B-Jet Veto:} The production of top-quark pairs ($t\bar{t}$) is a significant background, as it can produce two leptons and genuine $E_T^\text{miss}$ from W boson decays. This background is effectively suppressed by vetoing any event that contains one or more b-tagged jets.
\end{itemize}

\subsubsection{Signal Region Definitions}
The selection criteria described above were optimized to maximize signal significance and are now combined to define the two signal regions for the analysis.
\paragraph{Inclusive Signal Region}
This region is designed to measure the total ZZ production cross-section in this final state. It applies all the selection criteria developed above. The defining requirement on the missing transverse momentum is:
\begin{itemize}
    \item $E_T^\text{miss} > 110$ GeV.
\end{itemize}

\paragraph{VBS-like Signal Region}
This region is tailored to enhance the contribution from electroweak ZZ production in association with two jets (a signature of Vector Boson Scattering). It builds upon the inclusive selection but requires a more energetic and hadronic final state.
\begin{itemize}
    \item The missing transverse momentum requirement is tightened to $E_T^\text{miss} > 150$ GeV to select more energetic events characteristic of VBS.
    \item A requirement of at least two jets is imposed, with the leading and subleading jets required to have $p_T > 30$ GeV. This explicitly selects the desired ZZ+2-jets topology.
\end{itemize}
The specific cut values for all variables were chosen following an optimization procedure aimed at maximizing the expected signal significance.


% ======================================================================

%this subsection deatils the truth-level selection(which is called the phase space). 
%this subsection shall include the comparison and meaning of reco and truth selections. 
\subsubsection{Fiducial Phase Space Definition}
\label{sec:fiducial_definition}

The ultimate goal of this analysis is to measure the $ZZ$ production cross-section. However, a detector-level measurement is
inherently \textbf{convolved with} the detector's response, its geometric acceptance, and the efficiencies of the reconstruction and selection algorithms. To unfold these effects and present a result that can be
directly compared with theoretical calculations, the measurement is performed
within a well-defined \textbf{fiducial phase space}.

Defining this fiducial volume is a critical step that serves two main purposes. First, it provides a model-independent measurement target. By restricting the measurement to a phase space accessible to the detector, we minimize the reliance on theoretical models to extrapolate into regions with no experimental sensitivity, thereby reducing the associated theoretical uncertainties. Second, it provides the necessary framework for calculating the correction factors that relate the observed number of events at the detector level to the true number of events produced in the collisions.

The fiducial volume is defined using particle-level ("truth") kinematics from the Monte Carlo event generator. The selection criteria are chosen to be close to the reconstruction-level requirements but are intentionally relaxed. This ensures that the detector-level selection is fully contained within the fiducial volume, which provides a stable basis for calculating the selection efficiency and minimizes migrations at the selection boundaries. For this analysis, the requirements on the missing transverse momentum and the dilepton invariant mass are notably relaxed at the truth level.

\paragraph{Truth Object Definitions}
Before defining the event selection, the particle-level objects are constructed as follows:
\begin{itemize}
    \item \textbf{Dressed Leptons:} To account for final-state QED radiation, the four-momenta of all prompt photons within a cone of $\Delta R(\ell, \gamma) < 0.1$ around a prompt electron or muon are added to the lepton's four-momentum. This procedure creates "dressed" leptons, which provide a more realistic proxy for the energy measured in the electromagnetic calorimeter.
    \item \textbf{Truth Neutrinos:} The truth missing transverse momentum ($E_T^\text{miss, truth}$) is calculated as the magnitude of the vector sum of the transverse momenta of all prompt neutrinos in the event.
    \item \textbf{Truth Jets:} Jets are clustered from all stable final-state particles (excluding the dressed leptons and all neutrinos) using the anti-$k_T$ algorithm with a radius parameter of $R=0.4$.
\end{itemize}

\paragraph{Fiducial Selection for Inclusive Production}
For the inclusive $ZZ \rightarrow \ell\ell\nu\nu$ cross-section measurement, the fiducial volume is defined by the following requirements on the truth-level objects:
\begin{itemize}
    \item Exactly two same-flavour, opposite-sign (SFOS) dressed leptons with $|\eta^\ell| < 2.5$.
    \item Lepton transverse momenta of $p_T^\ell > 30$ GeV for the leading and $p_T^\ell > 20$ GeV for the subleading lepton.
    \item The dilepton invariant mass must be within the range $76 < m_{\ell\ell} < 106$ GeV.
    \item The truth missing transverse momentum must be $E_T^\text{miss, truth} > 95$ GeV.
    \item The dilepton angular separation must be $\Delta R(\ell,\ell) < 1.8$.
    \item The azimuthal separation between the dilepton system and the missing momentum must be $\Delta\Phi(\vec{p}_T^{\ell\ell}, \vec{E}_T^\text{miss, truth}) > 2.2$.
    \item The ratio of missing to total transverse momentum must be $E_T^\text{miss, truth} / H_T^\text{truth} > 0.65$.
\end{itemize}

\paragraph{Fiducial Selection for VBS-like Production}
For the $ZZjj \rightarrow \ell\ell\nu\nu jj$ cross-section measurement, the fiducial volume builds upon the inclusive selection with additional requirements on the hadronic activity:
\begin{itemize}
    \item All inclusive fiducial selection criteria are applied, with the exception of the truth missing transverse momentum, which is tightened to $E_T^\text{miss, truth} > 130$ GeV.
    \item The event must contain at least two truth jets with $p_T > 30$ GeV and $|\eta| < 4.5$.
\end{itemize}

\paragraph{Comparison of Detector-Level and Particle-Level Selections}
To clearly illustrate the relationship between the final signal region definitions and their corresponding fiducial volumes, a detailed comparison of the selection criteria is provided in Tables~\ref{tab:reco_truth_inclusive} and \ref{tab:reco_truth_vbs}. The deliberate relaxation of the particle-level cuts, particularly for $m_{\ell\ell}$ and $E_T^\text{miss}$, is highlighted. This strategy ensures that detector resolution effects do not migrate a significant fraction of signal events from inside the reconstructed selection to outside the fiducial volume, which would lead to an unstable efficiency calculation.

\begin{table}[!htbp]
    \centering
    \renewcommand{\arraystretch}{1.3}
    \caption{Comparison of selection criteria for the \textbf{Inclusive Signal Region} at detector-level (Reco) and its corresponding particle-level (Truth) fiducial volume. Key differences are highlighted in \textbf{bold}.}
    \label{tab:reco_truth_inclusive}
    \begin{tabular}{l|c|c}
        \hline \hline
        \textbf{Requirement} & \textbf{Detector-Level (Reco)} & \textbf{Particle-Level (Truth)} \\
        \hline
        Leptons & \multicolumn{2}{c}{Exactly 2 SFOS leptons} \\
        Lepton $p_T$ [GeV] & \multicolumn{2}{c}{$> 30$ (lead), $> 20$ (sublead)} \\
        Lepton $|\eta|$ & $< 2.5$ ($\mu$); $< 2.47$ ($e$) & $< 2.5$ ($e, \mu$) \\
        Electron $|\eta|$ Veto & \textbf{Veto $1.37 < |\eta| < 1.52$} & \textbf{Not Applied} \\
        Dilepton Mass ($m_{\ell\ell}$) [GeV] & $80 < m_{\ell\ell} < 100$ & \textbf{$76 < m_{\ell\ell} < 106$} \\
        \hline
        Missing $E_T$ ($E_T^\text{miss}$) [GeV] & $> 110$ & \textbf{$> 95$} \\
        Dilepton $\Delta R(\ell,\ell)$ & $< 1.8$ & $< 1.8$ \\
        $\Delta\Phi(\vec{p}_T^{\ell\ell}, \vec{E}_T^\text{miss})$ & $> 2.2$ & $> 2.2$ \\
        $E_T^\text{miss} / H_T$ & $> 0.65$ & $> 0.65$ \\
        \hline
        b-jet Veto & \textbf{Required} & \textbf{Not Applied} \\
        \hline \hline
    \end{tabular}
\end{table}

\begin{table}[!htbp]
    \centering
    \renewcommand{\arraystretch}{1.3}
    \caption{Comparison of selection criteria for the \textbf{VBS-like Signal Region} at detector-level (Reco) and its corresponding particle-level (Truth) fiducial volume. Differences are highlighted in \textbf{bold}.}
    \label{tab:reco_truth_vbs}
    \begin{tabular}{l|c|c}
        \hline \hline
        \textbf{Requirement} & \textbf{Detector-Level (Reco)} & \textbf{Particle-Level (Truth)} \\
        \hline
        Inclusive Selection Base & Applied & Applied \\
        \hline
        Missing $E_T$ ($E_T^\text{miss}$) [GeV] & $> 150$ & \textbf{$> 130$} \\
        Number of Jets ($p_T > 30$ GeV) & $\geq 2$ & $\geq 2$ \\
        Jet $|\eta|$ & \textbf{Depends on $p_T$ (e.g., $< 2.5$ or $< 4.5$)} & \textbf{$< 4.5$} \\
        \hline \hline
    \end{tabular}
\end{table}

The fiducial selections defined here provide the basis for calculating a correction factor, $C$, which maps the observed detector-level yield to the particle-level fiducial yield. This factor, derived from simulation, corrects the background-subtracted data for detector inefficiencies and resolution-induced migrations. It is calculated as the ratio of the number of simulated signal events passing the full reconstruction-level selection ($N_\text{reco}^\text{MC}$) to the number of simulated signal events passing the particle-level fiducial selection ($N_\text{fid}^\text{MC}$):
$$
C = \frac{N_\text{reco}^\text{MC}}{N_\text{fid}^\text{MC}}
$$
The fiducial cross-section, $\sigma_\text{fid}$, is then determined by:
$$
\sigma_\text{fid} = \frac{N_\text{obs} - N_\text{bkg}}{\mathcal{L}} \cdot \frac{1}{C}
$$
where $N_\text{obs}$ is the observed number of events in the signal region, $N_\text{bkg}$ is the estimated background contribution, and $\mathcal{L}$ is the integrated luminosity. This approach isolates the detector-dependent corrections ($C$) from any model-dependent extrapolations to the full phase space, which are not part of this measurement.






\clearpage
%\section{Background Estimation}
\section{Background Estimation}

In the measurement of the \( ZZ \to \ell^+\ell^-\nu\bar{\nu} \) cross-section, several background processes contribute. These can be categorized as follows:

\begin{itemize}

    % --- FIRST TOP-LEVEL ITEM ---
    \item \textbf{Irreducible Backgrounds}
    \begin{itemize}
        \item \textbf{Diboson processes:}
        \begin{itemize}
            \item \( WW \to \ell\nu\ell\nu \): The two neutrinos contribute to the MET signature, faking the \( ZZ \to \ell\ell\nu\nu \) process.
            \item \( WZ \to \ell\nu\ell\ell \): If one lepton from the \( Z \) is missing, this process could fake the signal.
            \item \( ZZ \to \ell\ell\ell\ell \): If two leptons escape detection, this process contributes as a background.
        \end{itemize}
    \end{itemize}

    % --- SECOND TOP-LEVEL ITEM ---
    \item \textbf{Reducible Backgrounds}
    \begin{itemize}
        \item \textbf{Top-quark backgrounds:}
        \begin{itemize}
            % THIS IS THE LINE THAT HAD THE ERROR
            \item \( t\bar{t} \): If b-jets in the \( t\bar{t} \) final states are not tagged properly, it may fake the signal.
            \item Single top \( (tW) \): Can contribute similarly.
        \end{itemize}
        \item \textbf{Drell-Yan + jets} (\( Z \to \ell\ell \) + jets): Jet energy mismeasurement can lead to artificial MET.
        \item \textbf{W + jets}: A single \( W \) boson decaying leptonically with jets can mimic the signal region due to fake MET.
    \end{itemize}

\end{itemize}

To achieve precise cross-section measurements, these backgrounds must be carefully estimated and subtracted using control regions. Considering the final states of the backgrounds, they can be categorized as following Control Regions(CR).

\begin{itemize}
    \item \textbf{Non-Resonant Backgrounds (Diboson and Top processes without a $Z$ boson)}
    \begin{itemize}
        \item \( WW \to \ell\nu\ell\nu \): The two neutrinos contribute to the MET signature, faking the \( ZZ \to \ell\ell\nu\nu \) process.
        \item \( t\bar{t} \to WbWb \to \ell\nu b \ell\nu b \): If b-jets are not tagged properly, it may fake the signal.
        \item Single top \( (tW) \): Can contribute similarly.
    \end{itemize}

    \item \textbf{Non-Resonant + 1b-Jet Backgrounds (Top backgrounds with a b-jet)}
    \begin{itemize}
        \item \( t\bar{t} \to WbWb \to \ell\nu b \ell\nu b \): Events with one identified b-jet contribute to this category.
    \end{itemize}
 
    
    \item \textbf{3L Backgrounds (Processes with three leptons in the final state)}
    \begin{itemize}
        \item \( WZ \to \ell\nu\ell\ell \): If one lepton from the \( Z \) is lost, this process mimics the signal.
    \end{itemize}
    
    \item \textbf{Z+Jets Backgrounds (Processes with a $Z$ boson and jets)}
    \begin{itemize}
        \item \textbf{Drell-Yan + jets} (\( Z/\gamma^* \to \ell\ell \) + jets): If jets are mismeasured, they create fake MET.
        \item \textbf{W + jets}: A single \( W \) boson decaying leptonically with jets can mimic the signal region due to fake MET.
    \end{itemize}
   
    \item \textbf{Instrumental Backgrounds (Detector-related effects)}
    \begin{itemize}
        \item \textbf{Fake MET from mismeasured jets:} Jet energy mismeasurement can lead to artificial MET.
        \item \textbf{Cosmic-ray muons or beam halo:} Rare cases where fake leptons or MET appear in the detector.
    \end{itemize}
\end{itemize}

To accurately estimate these background contributions, control regions (CRs) are defined, each enriched with a dominant background category. A simultaneous fit is performed across all CRs and the signal region to constrain both the yields and shapes of these backgrounds. This fit takes event distributions as input and derives scale factors to adjust Monte Carlo (MC) predictions to match data in the CRs. The scaled MC predictions then provide the background estimates in the signal region, ensuring a reliable determination of the differential cross-section.

To further refine background estimation, each control region is designed to enhance a specific type of background, allowing us to isolate and constrain individual contributions effectively. The simultaneous fit across all CRs and the signal region ensures a consistent estimation of background yields and shapes. By utilizing event distributions, this fit derives scale factors that adjust MC predictions to align with data in all CRs. These scaled MC predictions then serve as the background estimates in the signal region, providing a foundation for an accurate differential cross-section measurement.

The following sections describe the estimation strategy for each control region in detail.



\subsection{Non-Resonant with/without 1 b-jet)}

Considering the high percentage of the non-resonant backgrounds, it would be better if Control Regions are well defined to control the backgrounds from MC estimations. Therefore, 2 Control Regions are defined to further estimate the yields from those processes. 

One is the non-resonant background, whcih requires Opposite Flavor Opposite Sign(OFOS) leptons in the final state. For example, in the \( WW \to \ell\nu\ell\nu \) process, the observable final state could be either Same Flavor Opposite Sign(SFOS) leptons, which fakes the true signals, or Opposite Flavor Opposite Sign(OFOS) leptons. Considering that there is an equal chance for W boson decaying to electron or muon or tau, the yield ratio between SFOS and OFOS final states would be 1:2. Since the OFOS events are well seperated from the SFOS, this would be a good approach to estimate the background yields in Signal Region(SR) using the CR events. 

However, since the final state from top quark pair or single top decay contains 1 or 2 b jets, the observed final state could contain 1 or more b jets. Therefore define a CR with 1 b jet to further estimate the yield from the top backgrounds. This non-resonant with a b-jet CR defination is the same as the common non-resonant CR, except the b-jet number requirement. 

Since this is aimming to estimate the yields in the SR, the non-resonant background definition is the same as the SR, except requiring the OFOS events. the detailed definations of non-resonant with/without 1 b-jet are shown in the following table. 



This is a table TBA















\subsection{3l}











\subsection{Z+jet}











\subsection{Others}





\clearpage
%\section{Systematic Uncertainty}
\section{Systematic Uncertainties}
\label{sec:llvv_systematics}

This section summarizes the experimental and theoretical sources of systematic uncertainty affecting the event yields of the backgrounds and the signal. These sources impact the differential estimate of the signal cross-section after the unfolding procedure. 

The systematic uncertainties are assessed to account for residual discrepancies between data and simulation in the modelling of trigger efficiencies, lepton reconstruction, jet calibration, $b$-tagging, missing transverse energy ($E_{\text{T}}^{\text{miss}}$), luminosity, and the average number of proton--proton interactions per bunch crossing (pile-up).

Consistent with previous analyses of the $ZZjj$ final state, the measurement of the signal yields and resulting cross-sections is dominated by statistical uncertainty. Among the sources of systematic uncertainty, the largest contributions arise from jet reconstruction—specifically pile-up suppression and $\eta$-intercalibration—and from the unfolding bias. 
%The bias associated with the unfolding method is evaluated through the data-driven closure test discussed in Section~\ref{sec:unfolding_ddclosure}.

\subsection{Experimental Sources}
\label{subsec:exp_syst}

The experimental systematic uncertainties encompass detector effects and reconstruction efficiencies. For each source, the propagation to the final measurement is performed as follows:
\begin{enumerate}
    \item A detector-level Standard Model (SM) distribution is constructed using Monte Carlo (MC) predictions where the specific systematic parameter has been varied.
    \item The nominal background is subtracted from this varied distribution.
    \item The resulting background-subtracted distribution is unfolded using the \textit{nominal} response matrix.
    \item The systematic uncertainty is defined as the difference between this unfolded distribution and the nominal MC particle-level (truth) distribution.
\end{enumerate}

To ensure that upward and downward variations are statistically significant and symmetric, bootstrap replicas are utilized. In cases where the variations are symmetric within statistical uncertainties, the final systematic value is smoothed by averaging the absolute values of the variations.

The specific sources considered are listed below:

\begin{itemize}
    \item \textbf{Luminosity:} 
    The total integrated luminosity of the dataset is known to a precision of $0.83\%$. This uncertainty is determined using a combination of van der Meer beam separation scans in dedicated data-taking sessions and calibrated luminosity-sensitive detectors during regular data-taking periods~\cite{DAPR-2021-01}.
    
    \item \textbf{Pile-up Reweighting:} 
    The number of proton--proton interactions per bunch crossing is modelled in simulation by reweighting events to match the distribution observed in data. An uncertainty arises from the limited precision in the ratio of predicted to measured inelastic cross-sections within the ATLAS acceptance region~\cite{STDM-2015-05}. To account for this, a systematic variation is applied to the average number of interactions in the simulation via the nuisance parameter \texttt{PRW\_DATASF}.

    \item \textbf{Trigger Efficiency:} 
    Uncertainties in the trigger efficiency scale factors are separated by lepton flavor. For electrons, a single total nuisance parameter is used. For muons, the uncertainty is split into statistical and systematic components.

    \item \textbf{Lepton Efficiency (Reconstruction, ID, Isolation):} 
    Efficiencies for lepton reconstruction, identification, and isolation are corrected in simulation using scale factors derived from tag-and-probe control samples in data.
    \begin{itemize}
        \item \textbf{Electrons:} Modeled using a simplified de-correlation model containing 25 nuisance parameters for reconstruction, plus parameters for isolation and identification.
        \item \textbf{Muons:} Involves four uncertainty components covering statistical and systematic terms for both general and low-$p_{\text{T}}$ regimes, plus two parameters for isolation.
    \end{itemize}

    \item \textbf{Lepton Momentum and Scale:} 
    Differences between data and simulation in the lepton momentum scale and resolution are evaluated by applying additional scaling and smearing variations.
    \begin{itemize}
        \item \textbf{Electrons:} Two parameters, \texttt{EG\_RESOLUTION\_ALL} and \texttt{EG\_SCALE\_ALL}, account for resolution and scale uncertainties.
        \item \textbf{Muons:} Independent parameters are used for Inner Detector and Muon Spectrometer measurements, along with Sagitta correction uncertainties and momentum scale parameters.
    \end{itemize}

    \item \textbf{Jet Energy Scale (JES) and Resolution (JER):} 
    Given the hadronic nature of the VBS topology (two forward jets), jet systematics are a dominant source of uncertainty.
    \begin{itemize}
        \item \textbf{JES:} Derived using test-beam data, LHC collision data, and simulation. The \texttt{GlobalReduction} configuration is employed, corresponding to 23 distinct nuisance parameters.
        \item \textbf{JER:} Evaluated by applying additional smearing to the jet energy in simulated samples to match the resolution observed in data. The \texttt{FullJER} configuration (13 parameters) is used.
    \end{itemize}
    
    \item \textbf{Jet Vertex Tagger (JVT):} 
    Scale factor variations for the JVT are applied to account for uncertainties in pile-up jet suppression. Consistent with previous measurements, these variations result in a minor impact ($< 1\%$).

    \item \textbf{Flavor Tagging ($b$-tagging):}
    The uncertainty in the $b$-tagging efficiency, as well as the corresponding uncertainties in the $c$-jet and light-jet rejection scale factors, are taken into account~\cite{FTAG-2018-01}. These are propagated to the measurement through variations of the flavor-tagging scale factors.

    \item \textbf{Missing Transverse Energy ($E_{\text{T}}^{\text{miss}}$):}
    Uncertainties in $E_{\text{T}}^{\text{miss}}$ are evaluated by varying the soft term (signals not associated with high-$p_T$ objects) and by propagating the lepton and jet energy scale and resolution uncertainties to the $E_{\text{T}}^{\text{miss}}$ calculation.
\end{itemize}

\subsection{Theoretical Sources}
\label{subsec:theo_syst}

Theoretical variations typically induce only minor changes in the shapes of observables but can significantly impact predicted yields. These uncertainties are process-specific.

\begin{itemize}
    \item \textbf{QCD Scale Uncertainties:} 
    These are evaluated using on-the-fly weights corresponding to variations of the renormalization ($\mu_R$) and factorization ($\mu_F$) scales. Seven variations are considered: combinations of $\mu_R$ and $\mu_F$ at factors of 0.5, 1, and 2, excluding the extreme cases where one scale is halved and the other doubled. The envelope of these variations defines the uncertainty. Crucially, for the $ZZjj$ signal, the QCD-induced and EW-induced components are treated separately with \textbf{uncorrelated} scale uncertainties.
        
    \item \textbf{PDF and $\alpha_S$:} 
    Uncertainties are evaluated following the PDF4LHC prescription~\cite{Butterworth:2015oua}. This includes the RMS of 100 internal replicas of the NNPDF3.0 set. The effect of the strong coupling constant is assessed separately by varying $\alpha_s$ by $\pm 0.001$ and adding the result in quadrature.

    \item \textbf{MC Generator Modelling:} 
    To assess the uncertainty associated with the MC generator choice for the QCD-induced signal component, the nominal \textsc{Sherpa} sample is compared to an alternative sample generated with \textsc{Powheg}. The relative difference in the unfolded cross-section is taken as a systematic uncertainty.
    
    \item \textbf{Parton Shower:}     
    Uncertainties in the parton showering are calculated within \textsc{Sherpa} by varying the shower recoil scheme. Further details are documented in \ref{llvv_partonshower}.

    \item \textbf{$Z$+jets Background Normalization:}
    In the $ZZjj$ phase space, the $Z+\text{jets}$ background is estimated using MC simulation. A conservative normalization uncertainty of $50\%$ is assigned to this background to account for potential mismodelling in events with high jet multiplicity. This value is motivated by discrepancies observed in inclusive $Z+\geq 2$ jets control regions.
\end{itemize}

\subsection{Summary of Impact}

The experimental and theoretical uncertainties are incorporated into the analysis through variations in relevant nuisance parameters in simultaneous fits to data. The impact of these systematics on the measured fiducial cross-sections for the inclusive ($ZZ \to \ell\ell\nu\nu$) and VBS-enriched ($ZZjj \to \ell\ell\nu\nu jj$) phase spaces is summarized in Table~\ref{tab:uncertainties}.

\begin{table}[hbtp]
\caption{The breakdown of systematic uncertainties on the measured fiducial cross-sections $\sigma_{ZZ \to \ell \ell \nu\nu}$ and $\sigma_{ZZjj \to  \ell \ell \nu\nu jj}$. The values represent the relative uncertainty on the measured cross-section.}
\centering
\scriptsize
\renewcommand{\arraystretch}{1.2}
\begin{tabular}{lcc}
\toprule
\textbf{Systematic source} & \textbf{Impact on $\sigma_{ZZ \to \ell \ell \nu\nu}$} & \textbf{Impact on $\sigma_{ZZjj \to \ell \ell \nu\nu jj}$} \\
\midrule
Scale variation                          & 0.99\%  & 3.5\%   \\
PDF and $\alpha_s$          & 0.15\%  & 0.42\%  \\
Alternative signal generator & 0.38\%  & 0.93\%  \\
Z+jets modelling             & --      & 0.43\%  \\
\midrule
Jet energy scale and resolution & 1.7\%   & 8.7\%   \\
MC statistics                & 1.6\%   & 2.9\%   \\
Pile-up reweighting        & 0.39\%  & 4.1\%   \\
$E_\mathrm{T}^\text{miss}$  & 1.4\%   & 0.97\%  \\
Electrons     & 0.38\%  & 0.24\%  \\
Muons         & 0.35\%  & 0.42\%  \\
$b$-tagging                 & 0.28\%  & 1.2\%   \\
\midrule
Luminosity                  & 0.87\%  & 0.82\%  \\
\midrule
Total systematics             & 3.1\%   & 11\%   \\
\midrule
Statistics             & 3.6\%   & 14\%   \\
\midrule
\textbf{Total Uncertainty}            &  \textbf{4.8\%}  &  \textbf{18\%}   \\
\bottomrule
\end{tabular}
\label{tab:uncertainties}
\end{table}




\clearpage
%\section{Unfolding}
\section{Unfolding}
\label{sec:llvv_unfolding}

In measurements of differential cross-sections, the observed distributions are inevitably distorted by detector effects. These include the finite resolution of the detector, limited geometric acceptance, and reconstruction inefficiencies. To facilitate direct comparison with theoretical predictions and results from other experiments, these detector-level measurements must be corrected to the particle level. This correction process is referred to as \textit{unfolding}.

The unfolding procedure relies on Monte Carlo (MC) simulations, which provide a mapping between the particle level (truth) and the reconstruction level (detector). This mapping is used to construct a response matrix that models the probability of migration between bins, as well as efficiency and acceptance corrections.

\subsection{Detector Response Model}

The relationship between the true physical observable and the measured quantity is characterized by several key components derived from the MC simulation.

\subsubsection{Migration Matrix}
The measured value of an observable often differs from the true value due to the non-zero resolution of the detector elements. In binned distributions, if this deviation is significant relative to the bin width, an event generated in bin $i$ may be reconstructed in bin $j$.

The **Migration Matrix** ($M$) quantifies this effect. Each element $M_{ij}$ represents the probability that an event generated in the $i$-th bin at the particle level is reconstructed in the $j$-th bin. This matrix is populated using MC events that satisfy both the fiducial (truth) and reconstruction (reco) selection criteria.

\subsubsection{Purity and Stability}
To ensure a robust unfolding, the binning choice must be optimized such that the migration matrix remains diagonally dominant. Two metrics are defined to quantify this quality:

\begin{itemize}
    \item \textbf{Purity ($p_j$):} The fraction of events reconstructed in a specific bin $j$ that truly originated from that same bin at the particle level.
    \begin{align}
        p_j=\frac{M_{jj}}{\sum_{i}^{\text{reco bins}}M_{ij}}
    \end{align}
    \item \textbf{Stability ($s_i$):} The fraction of events generated in a specific bin $i$ that are also reconstructed in that same bin.
    \begin{align}
        s_i=\frac{M_{ii}}{\sum_{j}^{\text{truth bins}}M_{ij}}
    \end{align}
\end{itemize}

High values of purity and stability indicate that the chosen bin widths are large enough compared to the detector resolution to minimize bin-to-bin migration.

\subsubsection{Efficiency and Acceptance Corrections}
In addition to bin migration, the unfolding procedure must correct for event losses and spurious inclusions:

\begin{itemize}
    \item \textbf{Reconstruction Efficiency ($\epsilon_i$):} This factor accounts for signal events occurring within the fiducial volume that are not reconstructed due to detector inefficiencies or acceptance gaps. It is defined as:
    \begin{equation}
        \epsilon_i=\frac{N_{i}^{(\text{pass Reco} \cap \text{pass Truth})}}{N_{i}^{(\text{pass Truth})}}
    \end{equation}
    
    \item \textbf{Fiducial Fraction ($f_i$):} Also known as the purity of the selection, this factor corrects for events that are selected at the reconstruction level but originate from outside the fiducial phase space (background migrations). It is defined as:
    \begin{align}
        f_i=\frac{N_{i}^{(\text{pass Reco} \cap \text{pass Truth})}}{N_{i}^{(\text{pass Reco})}}
    \end{align}
\end{itemize}

\subsection{Unfolding Algorithm}

This analysis employs the **Iterative Bayesian Unfolding (IBU)** method \cite{dagostini2010improved}, implemented within the \texttt{RooUnfold} framework \cite{Adye:1349242}. 

The algorithm is based on Bayes' theorem. It begins with a prior probability distribution, initially assumed to be the particle-level distribution from the nominal MC simulation. In each iteration, the prior is updated based on the agreement between the measured data and the folded prediction. The algorithm takes as input the measured background-subtracted data, the migration matrix ($M$), and the efficiency/acceptance corrections described above.

The number of iterations serves as the regularization parameter. A low number of iterations may retain a bias towards the MC prior, while a high number of iterations reduces this bias but amplifies statistical fluctuations from the data. The optimization of this parameter is discussed in Section \ref{sec:unfolding_regularization}.

\subsection{Validation of the Unfolding Procedure}
\label{sec:unfolding_validation}

To verify the robustness of the unfolding procedure, a series of validation tests are performed. These tests check the mathematical validity of the algorithm (technical closure) and estimate the systematic uncertainty arising from model dependence (signal modelling bias).

\subsubsection{Technical Closure Tests}

The technical closure test verifies that the unfolding algorithm can correctly retrieve the true distribution when the detector response is perfectly known. In this test, the MC reconstruction-level distribution is treated as "pseudo-data" and unfolded using the response matrix derived from the same MC dataset.

Since the input distribution and the response matrix are consistent, the unfolded result must recover the MC truth-level distribution within numerical precision. This test has been performed for all differential variables. Three representative examples covering different kinematic regimes are presented below.

\paragraph{Angular Variables: $\Delta\Phi(l,l)$ (Inclusive)}
Figure \ref{fig:unfolding_technical_closure_dphill_inclusive} shows the validation for the azimuthal separation between leptons. The migration matrix (Fig. \ref{fig:unfolding_technical_closure_migration_dphill_inclusive}) shows strong diagonality. The unfolded result matches the truth perfectly, as confirmed by the ratio in Fig. \ref{fig:unfolding_technical_closure_distribution_dphill_inclusive} and the numerical values in Table \ref{tab:unfolding_technical_closure_dphill_inclusive}.

\begin{figure}[h!]
\centering
  \subfloat[\label{fig:unfolding_technical_closure_purity_eff_dphill_inclusive}]{\includegraphics[width=0.35\textwidth]{figures/llvv/unfolding/technical_closure/inclusive/dphill/all_dphill.png}}
  \subfloat[\label{fig:unfolding_technical_closure_migration_dphill_inclusive}]{\includegraphics[width=0.35\textwidth]{figures/llvv/unfolding/technical_closure/inclusive/dphill/migration_dphill.png}}
  \subfloat[\label{fig:unfolding_technical_closure_distribution_dphill_inclusive}]{\includegraphics[height=0.40\textwidth]{figures/llvv/unfolding/technical_closure/inclusive/dphill/final_dphill.png}}
  \caption{Technical closure test for $\Delta\Phi(l,l)$ in the Inclusive region: (a) purity and efficiency, (b) migration matrix, (c) unfolded cross section compared to truth.}
\label{fig:unfolding_technical_closure_dphill_inclusive}
\end{figure}

\begin{table}[h!]
\centering
\caption{Unfolded cross-sections (fb/rad) and uncertainties (\%) for the technical closure test of $\Delta\Phi(l,l)$ in the Inclusive region.}
\small
\begin{tabular}{c|ccccc}
\toprule
Bin (rad)       & 0 - 0.3   & 0.3 - 0.6  & 0.6 - 1    & 1 - 1.4    & 1.4 - 1.8 \\ 
\midrule
XS              & 7.092931  & 8.668584   & 11.756990  & 15.992146  & 10.534318 \\
\midrule
MET             & 0.46  & 0.80  & 0.42  & 0.35  & 1.57   \\
Jet             & 0.67  & 0.53  & 0.85  & 0.69  & 1.23   \\
Theory          & 0.93  & 0.32  & 1.12  & 0.42  & 1.27   \\
\midrule
Total           & 4.60  & 3.66  & 2.72  & 2.30  & 3.88   \\
\bottomrule
\end{tabular}
\label{tab:unfolding_technical_closure_dphill_inclusive}
\end{table}

\paragraph{Discrete Variables: Jet Multiplicity (Inclusive)}
Unfolding discrete variables can be challenging due to migration between adjacent bins. Figure \ref{fig:unfolding_technical_closure_n_jets_inclusive} demonstrates the performance for $N_{\text{jets}}$. Despite significant migration off-diagonal elements (Fig. \ref{fig:unfolding_technical_closure_migration_n_jets_inclusive}), the Bayesian unfolding correctly recovers the inclusive jet multiplicity spectrum.

\begin{figure}[h!]
\centering
  \subfloat[\label{fig:unfolding_technical_closure_purity_eff_n_jets_inclusive}]{\includegraphics[width=0.35\textwidth]{figures/llvv/unfolding/technical_closure/inclusive/n_jets/all_n_jets.png}}
  \subfloat[\label{fig:unfolding_technical_closure_migration_n_jets_inclusive}]{\includegraphics[width=0.35\textwidth]{figures/llvv/unfolding/technical_closure/inclusive/n_jets/migration_n_jets.png}}
  \subfloat[\label{fig:unfolding_technical_closure_distribution_n_jets_inclusive}]{\includegraphics[height=0.40\textwidth]{figures/llvv/unfolding/technical_closure/inclusive/n_jets/final_n_jets.png}}
  \caption{Technical closure test for $n_{\text{jets}}$ in the Inclusive region: (a) purity and efficiency, (b) migration matrix, (c) unfolded cross section compared to truth.}
\label{fig:unfolding_technical_closure_n_jets_inclusive}
\end{figure}

\begin{table}[h!]
\centering
\caption{Unfolded cross-sections (fb) and uncertainties (\%) for the technical closure test of $n_{\text{jets}}$ in the Inclusive region.}
\small
\begin{tabular}{c|cccc}
\toprule
Bin             & 0 - 1      & 1 - 2     & 2 - 3     & 3 - 11        \\
\midrule
XS              & 15.330063  & 3.808171  & 0.739250  & 0.020761 \\
\midrule
MET             & 0.39  & 1.46  & 1.26  & 1.57   \\
Jet             & 5.68  & 13.29 & 40.21 & 61.25  \\
Theory          & 0.90  & 2.06  & 7.15  & 5.09   \\
\midrule
Total           & 5.87  & 13.90 & 42.40 & 63.50  \\
\bottomrule
\end{tabular}
\label{tab:unfolding_technical_closure_n_jets_inclusive}
\end{table}

\FloatBarrier

\paragraph{VBS Variables: $m_{jj}$ (ZZjj Region)}
The dijet invariant mass $m_{jj}$ is the primary observable for Vector Boson Scattering. In the ZZjj region, resolution effects at high mass are significant. As shown in Figure \ref{fig:unfolding_technical_closure_mjj_ZZjj} and Table \ref{tab:unfolding_technical_closure_mjj_ZZjj}, the unfolding procedure successfully handles the wide bins at the tail of the distribution, ensuring an unbiased result.

\begin{figure}[h!]
\centering
  \subfloat[\label{fig:unfolding_technical_closure_purity_eff_mjj_ZZjj}]{\includegraphics[width=0.35\textwidth]{figures/llvv/unfolding/technical_closure/ZZjj/mjj/all_mjj.png}}
  \subfloat[\label{fig:unfolding_technical_closure_migration_mjj_ZZjj}]{\includegraphics[width=0.35\textwidth]{figures/llvv/unfolding/technical_closure/ZZjj/mjj/migration_mjj.png}}
  \subfloat[\label{fig:unfolding_technical_closure_distribution_mjj_ZZjj}]{\includegraphics[height=0.40\textwidth]{figures/llvv/unfolding/technical_closure/ZZjj/mjj/final_mjj.png}}
  \caption{Technical closure test for $m_{jj}$ in the ZZjj region: (a) purity and efficiency, (b) migration matrix, (c) unfolded cross section compared to truth.}
\label{fig:unfolding_technical_closure_mjj_ZZjj}
\end{figure}

\begin{table}[h!]
\centering
\caption{Unfolded cross-sections (fb/GeV) and uncertainties (\%) for the technical closure test of $m_{jj}$ in the ZZjj region.}
\small
\begin{tabular}{c|ccc}
\toprule
Bin (GeV)       & 0 - 400   & 400 - 800  & 800 - 8000      \\
\midrule
XS              & 0.002052  & 0.000202   & 0.000005       \\
\midrule
MET             & 0.13  & 1.62  & 3.22   \\
Jet             & 3.86  & 24.11 & 36.54  \\
Theory          & 0.92  & 7.13  & 5.27   \\
\midrule
Total           & 4.36  & 29.20 & 42.10  \\
\bottomrule
\end{tabular}
\label{tab:unfolding_technical_closure_mjj_ZZjj}
\end{table}

\FloatBarrier

\subsubsection{Signal Modelling and Unfolding Bias}
\label{sec:unfolding_bias_results}

The unfolding result carries an intrinsic dependence on the physics model (prior) used to generate the migration matrix. To estimate the potential bias introduced by this assumption, a stress test is performed using an alternative signal model.

The migration matrix is re-weighted using an alternative Monte Carlo sample (MadGraph qqZZ) to mimic a different underlying physics distribution. The nominal reconstruction-level distribution is then unfolded using this alternative matrix. The deviation of the unfolded result from the nominal truth is taken as the **unfolding bias**.

Figures \ref{fig:unfolding_modelling_bias_inclusive_1} and \ref{fig:unfolding_modelling_bias_ZZjj_1} summarize the bias tests for the Inclusive and ZZjj regions, respectively. In most bins, the bias is found to be negligible. However, to ensure a conservative result, any bias exceeding 1\% is symmetrized and applied as a systematic uncertainty on the final measurement.

\begin{figure}[h!]
\centering
  \subfloat[\label{fig:unfolding_modelling_bias_dphill_inclusive}]{\includegraphics[width=0.49\textwidth]{figures/llvv/unfolding/bias/inclusive/final_dphill.png}}
  \subfloat[\label{fig:unfolding_modelling_bias_leading_pT_lepton_inclusive}]{\includegraphics[width=0.49\textwidth]{figures/llvv/unfolding/bias/inclusive/final_leading_pT_lepton.png}} \\
  \subfloat[\label{fig:unfolding_modelling_bias_mt_zz_inclusive}]{\includegraphics[width=0.49\textwidth]{figures/llvv/unfolding/bias/inclusive/final_mt_zz.png}}
  \subfloat[\label{fig:unfolding_modelling_bias_n_jets_inclusive}]{\includegraphics[width=0.49\textwidth]{figures/llvv/unfolding/bias/inclusive/final_n_jets.png}}
  \caption{Unfolding closure test using reweighted migration matrix from alternative samples for (a) $\Delta\Phi(l,l)$, (b) $p_{T}^{\text{leading lepton}}$, (c) $m_{T}^{ZZ}$, and (d) jet multiplicity in the Inclusive region.}
\label{fig:unfolding_modelling_bias_inclusive_1}
\end{figure}

\begin{figure}[h!]
\centering
  \subfloat[\label{fig:unfolding_modelling_bias_dphill_ZZjj}]{\includegraphics[width=0.49\textwidth]{figures/llvv/unfolding/bias/ZZjj/final_dphill.png}}
  \subfloat[\label{fig:unfolding_modelling_bias_leading_jet_pt_ZZjj}]{\includegraphics[width=0.49\textwidth]{figures/llvv/unfolding/bias/ZZjj/final_leading_jet_pt.png}} \\
  \subfloat[\label{fig:unfolding_modelling_bias_mjj_ZZjj}]{\includegraphics[width=0.49\textwidth]{figures/llvv/unfolding/bias/ZZjj/final_mjj.png}}
  \subfloat[\label{fig:unfolding_modelling_bias_mt_zz_ZZjj}]{\includegraphics[width=0.49\textwidth]{figures/llvv/unfolding/bias/ZZjj/final_mt_zz.png}}
  \caption{Unfolding closure test using reweighted migration matrix from alternative samples for (a) $\Delta\Phi(l,l)$, (b) $p_{T}^{\text{leading jet}}$, (c) $m_{jj}$, and (d) $m_{T}^{ZZ}$ in the ZZjj region.}
\label{fig:unfolding_modelling_bias_ZZjj_1}
\end{figure}

\FloatBarrier

\subsection{Regularization and Iteration Optimization}
\label{sec:unfolding_regularization}

The number of iterations in the Bayesian unfolding acts as a regularization parameter. Increasing the number of iterations reduces the dependence on the MC prior (bias) but increases the statistical uncertainty propagated from the data. 

To determine the optimal number of iterations for each variable, the total uncertainty (stat $\oplus$ bias) was minimized. Based on the bias studies presented in Section \ref{sec:unfolding_bias_results}, an iteration number between 2 and 5 was found to be optimal for most variables. The specific iteration numbers selected for the final results are listed in Table \ref{tab:unfolding_technical_closure_n_itr}.

\begin{table}[htbp]
    \centering
    \caption{Selected iteration number for each variable in the Bayesian unfolding algorithm.}
    \begin{tabular}{l|cc} \toprule
    \textbf{Variable} & \textbf{Inclusive} & \textbf{ZZjj} \\ \midrule
        $\Delta\Phi(l,l)$             & 3         & 2 \\
        $p_{T}^{\text{leading lepton}}$ & 2         & - \\
        $p_{T}^{\text{leading jet}}$    & -         & 2 \\
        $m_{T}^{ZZ}$                  & 3         & 5 \\
        $m_{jj}$                      & -         & 4 \\
        Jet multiplicity              & 5         & - \\
        $p_{T}^{ZZ}$                  & 5         & 4 \\
        $p_{T}^{Z}$                   & 2         & 3 \\
        $y(Z)$                        & 2         & 3 \\
        $sin(\phi)cos(\theta)$        & 2         & - \\ \bottomrule
    \end{tabular}
    \label{tab:unfolding_technical_closure_n_itr}
\end{table}

\FloatBarrier


\clearpage
%\section{DNN Optimization}
\section{DNN Optimization}
\label{sec:dnn}

This chapter focus on cross-section investigation using the DNN approach. 


\clearpage
%\section{Results}
\section{Results}
\label{sec:results_llvv}

This section presents the results of the analysis performed on the full Run 2 dataset, corresponding to an integrated luminosity of $140~\text{fb}^{-1}$. The measurements include the extraction of signal strengths via a profile-likelihood fit, the determination of fiducial and total integrated cross-sections, and the measurement of differential cross-sections unfolded to particle level. Finally, limits are placed on anomalous neutral triple gauge couplings (aTGCs) and anomalous quartic gauge couplings (aQGCs) within the Effective Field Theory (EFT) framework.

\subsection{Signal Extraction and Fit Results}
\label{subsec:signal_extraction}

The signal yields are extracted using a binned profile-likelihood fit to the discriminative observable $\Delta\phi(\ell\ell, E_{\mathrm{T}}^{\text{miss}})$. This fit is performed simultaneously across the Signal Regions (SRs) and the dedicated Control Regions (CRs) described in Section~\ref{sec:event_selection}. 

For the inclusive $ZZ \to \ell\ell\nu\nu$ production, the observed signal strength parameter, defined as the ratio of the observed signal yield to the Standard Model (SM) prediction ($\mu = \sigma_{\text{obs}}/\sigma_{\text{exp}}$), is measured to be:
\begin{equation}
    \mu_{ZZ} = 1.05 \pm 0.05,
\end{equation}
indicating excellent agreement with the SM expectation. The dominant background normalization factors constrained by the fit are found to be close to unity, with $\mu_{WZ} = 1.00^{+0.06}_{-0.05}$ and $\mu_{\text{top}} = 0.98^{+0.13}_{-0.10}$. The normalization for the $Z+\text{jets}$ background varies by jet multiplicity, ranging from $1.14$ in the 0-jet region to $1.10$ in the $\geq 2$-jet region.

In the electroweak-enriched $ZZjj$ channel, the fitted signal strength is:
\begin{equation}
    \mu_{ZZjj} = 1.02^{+0.19}_{-0.17}.
\end{equation}
The background normalization factors for this channel are determined to be $0.97^{+0.24}_{-0.20}$ for the $WZ$ background and $1.08^{+0.19}_{-0.15}$ for the non-resonant backgrounds ($WW$ and top).

Post-fit distributions of $\Delta\phi(\ell\ell, E_{\mathrm{T}}^{\text{miss}})$ in the signal regions are shown in Figure~\ref{fig:postfit_control}. The data is well-described by the post-fit predictions within the assigned uncertainties. Representative kinematic distributions, such as the transverse mass of the $ZZ$ system ($m_{\mathrm{T}}^{ZZ}$) and the transverse momentum of the leading lepton, are shown in Figure~\ref{fig:postfit_kinematics}.

% Placeholder for Figure 3 from the paper
\begin{figure}[htbp]
    \centering
    \includegraphics[width=0.45\textwidth]{figures/postfit_dphi_ZZ.pdf}
    \includegraphics[width=0.45\textwidth]{figures/postfit_dphi_ZZjj.pdf}
    \caption{The post-fit distributions of $\Delta\phi(\vec{E}_{\mathrm{T}}^{\text{miss}}, Z)$ in the inclusive $ZZ$ (left) and $ZZjj$ (right) signal regions. The bottom panels show the ratio of the data to the total prediction.}
    \label{fig:postfit_control}
\end{figure}

\subsection{Integrated Cross-Section Measurements}
\label{subsec:integrated_xsec}

The measured signal strengths are converted into fiducial cross-sections using the SM predictions from \textsc{Sherpa} and \textsc{MadGraph5}. The results for both the inclusive and VBS-enriched phase spaces are summarized below.

\subsubsection{Inclusive $ZZ \to \ell\ell\nu\nu$}

The measured fiducial cross-section for the inclusive $ZZ \to \ell\ell\nu\nu$ process is:
\begin{equation}
    \sigma_{ZZ \to \ell\ell\nu\nu}^{\text{fid}} = 21.0 \pm 1.0~\text{fb}.
\end{equation}
The total uncertainty is composed of $\pm 0.73$ (stat), $\pm 0.57$ (exp), $\pm 0.18$ (theory), and $\pm 0.18$ (lumi) fb. This result represents a significant improvement in precision compared to previous ATLAS measurements using partial Run 2 data, with the total uncertainty reduced from 7.0\% to approximately 4.8\%. 

The measurement is consistent with the state-of-the-art SM prediction calculated using \textsc{Matrix} at NNLO in QCD and NLO in EW corrections, which yields $21.03 \pm 0.30$ fb.

Using the fiducial acceptance factor $A_{ZZ}$ and the branching ratio for $ZZ \to 2\ell 2\nu$, the result is extrapolated to the full phase space for $Z$ boson masses between 66 and 116 GeV. The total $pp \to ZZ$ cross-section is determined to be:
\begin{equation}
    \sigma_{pp \to ZZ}^{\text{total}} = 15.38 \pm 0.81~\text{pb},
\end{equation}
which agrees well with the \textsc{Matrix} prediction of $15.40 \pm 0.38$ pb.

\subsubsection{Electroweak $ZZjj \to \ell\ell\nu\nu jj$}

The fiducial cross-section for the $ZZjj$ production, which requires at least two jets in the final state, is measured to be:
\begin{equation}
    \sigma_{ZZjj \to \ell\ell\nu\nu jj}^{\text{fid}} = 0.96^{+0.18}_{-0.16}~\text{fb}.
\end{equation}
This value is compatible with the SM prediction of $0.94 \pm 0.20$ fb derived from \textsc{Sherpa} (QCD component) and \textsc{MadGraph5} (EW component). The uncertainty is dominated by statistical limitations due to the rarity of the process.

\subsection{Differential Cross-Sections}
\label{subsec:diff_xsec}

Differential cross-sections are measured to probe the kinematic properties of the $ZZ$ production and test perturbative QCD predictions. The detector-level distributions are unfolded to the stable particle level using an iterative Bayesian method to correct for detector resolution and acceptance effects.

For the inclusive $ZZ$ case, eight kinematic variables were measured. Figure~\ref{fig:diff_ZZ} (left) shows the differential cross-section as a function of the transverse momentum of the $Z$ boson ($p_{\mathrm{T}}^Z$). The data shows good agreement with both the \textsc{Sherpa} and \textsc{Matrix} predictions, though the \textsc{Matrix} calculation (incorporating NNLO QCD) tends to describe the normalization slightly better.

For the $ZZjj$ case, seven variables were examined. Figure~\ref{fig:diff_ZZ} (right) displays the transverse mass of the $ZZ$ system ($m_{\mathrm{T}}^{ZZ}$). Despite the larger statistical uncertainties inherent to the $ZZjj$ channel, the shape of the distribution is well-modeled by the SM simulation.

% Placeholder for differential plots (Fig 5 and 6 in paper)
\begin{figure}[htbp]
    \centering
    \includegraphics[width=0.45\textwidth]{figures/diff_pTZ_ZZ.pdf}
    \includegraphics[width=0.45\textwidth]{figures/diff_mTZZ_ZZjj.pdf}
    \caption{Measured differential cross-sections for inclusive $ZZ$ production as a function of $p_{\mathrm{T}}^Z$ (left) and for $ZZjj$ production as a function of $m_{\mathrm{T}}^{ZZ}$ (right). The data are compared to SM predictions from \textsc{Sherpa} and \textsc{Matrix}.}
    \label{fig:diff_ZZ}
\end{figure}

\subsection{Constraints on Anomalous Couplings}
\label{subsec:anom_couplings}

The measured kinematic distributions are used to search for physics Beyond the Standard Model (BSM) manifesting as anomalous gauge boson self-interactions. As no significant deviations from the SM predictions are observed, limits are set on the coupling parameters.

\subsubsection{Anomalous Neutral Triple Gauge Couplings (aTGCs)}

The inclusive $ZZ$ production is sensitive to anomalous $ZZZ$ and $ZZ\gamma$ couplings, which are forbidden at tree level in the SM. Constraints are derived by fitting the reconstructed $p_{\mathrm{T}}$ distribution of the dilepton system. 

Limits are set using two frameworks: the vertex function (VF) formalism and the SM Effective Field Theory (SMEFT). The 95\% confidence level (CL) intervals for the VF parameters $f_4^{\gamma/Z}$ (CP-violating) and $f_5^{\gamma/Z}$ (CP-conserving) are presented in Table~\ref{tab:atgc_limits}. These results improve upon the previous ATLAS $ZZ \to \ell\ell\nu\nu$ measurement by a factor of approximately 1.7.

\begin{table}[htbp]
    \centering
    \caption{Observed and expected 95\% CL intervals for the neutral aTGC parameters in the vertex function formalism.}
    \label{tab:atgc_limits}
    \begin{tabular}{lcc}
        \hline
        Parameter & Expected ($10^{-4}$) & Observed ($10^{-4}$) \\
        \hline
        $f_4^{\gamma}$ & $[-7.7, 7.7]$ & $[-6.4, 6.4]$ \\
        $f_4^{Z}$ & $[-6.8, 6.8]$ & $[-5.7, 5.7]$ \\
        $f_5^{\gamma}$ & $[-7.8, 7.7]$ & $[-6.4, 6.5]$ \\
        $f_5^{Z}$ & $[-6.9, 6.9]$ & $[-5.8, 5.8]$ \\
        \hline
    \end{tabular}
\end{table}

\subsubsection{Anomalous Quartic Gauge Couplings (aQGCs)}

The $ZZjj$ channel provides a unique sensitivity to the $ZZWW$ and $ZZZZ$ quartic vertices via Vector Boson Scattering (VBS) topologies. Limits are placed on dimension-8 SMEFT operators ($T0$ through $T9$) by fitting the transverse mass of the $ZZ$ system ($m_{\mathrm{T}}^{ZZ}$).

The observed limits on the Wilson coefficients $f_{T,i}/\Lambda^4$ are consistent with zero. To ensuring the validity of the EFT expansion, unitarity bounds are calculated as a function of the energy cut-off scale $E_c$. The constraints obtained in this analysis are significantly tighter than those from the $ZZ \to 4\ell jj$ channel, improving sensitivity by more than a factor of two, driven primarily by the larger branching ratio of the $\ell\ell\nu\nu$ final state.

The evolution of the limits for the $f_{T0}$ operator as a function of the cutoff scale is visualized in Figure~\ref{fig:aqgc_limits}.

% Placeholder for Fig 11a
\begin{figure}[htbp]
    \centering
    \includegraphics[width=0.6\textwidth]{figures/limit_fT0.pdf}
    \caption{Expected and observed 95\% CL intervals for the Wilson coefficient $f_{T0}/\Lambda^4$ as a function of the EFT cut-off scale $E_c$. The solid green curve indicates the unitarity bound.}
    \label{fig:aqgc_limits}
\end{figure}

\subsection{Summary of Results}

In summary, the fiducial and differential cross-sections for $ZZ \to \ell\ell\nu\nu$ and $ZZjj \to \ell\ell\nu\nu jj$ have been measured with high precision using the full ATLAS Run 2 dataset. The inclusive fiducial cross-section is measured with a precision of 4.8\%, limited by statistics. All measurements show excellent agreement with Standard Model predictions. Stringent limits have been placed on anomalous triple and quartic gauge couplings, significantly constraining the phase space for new physics in the electroweak sector.


\clearpage
%\section{llvvjj}
\section{Electroweak $ZZjj$ Production and Search for aQGCs}
\label{sec:llvvjj}

While the inclusive $ZZ$ measurement provides a precision test of QCD and trilinear gauge couplings, the study of $ZZ$ production in association with two jets ($ZZjj$) offers a unique window into the electroweak sector. This final state is sensitive to Vector Boson Scattering (VBS), a process directly probing the quartic gauge boson self-interactions ($VVVV$) predicted by the Standard Model. This section presents the measurement of the $ZZjj \to \ell\ell\nu\nu jj$ cross-section and the subsequent search for Anomalous Quartic Gauge Couplings (aQGCs).

\subsection{Event Selection and Optimization}
\label{subsec:zzjj_selection}

To target the VBS topology, the event selection is modified from the inclusive $\ell\ell\nu\nu$ baseline described in Section~\ref{sec:llvv_event_selection}. The selection is optimized to suppress the dominant QCD-induced $ZZ$ background and enhance the electroweak signal-to-background ratio.

The definitions for leptons and the $Z$ boson candidate remain identical to the inclusive analysis. However, specific requirements are imposed on the hadronic activity and missing transverse momentum:

\begin{itemize}
    \item \textbf{Jet Multiplicity:} Events must contain at least two hadronic jets with $p_{\mathrm{T}} > 30$~GeV and $|\eta| < 4.5$. This ensures the presence of the "tagging jets" characteristic of VBS processes.
    \item \textbf{Missing Transverse Momentum:} The threshold is raised significantly to $E_{\mathrm{T}}^{\text{miss}} > 150$~GeV (compared to $110$~GeV in the inclusive channel). This stricter cut is essential to suppress the $Z+\text{jets}$ background, which is more abundant in high jet-multiplicity regions.
\end{itemize}

Standard VBS analyses often apply cuts on the dijet invariant mass ($m_{jj}$) or pseudorapidity separation ($\Delta y_{jj}$). However, in this fiducial measurement, these kinematic variables are inspected differentially rather than being used as cut thresholds, allowing for a broader phase space measurement.

\subsection{Fiducial Cross-Section Measurement}
\label{subsec:zzjj_xsec}

The signal yield is extracted using a profile-likelihood fit to the $\Delta\phi(\ell\ell, E_{\mathrm{T}}^{\text{miss}})$ distribution, similar to the inclusive strategy. The fit includes dedicated control regions for the $WZjj$ and non-resonant backgrounds.

The observed signal strength for the $ZZjj$ process is:
\begin{equation}
    \mu_{ZZjj} = 1.02^{+0.19}_{-0.17}.
\end{equation}
The background normalization factors are determined to be $0.97^{+0.24}_{-0.20}$ for $WZ$ and $1.08^{+0.19}_{-0.15}$ for the non-resonant $WW/t\bar{t}$ component.

The corresponding measured fiducial cross-section is:
\begin{equation}
    \sigma_{ZZjj \to \ell\ell\nu\nu jj}^{\text{fid}} = 0.96^{+0.18}_{-0.16}~\text{fb}.
\end{equation}
This result is in good agreement with the Standard Model prediction of $0.94 \pm 0.20$~fb, which combines the QCD-induced production (simulated by \textsc{Sherpa}) and the electroweak-induced production (simulated by \textsc{MadGraph5}). The uncertainty is dominated by data statistics due to the rarity of the process.

Differential cross-sections were also measured for seven kinematic observables, including the dijet invariant mass $m_{jj}$ and the transverse mass of the $ZZ$ system ($m_{\mathrm{T}}^{ZZ}$). As shown in Figure~\ref{fig:postfit_observed_zzjj} and Figure~\ref{fig:diff_xs_ZZjj}, the data is well described by the SM predictions within the larger uncertainties of this channel.

\begin{figure}[!htbp]
    \centering
    \subfloat[]{
        \includegraphics[width=0.32\linewidth]{figures/llvv/fig_postfit_d.pdf}
        \label{fig:postfit_leading_jet_pt}
    }
    \hfill
    \subfloat[]{
        \includegraphics[width=0.32\linewidth]{figures/llvv/fig_postfit_e.pdf}
        \label{fig:postfit_mjj}
    }
    \hfill
    \subfloat[]{
        \includegraphics[width=0.32\linewidth]{figures/llvv/fig_postfit_f.pdf}
        \label{fig:postfit_mtzz}
    }
    \caption{Post-fit kinematic distributions for the $ZZjj$ SR: 
    (a) the leading jet $p_\text{T}^\text{leading jet}$, 
    (b) the invariant mass of the dijet system $m_\text{jj}$, and 
    (c) the transverse mass of the $ZZ$ system $m_\text{T}^\text{ZZ}$. 
    The data points are shown with statistical error bars, while the shaded band represents the total post-fit uncertainty in the prediction, combining statistical and systematic uncertainties. Open markers indicate data points lying outside the vertical range of the plot.}
    \label{fig:postfit_observed_zzjj}
\end{figure}

\begin{figure}[!hbtp]
    \centering
    % Row 1: 3 images
    \subfloat[]{
        \includegraphics[width=0.32\linewidth]{figures/llvv/fig_diff_llvvjj_a.pdf}
        \label{fig:diff_llvvjj_z_pt}
    }
    \hfill
    \subfloat[]{
        \includegraphics[width=0.32\linewidth]{figures/llvv/fig_diff_llvvjj_b.pdf}
        \label{fig:diff_llvvjj_dphi}
    }
    \hfill
    \subfloat[]{
        \includegraphics[width=0.32\linewidth]{figures/llvv/fig_diff_llvvjj_c.pdf}
        \label{fig:diff_llvvjj_y_z}
    }
    \\ % Row 2: 2 images (slightly wider for visibility)
    \subfloat[]{
        \includegraphics[width=0.45\linewidth]{figures/llvv/fig_diff_llvvjj_d.pdf}
        \label{fig:diff_llvvjj_pt_zz}
    }
    \hfill
    \subfloat[]{
        \includegraphics[width=0.45\linewidth]{figures/llvv/fig_diff_llvvjj_e.pdf}
        \label{fig:diff_llvvjj_mt_zz}
    }
    \\ % Row 3: 2 images
    \subfloat[]{
        \includegraphics[width=0.45\linewidth]{figures/llvv/fig_diff_llvvjj_f.pdf}
        \label{fig:diff_llvvjj_jet_pt}
    }
    \hfill
    \subfloat[]{
        \includegraphics[width=0.45\linewidth]{figures/llvv/fig_diff_llvvjj_g.pdf}
        \label{fig:diff_llvvjj_mjj}
    }

    \caption{Differential cross-section measurements in the $ZZjj \rightarrow \ell\ell\nu\nu jj$ fiducial phase space, compared with SM predictions for a set of kinematic variables:
    (a) $p_\text{T}^\text{Z}$,
    (b) $\Delta\phi(\ell, \ell)$,
    (c) $|y^Z|$,
    (d) $p_\text{T}^\text{ZZ}$,
    (e) $m_\text{T}^\text{ZZ}$,
    (f) $p_\text{T}^\text{leading jet}$, and 
    (g) $m_\text{jj}$.
    The SM predictions are obtained using \textsc{Sherpa} for strong production and \textsc{MadGraph5} for electroweak production. The statistical and total uncertainties in the data points are displayed in both the differential cross-sections and the corresponding ratios. The total theory uncertainties in the MC predictions are represented as shaded bands in the cross-section panels and as error bars in the ratio panels.}
    \label{fig:diff_xs_ZZjj}
\end{figure}

\subsection{Constraints on Anomalous Quartic Gauge Couplings}
\label{subsec:aqgc_limits}

The $ZZjj$ channel provides direct sensitivity to the $ZZWW$ and $ZZZZ$ quartic vertices. In the Standard Model, the cross-section contribution from these vertices is small. However, physics Beyond the Standard Model (BSM) could manifest as Anomalous Quartic Gauge Couplings (aQGCs), leading to a significant enhancement of the cross-section at high energy scales.

\subsubsection{EFT Interpretation}

The aQGCs are parameterized using the Standard Model Effective Field Theory (SMEFT) framework. This analysis considers dimension-8 operators, specifically the transversal operators $T0$ through $T9$, which induce quartic couplings without modifying triple gauge couplings.

The most sensitive observable for detecting aQGCs is the transverse mass of the $ZZ$ system, defined as:
\begin{equation}
    m_{\mathrm{T}}^{ZZ} = \sqrt{ \left( \sqrt{m_Z^2 + (p_{\mathrm{T}}^{\ell\ell})^2} + \sqrt{m_Z^2 + (E_{\mathrm{T}}^{\text{miss}})^2} \right)^2 - \left| \vec{p}_{\mathrm{T}}^{\ell\ell} + \vec{E}_{\mathrm{T}}^{\text{miss}} \right|^2 }.
\end{equation}
The presence of aQGCs would appear as an excess of events in the high-$m_{\mathrm{T}}^{ZZ}$ tail.

\subsubsection{Limits on Wilson Coefficients}

A binned likelihood fit is performed on the $m_{\mathrm{T}}^{ZZ}$ distribution in the $ZZjj$ signal region. No significant deviation from the SM background expectation is observed. Consequently, 95\% confidence level (CL) limits are set on the Wilson coefficients $f_{T,i}/\Lambda^4$.

The observed limits are consistent with zero for all operators. To ensure the validity of the EFT expansion, unitarity bounds are calculated as a function of the energy cut-off scale $E_c$. The evolution of the limits for the $f_{T0}$ - $f_{T3}$ operator is shown in Figure~\ref{fig:aqgc_limits_fT0_fT3}.

Notably, the constraints obtained in this $\ell\ell\nu\nu jj$ analysis are significantly tighter than those from the fully leptonic $ZZ \to 4\ell jj$ channel, improving the sensitivity by more than a factor of two. This improvement is driven primarily by the larger branching ratio of the $Z \to \nu\nu$ decay compared to $Z \to \ell\ell$, which preserves statistics in the high-$m_{\mathrm{T}}^{ZZ}$ tail where the aQGC sensitivity is highest.

\begin{figure}[htbp]
    \centering
    % Top Row: fT0 and fT1
    \subfloat[]{
        \includegraphics[width=0.45\linewidth]{figures/llvv/fig_aqgc_limit_fT0.pdf}
        \label{fig:aqgc_fT0}
    }
    \hfill
    \subfloat[]{
        \includegraphics[width=0.45\linewidth]{figures/llvv/fig_aqgc_limit_fT1.pdf}
        \label{fig:aqgc_fT1}
    }
    \\ % Line break for Bottom Row
    % Bottom Row: fT2 and fT3
    \subfloat[]{
        \includegraphics[width=0.45\linewidth]{figures/llvv/fig_aqgc_limit_fT2.pdf}
        \label{fig:aqgc_fT2}
    }
    \hfill
    \subfloat[]{
        \includegraphics[width=0.45\linewidth]{figures/llvv/fig_aqgc_limit_fT3.pdf}
        \label{fig:aqgc_fT3}
    }
    
    \caption{Expected and observed 95\% CL intervals for the Wilson coefficients (a) $f_{\text{T0}}$, (b) $f_{\text{T1}}$, (c) $f_{\text{T2}}$, and (d) $f_{\text{T3}}$ as a function of the EFT cut-off scale $E_\text{c}$. 
    The cut-off scale imposes a restriction such that the BSM amplitudes are set to zero when $m_{ZZ} > E_\text{c}$. 
    The dashed (solid) lines show the expected (observed) limits, and the solid green curve corresponds to the unitarity bounds.}
    \label{fig:aqgc_limits_fT0_fT3}
\end{figure}


% \begin{figure}[htbp]
%     \centering
%     % Row 1
%     \subfloat[]{
%         \includegraphics[width=0.32\textwidth]{figures/llvv/fig_aqgc_limit_fT4.pdf}
%         \label{fig:aqgc_fT4}
%     }
%     \hfill
%     \subfloat[]{
%         \includegraphics[width=0.32\textwidth]{figures/llvv/fig_aqgc_limit_fT5.pdf}
%         \label{fig:aqgc_fT5}
%     }
%     \hfill
%     \subfloat[]{
%         \includegraphics[width=0.32\textwidth]{figures/llvv/fig_aqgc_limit_fT6.pdf}
%         \label{fig:aqgc_fT6}
%     }
%     \\ % Line break
%     % Row 2
%     \subfloat[]{
%         \includegraphics[width=0.32\textwidth]{figures/llvv/fig_aqgc_limit_fT7.pdf}
%         \label{fig:aqgc_fT7}
%     }
%     \hfill
%     \subfloat[]{
%         \includegraphics[width=0.32\textwidth]{figures/llvv/fig_aqgc_limit_fT8.pdf}
%         \label{fig:aqgc_fT8}
%     }
%     \hfill
%     \subfloat[]{
%         \includegraphics[width=0.32\textwidth]{figures/llvv/fig_aqgc_limit_fT9.pdf}
%         \label{fig:aqgc_fT9}
%     }
%     \caption{Expected and observed 95\% CL intervals for the Wilson coefficients $f_{\text{T4}}$ through $f_{\text{T9}}$ as a function of the EFT cut-off scale $E_\text{c}$. The dashed (solid) lines show the expected (observed) limits, and the solid green curve corresponds to the unitarity bounds.}
%     \label{fig:aqgc_limits_fT4_fT9}
% \end{figure}