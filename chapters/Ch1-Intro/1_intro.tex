% !TeX root = ../main.tex

%\chapter{Introduction}

%-------------ATLAS----------------
\section{Particle Physics}

Particle physics is a subject that focuses on the fundamental properties of elementary particles. 

Back in the 1930s, the weak interaction was first postulated in the 1930s to explain beta decay, but the mediate particle remains hidden. After the discovery of CP violation in the WU experiment and the formulation of Yang-Mills theory in 1950s, during 1964-1971, Sheldon Glashow, Abdus Salam, and Steven Weinberg formulated the electroweak theory, which predicted the Z boson as the mediator of neutral current weak interactions. However, the mass of this unknown heavy mediator particle has remained unknown for more than 10 years. In 1983, the Super Proton Synchrotron(SPS) discovered the Z boson in both UA1 and UA2 experiments. This is the first significant verification of the Standard Model(SM), and it keeps an important component of SM.

The Z boson remains the important rule in the Standard Model(SM), and the properties of Z boson are important parameters in the electroweak theory. Measurement of the mass, width, and other properties of the Z boson can provide insights into the electroweak symmetry breaking mechanism. 

The precise measurements of the Z boson's decay widths also allow for stringent tests of the SM and constraints on new physics beyond the SM(BSM), which is one of the most compelling challenges in modern particle physics. Since the Standard Model does not account for the presence of DM, which constitutes approximately 27\% of the universe's mass-energy content, it hint the theories beyond the Standard Model. Beyond Standard Model (BSM) theories propose various candidates and mechanisms to explain the existence of the dark matter. 

Several BSM theories propose particles that could serve as DM candidates. These theories often extend the SM by introducing new particles and interactions. However, the dark matter particles are expected to be neutral and weakly interacting, meaning they won’t leave direct signals in the detectors. Therefore, the detection of dark matter relies on observing its decay products or the invisible component resulting from visible parent processes. Current dark matter detection methods at the Large Hadron Collider (LHC) are based on precise measurements of Missing Transverse Energy (MET) and constraints from existing process decay ratios.

Both methods require the high-precision measurement of the existing processes in the SM, as many processes from the expected SM behavior could indicate the presence of dark matter. This precision is crucial for accurately identifying the subtle signals of dark matter among the complex background of SM processes.



\section{Analyses}

\subsection{Measurement of ZZ Production}

The differential cross-section measurement of the process $ZZ \rightarrow \ell^+\ell^- \nu\bar{\nu}$ is a crucial component in the study of electroweak interactions and the Higgs mechanism. This process can provide a stringent test of the SM predictions, including the leptonic decay properties of Z boson. Since there are only leptons in the final state, this channel has a clear and distinctive signature that can be precisely measured on the ATLAS detector. 













\subsection{\texorpdfstring{Search for Dark Matter with $ll+E^{miss}_{T}$ Final State}{Search for Dark Matter with ll+MET Final State}}

The \( ZZ \rightarrow \ell^+\ell^- \nu\bar{\nu} \) decay channel plays an important rule in the search for DM with the following several reasons:

\begin{enumerate}
    \item \textbf{Clean Signature}: The final state with two charged leptons and missing energy provides a clean and relatively background-free signature. The charged leptons can be precisely measured, and the missing energy can be attributed to neutrinos and potential DM particles.

    \item \textbf{Sensitivity to New Physics}: The presence of missing energy makes this channel highly sensitive to new physics scenarios, including those predicting DM production. Deviations from the SM predictions in the measured differential cross-sections can signal the existence of BSM particles.

    \item \textbf{Complementary to Other Searches}: While direct detection experiments aim to observe DM interactions with ordinary matter, collider searches like those involving the \( ZZ \rightarrow \ell^+\ell^- \nu\bar{\nu} \) channel provide complementary information by attempting to produce DM particles directly. This multi-faceted approach increases the likelihood of discovering DM.
\end{enumerate}

The study of the \( ZZ \rightarrow \ell^+\ell^- \nu\bar{\nu} \) decay channel is vital for probing BSM theories and searching for DM. By leveraging the clean signature and sensitivity to missing energy, physicists can test various DM candidates and models. This channel, combined with other experimental searches, enhances the potential to uncover the nature of dark matter and extend our understanding beyond the Standard Model.





\section{Hardware Work}
