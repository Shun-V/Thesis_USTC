\section{Results}
\label{sec:llvv_results}

This section presents the results of the inclusive $ZZ \to \ell\ell\nu\nu$ analysis performed on the full Run 2 dataset, corresponding to an integrated luminosity of $140~\text{fb}^{-1}$. The measurements include the extraction of signal strengths via a profile-likelihood fit, the determination of fiducial and total integrated cross-sections, and the measurement of differential cross-sections unfolded to particle level. Finally, limits are placed on anomalous neutral triple gauge couplings (aTGCs) within the Effective Field Theory (EFT) framework.

\subsection{Signal Extraction and Fit Results}
\label{subsec:signal_extraction}

The signal yield is extracted using a binned profile-likelihood fit to the discriminative observable $\Delta\phi(\ell\ell, E_{\mathrm{T}}^{\text{miss}})$. This fit is performed simultaneously across the Signal Region (SR) and the dedicated Control Regions (CRs) described in Section~\ref{sec:llvv_event_selection}. 

The observed signal strength parameter, defined as the ratio of the observed signal yield to the Standard Model (SM) prediction ($\mu = \sigma_{\text{obs}}/\sigma_{\text{exp}}$), is measured to be:
\begin{equation}
    \mu_{ZZ} = 1.05 \pm 0.05,
\end{equation}
indicating excellent agreement with the SM expectation. The dominant background normalization factors constrained by the fit are found to be close to unity, with $\mu_{WZ} = 1.00^{+0.06}_{-0.05}$ and $\mu_{\text{top}} = 0.98^{+0.13}_{-0.10}$. The normalization for the $Z+\text{jets}$ background varies by jet multiplicity, ranging from $1.14$ in the 0-jet region to $1.10$ in the $\geq 2$-jet region.

The post-fit distribution of $\Delta\phi(\ell\ell, E_{\mathrm{T}}^{\text{miss}})$ in the signal region is shown in Figure~\ref{fig:postfit_observed}. The data is well-described by the post-fit predictions within the assigned uncertainties. Representative kinematic distributions, such as the transverse mass of the $ZZ$ system ($m_{\mathrm{T}}^{ZZ}$) and the transverse momentum of the leading lepton, are shown in Figure~\ref{fig:postfit_observed}.

\begin{figure}[!htbp]
    \centering
    \subfloat[]{
        \includegraphics[width=0.32\linewidth]{figures/llvv/fig_postfit_a.pdf}
        \label{fig:postfit_leading_lep_pt}
    }
    \hfill
    \subfloat[]{
        \includegraphics[width=0.32\linewidth]{figures/llvv/fig_postfit_b.pdf}
        \label{fig:postfit_met}
    }
    \hfill
    \subfloat[]{
        \includegraphics[width=0.32\linewidth]{figures/llvv/fig_postfit_c.pdf}
        \label{fig:postfit_dphi_ll}
    }
    
    \caption{Post-fit kinematic distributions for the inclusive $ZZ$ SR
    (a) the transverse momentum of the leading lepton $p_\text{T}^\text{leading lepton}$, 
    (b) $E_\text{T}^\text{miss}$, and 
    (c) the azimuthal angle separation between two charged leptons $\Delta\phi(\ell, \ell)$. 
    The data points are shown with statistical error bars, while the shaded band represents the total post-fit uncertainty in the prediction, combining statistical and systematic uncertainties. Open markers indicate data points lying outside the vertical range of the plot.}
    \label{fig:postfit_observed}
\end{figure}

\subsection{Integrated Cross-Section Measurements}
\label{subsec:integrated_xsec}

The measured signal strength is converted into a fiducial cross-section using the SM predictions from \textsc{Sherpa} and \textsc{MadGraph5}. 

The measured fiducial cross-section for the inclusive $ZZ \to \ell\ell\nu\nu$ process is:
\begin{equation}
    \sigma_{ZZ \to \ell\ell\nu\nu}^{\text{fid}} = 21.0 \pm 1.0~\text{fb}.
\end{equation}
The total uncertainty is composed of $\pm 0.73$ (stat), $\pm 0.57$ (exp), $\pm 0.18$ (theory), and $\pm 0.18$ (lumi) fb. This result represents a significant improvement in precision compared to previous ATLAS measurements using partial Run 2 data, with the total uncertainty reduced from 7.0\% to approximately 4.8\%. 

The measurement is consistent with the state-of-the-art SM prediction calculated using \textsc{Matrix} at NNLO in QCD and NLO in EW corrections, which yields $21.03 \pm 0.30$ fb.

Using the fiducial acceptance factor $A_{ZZ}$ and the branching ratio for $ZZ \to 2\ell 2\nu$, the result is extrapolated to the full phase space for $Z$ boson masses between 66 and 116 GeV. The total $pp \to ZZ$ cross-section is determined to be:
\begin{equation}
    \sigma_{pp \to ZZ}^{\text{total}} = 15.38 \pm 0.81~\text{pb},
\end{equation}
which agrees well with the \textsc{Matrix} prediction of $15.40 \pm 0.38$ pb.

\subsection{Differential Cross-Sections}
\label{subsec:diff_xsec}

Differential cross-sections are measured to probe the kinematic properties of the inclusive $ZZ$ production and test perturbative QCD predictions. The detector-level distributions are unfolded to the stable particle level using an iterative Bayesian method to correct for detector resolution and acceptance effects.

Eight kinematic variables were measured for the inclusive channel. Figure~\ref{fig:diff_xs_ZZ} shows the differential cross-section as a function of the transverse momentum of the $Z$ boson ($p_{\mathrm{T}}^Z$). The data shows good agreement with both the \textsc{Sherpa} and \textsc{Matrix} predictions, though the \textsc{Matrix} calculation (incorporating NNLO QCD) tends to describe the normalization slightly better.


\begin{figure}[!hbtp]
    \centering
    % Row 1
    \subfloat[]{
        \includegraphics[width=0.32\linewidth]{figures/llvv/fig_diff_llvv_a.pdf}
        \label{fig:diff_llvv_lep_pt}
    }
    \hfill
    \subfloat[]{
        \includegraphics[width=0.32\linewidth]{figures/llvv/fig_diff_llvv_b.pdf}
        \label{fig:diff_llvv_z_pt}
    }
    \hfill
    \subfloat[]{
        \includegraphics[width=0.32\linewidth]{figures/llvv/fig_diff_llvv_c.pdf}
        \label{fig:diff_llvv_dphi}
    }
    \\ % Line break for Row 2
    \subfloat[]{
        \includegraphics[width=0.32\linewidth]{figures/llvv/fig_diff_llvv_d.pdf}
        \label{fig:diff_llvv_y_z}
    }
    \hfill
    \subfloat[]{
        \includegraphics[width=0.32\linewidth]{figures/llvv/fig_diff_llvv_e.pdf}
        \label{fig:diff_llvv_pt_zz}
    }
    \hfill
    \subfloat[]{
        \includegraphics[width=0.32\linewidth]{figures/llvv/fig_diff_llvv_f.pdf}
        \label{fig:diff_llvv_mt_zz}
    }
    \\ % Line break for Row 3
    \subfloat[]{
        \includegraphics[width=0.32\linewidth]{figures/llvv/fig_diff_llvv_g.pdf}
        \label{fig:diff_llvv_njets}
    }
    \hfill
    \subfloat[]{
        \includegraphics[width=0.32\linewidth]{figures/llvv/fig_diff_llvv_h.pdf}
        \label{fig:diff_llvv_cp}
    }
    % Filler to align the last row to the left if needed, or center (default)
    
    \caption{Differential cross-section measurements in the inclusive $ZZ\rightarrow \ell\ell\nu\nu$ fiducial phase space, compared with MC predictions calculated with the nominal \textsc{Sherpa} simulation and with \textsc{Matrix} for a set of kinematic variables:
    (a) $p_\text{T}^\text{leading lepton}$,
    (b) $p_\text{T}^\text{Z}$,
    (c) $\Delta\phi(\ell, \ell)$,
    (d) $|y^Z|$,
    (e) $p_\text{T}^\text{ZZ}$,
    (f) $m_\text{T}^\text{ZZ}$,
    (g) $N_{\text{jets}}$, and
    (h) a CP-sensitive angular variable, $\sin(\varphi)\cos(\theta)$.
    The statistical and total uncertainties in the data points are displayed in both the differential cross-sections and the corresponding ratios. The total theory uncertainties in the \textsc{Sherpa} and \textsc{Matrix} predictions are represented as shaded bands in the cross-section panels and as error bars in the ratio panels.}
    \label{fig:diff_xs_ZZ}
\end{figure}

\subsection{Constraints on Anomalous Neutral Triple Gauge Couplings}
\label{subsec:anom_couplings}

The inclusive $ZZ$ production is sensitive to anomalous $ZZZ$ and $ZZ\gamma$ couplings (aTGCs), which are forbidden at tree level in the SM. Constraints are derived by fitting the reconstructed $p_{\mathrm{T}}$ distribution of the dilepton system, as no significant deviations from the SM predictions are observed in the kinematic tails.

Limits are set using two frameworks: the vertex function (VF) formalism and the SM Effective Field Theory (SMEFT). The 95\% confidence level (CL) intervals for the VF parameters $f_4^{\gamma/Z}$ (CP-violating) and $f_5^{\gamma/Z}$ (CP-conserving) are presented in Table~\ref{tab:atgc_limits}. These results improve upon the previous ATLAS $ZZ \to \ell\ell\nu\nu$ measurement by a factor of approximately 1.7.

\begin{table}[htbp]
    \centering
    \caption{Observed and expected 95\% CL intervals for the neutral aTGC parameters in the vertex function formalism.}
    \label{tab:atgc_limits}
    \begin{tabular}{lcc}
        \hline
        Parameter & Expected ($10^{-4}$) & Observed ($10^{-4}$) \\
        \hline
        $f_4^{\gamma}$ & $[-7.7, 7.7]$ & $[-6.4, 6.4]$ \\
        $f_4^{Z}$ & $[-6.8, 6.8]$ & $[-5.7, 5.7]$ \\
        $f_5^{\gamma}$ & $[-7.8, 7.7]$ & $[-6.4, 6.5]$ \\
        $f_5^{Z}$ & $[-6.9, 6.9]$ & $[-5.8, 5.8]$ \\
        \hline
    \end{tabular}
\end{table}

\subsection{Summary of Results}

In summary, the fiducial and differential cross-sections for the inclusive $ZZ \to \ell\ell\nu\nu$ process have been measured with high precision using the full ATLAS Run 2 dataset. The inclusive fiducial cross-section is measured with a precision of 4.8\%, limited by statistics. All measurements show excellent agreement with Standard Model predictions. Stringent limits have been placed on anomalous triple gauge couplings, significantly constraining the phase space for new physics in the electroweak sector.